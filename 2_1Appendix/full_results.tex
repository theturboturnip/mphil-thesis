\chapter{Full test results\label{chap:fullresults}}

\section{Initial Smoke Tests}
Some simple smoke tests were constructed to test the basic functionality of the emulator, particularly under CHERI.
\code{hello\_world} runs three small functions which calculate Fibonacci numbers and factorials.
Fibonacci is calculated with a simple recursive function, and with \emph{memoization} where previous outputs are cached in a static array.
The tests compile on all compilers, and output the correct results on all architectures.
\begin{table}[h]
    \centering
    \CatchFileDef{\tablehelloworld}{1_50Evaluation/data/hello_world_rows.tex}{}
    \begin{tabular}{rcccccc}
    \tablehelloworld
    \end{tabular}
    \caption{\code{hello\_world} results --- Basic program tests}\label{tab:fullresults:helloworld}
\end{table}


\section{\code{vector\_memcpy}}
The scope of this test, including testing many permutations of \code{vtype}, meant the full table couldn't be included in the main paper.
\begin{longtable}{rcccccc}
\caption{Results --- Vectorized memcpy}\label{tab:fullresults:vectormemcpy}\\
\toprule
& RV32 & \multicolumn{5}{c}{RV-64} \\
\cmidrule(lr){2-2} \cmidrule(lr){3-7}
& \code{llvm-13} & \code{llvm-13} & \code{llvm-15} & \code{gcc} & CHERI & CHERI (Int) \\
\midrule
\endhead
\bottomrule
\endfoot
\endlastfoot
Unit Stride e8m1 & Y & Y & Y & Y & Y & Y\\
Unit Stride e16m2 & Y & Y & Y & Y & Y & Y\\
Unit Stride e32m4 & Y & Y & Y & Y & Y & Y\\
Unit Stride e64m8 & Y & Y & Y & Y & Y & Y\\
Unit Stride e32mf2 & Y & Y & Y & - & Y & Y\\
Unit Stride e16mf4 & Y & Y & Y & - & Y & Y\\
Unit Stride e8mf8 & Y & Y & Y & - & Y & Y\\
Strided e8m1 & Y & Y & Y & Y & Y & Y\\
Strided e16m2 & Y & Y & Y & Y & Y & Y\\
Strided e32m4 & Y & Y & Y & Y & Y & Y\\
Strided e64m8 & Y & Y & Y & Y & Y & Y\\
Strided e32mf2 & Y & Y & Y & - & Y & Y\\
Strided e16mf4 & Y & Y & Y & - & Y & Y\\
Strided e8mf8 & Y & Y & Y & - & Y & Y\\
Indexed e8m1 & Y & Y & Y & Y & Y & Y\\
Indexed e16m2 & Y & Y & Y & Y & Y & Y\\
Indexed e32m4 & Y & Y & Y & Y & Y & Y\\
Indexed e64m8 & Y & Y & Y & Y & Y & Y\\
Indexed e32mf2 & Y & Y & Y & - & Y & Y\\
Indexed e16mf4 & Y & Y & Y & - & Y & Y\\
Indexed e8mf8 & Y & Y & Y & - & Y & Y\\
Unit Stride Masked e8m1 & Y & Y & Y & Y & Y & Y\\
Unit Stride Masked e16m2 & Y & Y & Y & Y & Y & Y\\
Unit Stride Masked e32m4 & Y & Y & Y & Y & Y & Y\\
Unit Stride Masked e64m8 & Y & Y & Y & Y & Y & Y\\
Unit Stride Masked e32mf2 & Y & Y & Y & - & Y & Y\\
Unit Stride Masked e16mf4 & Y & Y & Y & - & Y & Y\\
Unit Stride Masked e8mf8 & Y & Y & Y & - & Y & Y\\
Bytemask Load e8m1 & - & - & Y & - & - & -\\
Bytemask Load e16m2 & - & - & Y & - & - & -\\
Bytemask Load e32m4 & - & - & Y & - & - & -\\
Bytemask Load e64m8 & - & - & Y & - & - & -\\
Bytemask Load e32mf2 & - & - & Y & - & - & -\\
Bytemask Load e16mf4 & - & - & Y & - & - & -\\
Bytemask Load e8mf8 & - & - & Y & - & - & -\\
Unit Stride Segmented e8m2 & Y & Y & Y & Y & Y & Y\\
Unit Stride Segmented e16m2 & Y & Y & Y & Y & Y & Y\\
Unit Stride Segmented e32m2 & Y & Y & Y & Y & Y & Y\\
Unit Stride Segmented e64m2 & Y & Y & Y & Y & Y & Y\\
Unit Stride Segmented e32mf2 & Y & Y & Y & - & Y & Y\\
Whole-Register e64m1 & Y & Y & Y & Y & Y & Y\\
Whole-Register e64m2 & Y & Y & Y & Y & Y & Y\\
Whole-Register e64m4 & Y & Y & Y & Y & Y & Y\\
Whole-Register e64m8 & Y & Y & Y & Y & Y & Y\\
FoF Memcpy e8m1 & Y & Y & Y & Y & Y & Y\\
FoF Memcpy e16m2 & Y & Y & Y & Y & Y & Y\\
FoF Memcpy e32m4 & Y & Y & Y & Y & Y & Y\\
FoF Memcpy e64m8 & Y & Y & Y & Y & Y & Y\\
FoF Memcpy e32mf2 & Y & Y & Y & - & Y & Y\\
FoF Memcpy e16mf4 & Y & Y & Y & - & Y & Y\\
FoF Memcpy e8mf8 & Y & Y & Y & - & Y & Y\\
FoF Boundary e8m1 & Y & Y & Y & Y & Y & Y\\
FoF Boundary e16m2 & Y & Y & Y & Y & Y & Y\\
FoF Boundary e32m4 & Y & Y & Y & Y & Y & Y\\
FoF Boundary e64m8 & Y & Y & Y & Y & Y & Y\\
FoF Boundary e32mf2 & Y & Y & Y & - & Y & Y\\
FoF Boundary e16mf4 & Y & Y & Y & - & Y & Y\\
FoF Boundary e8mf8 & Y & Y & Y & - & Y & Y\\
\bottomrule
\end{longtable}


\pagebreak
\section{\code{vector\_memcpy\_pointers}}
This is already referenced in the main paper (\cref{chap:capinvec:eval}) and included here for completeness.\\


\CatchFileDef{\tablevecmemcpypointers}{1_50Evaluation/data/vector_memcpy_pointers_rows}{}
\begin{tabular}{rcccccc}
\tablevecmemcpypointers
\end{tabular}
