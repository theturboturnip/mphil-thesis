\chapter{Fast path vector checks - Pre-checks}\label{appx:fastpathfull}
This appendix describes methods of calculating tight bounds for vector memory accesses (\cref{chap:hardware:subsec:checking}) and ways that architectural complexity can be traded off to calculate \emph{wider} bounds.
These methods calculate the entire bounds up front, and while they are used in the emulator a hardware implementation may find it introduces too much latency.
\todomark{Justify appendix existence}
\todomark{reference Success, Indeterminate, Failure cases}
\todomark{fast-path could be split up? i.e. for LMUL = 8, could execute a fast-path for each register in the group rather than all 8 at once}

\newcommand{\vstart}{\code{vstart}}
\newcommand{\vstartactive}{\code{vstart}_{\mathit{active}}}
\newcommand{\evl}{\code{evl}}
\newcommand{\evlactive}{\code{evl}_{\mathit{active}}}
\newcommand{\baseaddr}{\mathit{base}}


\section{Masked accesses}
For all masked accesses, masked-out/inactive segments should not trigger capability exceptions.
Therefore, a tight bounds must include only the smallest and largest active segments.
These segments can be found by inspecting the mask vector: either checking each bit in turn or using parallel logic to find the lowest/highest set bits.
Care must be taken with these checks to ensure elements outside the range $[\textit{vstart}, \textit{evl})$ are not counted.

\begin{align}
    \vstartactive{} &= \min(i\ \forall\ \vstart{} \le i < \evl{}\ \text{where}\ \mathit{mask}[i] = 1) \\
    \evlactive{}    &= \max(i\ \forall\ \vstart{} \le i < \evl{}\ \text{where}\ \mathit{mask}[i] = 1) + 1
\end{align}

\subsection*{Tradeoffs}
If using parallel logic to find the lowest/highest bits, it could be difficult to account for $[\textit{vstart}, \textit{evl})$.
An implementation could choose to only calculate tight bounds when the mask is fully utilized, i.e. $\textit{vstart} = 0, \textit{evl} = \textit{VLEN}$, and assume wider bounds otherwise.

Accounting for masked accesses at all may not be worth the extra complexity.
Only elements masked off on the edges make any difference, and it may be uncommon for long runs of edge elements to be masked off.
Thus, an implementation could choose to ignore masking entirely when computing the ranges.
This does mean that all failures become Likely-Failure when masking is enabled, because all elements outside the capability bounds may be masked off.

\todomark{iterating over elements may still be more energy-efficient than doing individual capability checks?}

\section{Unit accesses}
For unit segmented accesses, which includes fault-only first, the tight address range for an access is simple to calculate.
Whole register and bytemask accesses can simplify this by fixing \code{nf = 1} and \code{eew = 8}.

\begin{equation}
    \baseaddr{} + [\vstartactive{} * \code{nf} * \code{eew}, \evlactive{} * \code{nf} * \code{eew})
\end{equation}
\todomark{Note that the equation is a simplification of strided for \code{stride = nf * eew}}

\subsection*{Tradeoffs}
\code{nf} is not guaranteed to be a power of two (except for the whole-register case), so calculating the `tight' address range would require a multiplication by an arbitrary four-bit value between 1 and 8.
If this multiplication is too expensive, implementations could choose to classify all $\code{nf} > 1$ cases as Unchecked.
% \todomark{if nf == 1; as normal; if nf != 1 Unchecked}

Unless extra restrictions are placed on \code{vstart}, calculating the start of this range requires another arbitrary multiplication.
To avoid this one could assume \code{vstart = 0} and treat failures as Likely-Failure for other cases.
%for the calculation, perform the capability check and treat failures as Likely-Failure (because the failure could be due to an element before \code{vstart}).
% \todomark{make it clear that in this case vstart = 0 would still be a complete failure}
Once could also classify all nonzero \code{vstart} accesses as Unchecked.
% \todomark{if vstart == 0 failure = failure; if vstart != 0 all = Likely-Failure}

Even if the previous two optimizations are applied, the final range still requires a multiplication $\evl{} * \code{eew}$.
Thankfully, because \code{eew} may only be one of four powers-of-two, this can be encoded as a simple shift.

\section{Strided accesses}
Strided accesses bring further complication, especially as the stride may be negative.

\begin{equation}
\mathit{bounds}(\code{stride}) = \baseaddr{} + \left\{
    \begin{array}{ll}
          [\vstartactive{} * \code{stride}, (\evlactive{} - 1) * \code{stride} + \code{nf} * \code{eew}) & \code{stride} \ge 0 \\
          
          [(\evlactive{} - 1) * \code{stride}, \vstartactive{} * \code{stride} + \code{nf} * \code{eew}) & \code{stride} < 0
    \end{array} 
\right.
\end{equation}

This is formed of three components:
\begin{itemize}
    \item $\vstartactive{} * \code{stride}$, the start of the first segment. This can be simplified to 0, just like for unit accesses, to avoid an arbitrary multiplication.
    \item $(\evlactive{} - 1) * \code{stride}$, the start of the final segment. This requires an arbitrary multiplication, unless strided accesses are all Unchecked.
    \item $\code{nf} * \code{eew}$, the length of a segment, which can be implemented with a shift.
\end{itemize}
% Because the stride is not guaranteed to be larger than a segment (or even larger than a single element)

% \todomark{stride may be negative}

\section{Indexed accesses}
This is the most complicated access of the bunch, because the addresses cannot be computed without reading the index register.

\begin{align}
    [&\baseaddr{}\ +\ \min(\code{offsets}[\vstartactive{}..\evlactive{}]), \\
    &\baseaddr{}\ +\ \max(\code{offsets}[\vstartactive{}..\evlactive{}])\ +\ \code{nf} * \code{eew})
\end{align}

The most expensive components here are of course $min,\,max$ of the offsets.
These could be calculated in hardware through parallel reductions, making it slightly more efficient than looping over each element.
A low-hanging optimization could be to remove the $\vstartactive{}..\evlactive{}$ range condition, performing the reduction over the whole register group, which would make failures Likely-Failure where $\vstartactive{}\ !=\ 0\ ||\ \evlactive{}\ !=\ \code{VLMAX}$.
This calculation could also be restricted to certain register configurations to reduce the amount of required hardware.
Indeed, the amount of hardware could be reduced to zero by simply classifying all indexed accesses as Unchecked.