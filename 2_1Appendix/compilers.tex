\chapter{Compiler information}

\section{Vanilla RVV command-line options\label{appx:vector_command_line}}
\begin{table}[h]
    \centering
\begin{tabularx}{\linewidth}{llX}
    \toprule
    Compiler & Required Arguments & Notes \\ 
    \midrule
    Clang-13 & \code{-march=rv64gv0p10}  & \multirow[t]{2}{=}{Supports intrinsics, inline assembly for RVV v0.1} \\
    & \code{-menable-experimental-extensions} & \\
    \midrule
    Clang-14+ & \code{-march=rv64gv} & Supports intrinsics, inline assembly for RVV v1.0 \\
    \midrule
    GCC 10.1 & \code{-march=rv64g_v} & Requires special toolchain (see \cref{appx:building_rvv_gcc_toolchain}) and has incomplete support (see \cref{compilerdifferences}) \\
    \bottomrule
\end{tabularx}
    \caption{Command-line arguments for compiling RVV code on non-CHERI compilers\\(assuming the base ISA is \code{rv64g})}
    \label{tab:rvv_cmdline_nocheri}
\end{table}

\section{CHERI-RVV command-line options}\label{chericlang_cmdline}
\begin{table}[h]
    \centering
\begin{tabularx}{\linewidth}{llX}
    \toprule
    Compiler & Required Arguments & Notes \\ 
    \midrule
    \multicolumn{1}{c}{CHERI} & \code{-march=rv64gv0p10xcheri} & \multirow[t]{2}{=}{Supports intrinsics, inline assembly for RVV v0.1} \\
    Clang-13 & \code{-menable-experimental-extensions} & \\
    & \code{-mabi=l64pc128} & ABI string sets capability width. \\
    & \code{-mno-relax} & Must disable linker relaxations. \\
    \bottomrule
\end{tabularx}
    \caption{Command-line arguments for compiling CHERI-RVV code\\(assuming the base ISA is \code{rv64g})}
    \label{tab:rvv_cmdline_cheri}
\end{table}

By default CHERI-Clang doesn't actually compile capability-enabled code.
The documentation on enabling capabilities is unfortunately sparse and outdated.
In particular, the CHERI-Clang help menu states that \code{--cheri} will \enquote{Enable CHERI support with the default capability size}, but this has no effect (at least on RISC-V).
To find up-to-date answers, we consulted the source code for the CHERIbuild build tool\footnote{\gitrepo{CSTRD-CHERI/cheribuild}{https://github.com/CTSRD-CHERI/cheribuild}}.

CHERIbuild's code%
\footnote{%
\gitfile{config/compilation_targets.py:176}%
    {CSTRD-CHERI/cheribuild}%
    {https://github.com/CTSRD-CHERI/cheribuild/blob/ba3a0b6388436224968c906192c61d2ccbdd7616/pycheribuild/config/compilation_targets.py\#L176}%
} revealed three requirements:
\begin{itemize}
    \item The architecture string must contain \code{xcheri}
    \item The capability length must be set using the ABI string
    \begin{itemize}
        \item In pure-capability mode, pointers and capabilities are \code{CLEN} long
        \begin{itemize}
            \item Example string: \code{l64pc128}
            \item Integer width (\code{long}, or \code{l}) = XLEN = 64-bits
            \item Pointer width (\code{p}) = Capability width (\code{p}) = CLEN = 128-bits
        \end{itemize}
        \item For hybrid mode, pointers remain \code{XLEN} long and capability length is not specified
        \begin{itemize}
            \item Example string: \code{lp32}
            \item Integer width (\code{l}) = XLEN = Pointer width (\code{p}) = 32-bits
        \end{itemize}
    \end{itemize}
    \item ``Linker relaxations'', where function calls are converted to short jumps\cite{chenCompilerSupportLinker2019}, must be disabled.
    \begin{itemize}
        \item This is likely because CHERI requires function calls to go through capabilities
        \item However the code that adds this option wasn't documented, so there may be more to it
    \end{itemize}
\end{itemize}

Once the above options are set, plain CHERI-RISC-V code compiles without a hitch.
Changes to CHERI-Clang itself are required to compile vectors (\cref{addingtochericlang}).

\section{Compiler support for RVV}\label{compilerdifferences}
While most compilers support all memory access archetypes, there are a few notable exceptions.
GCC has the most: there is no support for fractional LMUL or bytemask accesses, the intrinsics for segmented accesses are named differently, and fault-only-first intrinsics emit incorrect instructions\footnote{On GCC, fault-only-first intrinsics seem to emit \code{vsetvli}.}.
GCC RVV support has been deprioritized in favor of LLVM\footnote{\url{https://github.com/riscv-collab/riscv-gcc/issues/320}}, so the rough edges make sense.
LLVM-13-based compilers (including CHERI-Clang) support all specified archetypes except bytemask accesses.
CHERI-Clang doesn't support intrinsics, but all inline assembly support is intact.
Support for bytemask accesses is only available in LLVM-14 and up.

\pagebreak
\section{Ensuring compatibility between different compilers}
The \code{vector\_memcpy} test program uses the preprocessor to identify the current compiler and how that compiler supports various vector instructions.
It is reproduced here in case it can be useful for other vector-agnostic programs.

\inputframedminted[breaklines=true]{c}{./code/compatibility_snippet.c}


\pagebreak
\section{Building \code{riscv-gnu-toolchain} with vector support}\label{appx:building_rvv_gcc_toolchain}

As of May 2022, the RISC-V GNU toolchain (hosted at \gitrepo{riscv-collab/riscv-gnu-toolchain}{https://github.com/riscv-collab/riscv-gnu-toolchain/}) does not support the vector extension or it's intrinsics.
The \code{rvv-intrinsic} branch of this repository claimed to support vector intrinsics, but it was slightly outdated and has been deleted as of 17th May 2022.
It referenced a repository for \code{glibc} that no longer exists as a submodule, which makes compilation impossible.
We have archived this branch online (\redact{\gitrepo{theturboturnip/riscv-gnu-toolchain}{https://github.com/theturboturnip/riscv-gnu-toolchain}}) and fixed that issue.
This appendix describes how to build the toolchain.

To build the full toolchain with intrinsic support, perform the following steps (derived by the author independently, then amended based on macOS instructions from~\footnote{\url{https://github.com/riscv-collab/riscv-gcc/issues/323}}):
\begin{enumerate}
    % \todomark{Example}
    % \begin{itemize}
    %     \item Alternatively, this can be pointed at \url{https://github.com/riscvarchive/riscv-glibc.git}, the archived version of the old repository\todocite{macOS GCC instructions}.
    % \end{itemize}
    \item Clone the repository itself:
    \item[\code{\$}] \code{git clone \redact{https://github.com/theturboturnip/riscv-gnu-toolchain}}
    \item Clone the \code{riscv-gcc} submodule:
    \item[\code{\$}] \code{git submodule update --init --recursive --progress --force ./riscv-gcc}
    \begin{itemize}
        \item On macOS, it may be necessary to disable SSL:
        \item[\code{\$}] \code{git -c http.sslVerify=false submodule ...}
    \end{itemize}
    \item Configure the compiler so it supports all General extensions, Compressed instructions, and Vector extension:
    \item[\code{\$}] \code{./configure --prefix=<output directory> --with-arch=rv64gcv --with-abi=lp64d}
    \item Build the \code{newlib} version to compile for bare-metal platforms:
    \item[\code{\$}] \code{make newlib -j\$(nproc)}
\end{enumerate}