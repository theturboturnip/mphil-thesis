\chapter{CHERI-RVV changes from CHERI and RVV}

This appendix summarizes the differences between a CHERI-RVV ISA, as emulated by \code{riscv-v-lite}, and a CHERI-RISC-V ISA with the vanilla RVV extension.

\section{Loading/storing with capabilities}
All CHERI-RISC-V instructions are unchanged.
The RVV memory access instructions (listed in \cref{chap:bg:sec:rvvmemory}) have their behaviour changed, and all other RVV memory access instructions are unchanged.

In capability mode, vector memory accesses are changed to use capabilities.
The \code{rs1} field in the encoding for all memory access instructions is changed to \code{cs1}, which specifies the capability register holding the \emph{base capability}.
In integer mode, the value held in register \code{rs1} is added to the DDC to create the \emph{base capability}.
The base capability's \emph{cursor} is used as the base address for all accesses.

The behaviour of each vector memory access is unchanged, except for exception behaviour.
A synchronous exception is raised on element \code{i} of width \code{eew} at address \code{addr} if
\begin{itemize}
    \item The base capability tag is not set.
    \item The base capability is sealed.
    \item The base capability does not have adequate permissions.
    \begin{itemize}
        \item e.g. \textsc{Permit\_Load} for loads,
        \item[] \textsc{Permit\_Store} for stores.
    \end{itemize}
    \item \code{addr < base\_capability.base}.
    \item \code{addr + eew > base\_capability.top}.
\end{itemize}
All other exceptions that would be raised by a vanilla RVV equivalent, e.g. unaligned access exceptions, are also raised.
Fault-only-first loads silently swallow capability-related exceptions when \code{i > 0}, setting \code{vl} instead of taking a trap, just like all other synchronous exceptions.

In a future version, it may be desirable for some capability exceptions to trap before any accesses are attempted.\todomark{Make sure this is made clear in main content}
For example, passing a sealed, untagged, or permissionless capability to a memory access usually reflects a serious programming error, and should not be ignored in any case.
The current emulator would ignore those errors if all elements were masked out, and would silently swallow them in fault-only-first.
Element-specific exceptions, i.e. bounds violations, should still always be swallowed by fault-only-first.

% \todomark{CHERI-RVV changes RVV instructions to use capabilities as address references in all places, no other behaviour changes}

% \todomark{CHERI-RVV treats capability out-of-bounds for an element as a synchronous exception, permission checks for a vector access are treated as a synchronous exception on element 0.}
% \todomark{The above means fault-only-first will silently swallow capability permission errors if element 0 is masked out.}

% \todomark{CHERI-RVV doesn't change CHERI behaviour}

\section{Capabilities-in-vectors changes}\label{appx:capinvec}
\code{vsetvl} instructions are modified to accept a \code{SEW} value mapping to 128-bit elements (\cref{tab:capinvec:vtypewidth}).
The arithmetic RVV instructions implemented by the emulator, listed in \cref{rvvlite_arithmetic}, are modified to handle 128-bit elements.
We have not investigated changing any other instructions, but believe it should be trivial to extend them.
The RVV specification notes how some instructions would handle 128-bit elements, e.g. \cite[Chapter 13]{specification-RVV-v1.0}.

The RVV memory access instructions are changed to support a new element width (\cref{tab:capinvec:accesswidth}).
This should be supported in all instructions, but was only tested for unit loads and stores.
Indeed, the only instructions added to CHERI-Clang were \code{vle128.v} and \code{vse128.v}.
Memory access instructions of 128-bit elements are the only instructions that access vector registers in a capability context.

Of the non-unit loads and stores, indexed memory accesses are potentially concerning.
These accesses use offsets of the \code{vtype}-encoded width, so could try to use 128-bit offsets.
Indexed memory accesses have not been tested with 128-bit offsets, and further work is required to decide how that case should be handled.\todomark{Make sure this is made clear in main content}

% \todomark{Adds unit load/store 128, element width 128}

\begin{table}[h]
    \centering
    \begin{minipage}[c]{.49\textwidth}
        \centering
    \begin{tabular}{lx{0.5cm}x{0.5cm}x{0.5cm}}
        \multicolumn{1}{c}{SEW} & \multicolumn{3}{c}{\code{vsew[2:0]}} \\
        \cmidrule(lr){2-4}
        8 & 0 & 0 & 0 \\
        16 & 0 & 0 & 1 \\
        32 & 0 & 1 & 0 \\
        64 & 0 & 1 & 1 \\
        128$^{\text{new}}$ & 1 & 0 & 0 \\
    \end{tabular}
    \captionof{table}{Selected element width encoding}
    \label{tab:capinvec:vtypewidth}
\end{minipage}\hfill
\begin{minipage}[c]{.49\textwidth}
    \centering
    \begin{tabular}{lcx{0.5cm}x{0.5cm}x{0.5cm}}
        Access Type & \code{mew} & \multicolumn{3}{c}{\code{width[2:0]}} \\
        \cmidrule(lr){2-2} \cmidrule(lr){3-5}
        Vector(8) & 0 & 0 & 0 & 0 \\
        Vector(16) & 0 & 1 & 0 & 1 \\
        Vector(32) & 0 & 1 & 1 & 0 \\
        Vector(64) & 0 & 1 & 1 & 1 \\
        Vector(128)$^{\text{new}}$ & 1 & 0 & 0 & 0 \\
    \end{tabular}
    \captionof{table}{Width encoding for vector loads and stores}\label{tab:capinvec:accesswidth}
\end{minipage}
\end{table}


\begin{table}
    \centering
    \begin{tabular}{l}
        \code{vmv.v.v} \\
        \code{vmv.v.i} \\
        \code{vmerge.vim} \\
        \code{vmv<nr>r.v} \\
        \code{vmseq.vi} \\
        \code{vmsne.vi} \\
        \code{vadd.v.i} \\
    \end{tabular}
    \caption{\code{riscv-v-lite} supported arithmetic instructions}\label{rvvlite_arithmetic}
\end{table}


\pagebreak
\subsection{Relevant properties}\label{appx:capinvec:properties}

This subsection summarizes the properties of the emulator described in \cref{chap:capinvec} that enable capabilities-in-vectors.
These properties are not absolute requirements for all capability-in-vector implementations fulfilling \cref{hyp:cap_in_vec_storage,hyp:cap_in_vec_load_store,hyp:cap_in_vec_manip}, but can be used as a starting point.

\begin{itemize}
    \item \code{ELEN} = 128(+1) i.e. the length of a capability.
    \begin{itemize}
        \item For a program to manipulate, load, and store capability values securely and atomically, it must be able to operate on appropriately sized logical elements.
    \end{itemize}
    \item \code{VLEN} = 128(+1) i.e. the length of a capability.
    \begin{itemize}
        \item $\code{VLEN} \ge \code{ELEN}$ (\cite[Chapter 2]{specification-RVV-v1.0})
        \item Larger \code{VLEN} could be supported, which must be a power-of-two\cite[Chapter 2]{specification-RVV-v1.0} and therefore will be a multiple of the capability length.
    \end{itemize}
    \item The memory interface is identical to that used by the scalar processor, where each vector access is split into a set of sequential accesses less than or equal to 128 bits.
    \begin{itemize}
        \item Therefore the safety properties for the scalar code still hold.
        \item Example: capabilities can only be stored through a capability with the \code{STORE\_CAP} permission.
    \end{itemize}
    \item Capability memory accesses use \code{SafeTaggedCap} as a unit.
    \begin{itemize}
        \item This means it is impossible to set the tag bit on invalid capabilities.
        \item It also means all capability accesses are 128-bit aligned, and atomic.
    \end{itemize}
    \pagebreak
    \item The only instruction that can place capabilities in a vector register is 128-bit element loads.
    \begin{itemize}
        \item There are no vectorized capability-to-capability or integer-to-capability instructions, such as \code{CSetBounds} or \code{CFromPtr}.
        \item All other instructions that write to registers (e.g. arithmetic, 64-bit loads, etc.) unset the tag bit.
        \item The only way to place a valid capability in a vector register is to copy a valid capability from memory.
        \item Therefore the Provenance and Monotonicity properties are always upheld.
    \end{itemize}
    \item The only 128-bit element vector load instruction is unit-stride.
    \begin{itemize}
        \item Strided and Indexed accesses could be supported as long as they enforced alignment and atomicity correctly.
        \item Whole-register accesses would need to be updated to always use 128-bit elements, in case capabilities are being accessed.
    \end{itemize}
    \item Capabilities-in-vectors cannot be dereferenced directly.
    \begin{itemize}
        \item Therefore Integrity cannot be violated by vector operations.
    \end{itemize}
\end{itemize}