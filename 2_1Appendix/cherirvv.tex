\chapter{CHERI-RVV changes from CHERI and RVV}

\todomark{CHERI-RVV changes RVV instructions to use capabilities as address references in all places, no other behaviour changes}

\todomark{CHERI-RVV treats capability out-of-bounds for an element as a synchronous exception, permission checks for a vector access are treated as a synchronous exception on element 0.}
\todomark{The above means fault-only-first will silently swallow capability permission errors if element 0 is masked out.}

\todomark{CHERI-RVV doesn't change CHERI behaviour}

\section{Capabilities-in-vectors}\label{appx:capinvec}
\todomark{Adds unit load/store 128, element width 128}

\begin{table}[h]
    \centering
    \begin{minipage}[c]{.49\textwidth}
        \centering
    \begin{tabular}{lx{0.5cm}x{0.5cm}x{0.5cm}}
        \multicolumn{1}{c}{SEW} & \multicolumn{3}{c}{\code{vsew[2:0]}} \\
        \cmidrule(lr){2-4}
        8 & 0 & 0 & 0 \\
        16 & 0 & 0 & 1 \\
        32 & 0 & 1 & 0 \\
        64 & 0 & 1 & 1 \\
        128$^{\text{new}}$ & 1 & 0 & 0 \\
    \end{tabular}
    \captionof{table}{Selected element width encoding}
    \label{tab:capinvec:vtypewidth}
\end{minipage}\hfill
\begin{minipage}[c]{.49\textwidth}
    \centering
    \begin{tabular}{lcx{0.5cm}x{0.5cm}x{0.5cm}}
        Access Type & \code{mew} & \multicolumn{3}{c}{\code{width[2:0]}} \\
        \cmidrule(lr){2-2} \cmidrule(lr){3-5}
        Vector(8) & 0 & 0 & 0 & 0 \\
        Vector(16) & 0 & 1 & 0 & 1 \\
        Vector(32) & 0 & 1 & 1 & 0 \\
        Vector(64) & 0 & 1 & 1 & 1 \\
        Vector(128)$^{\text{new}}$ & 1 & 0 & 0 & 0 \\
    \end{tabular}
    \captionof{table}{Width encoding for vector loads and stores}\label{tab:capinvec:accesswidth}
\end{minipage}
\end{table}


\subsection{Relevant properties}\label{appx:capinvec:properties}

This subsection summarizes the properties of the emulator described in \cref{chap:capinvec} than enable capabilities-in-vectors.
These properties are not absolute requirements for all capability-in-vector implementations fulfilling \cref{hyp:cap_in_vec_storage,hyp:cap_in_vec_load_store,hyp:cap_in_vec_manip}, but can be used as a starting point.

\begin{itemize}
    \item \code{ELEN} = 128(+1) i.e. the length of a capability.
    \begin{itemize}
        \item For a program to manipulate, load, and store capability values securely and atomically, it must be able to operate on appropriately sized logical elements.
    \end{itemize}
    \item \code{VLEN} = 128(+1) i.e. the length of a capability.
    \begin{itemize}
        \item $\code{VLEN} \ge \code{ELEN}$ (\cite[Chapter 2]{specification-RVV-v1.0})
        \item Larger \code{VLEN} could be supported, which must be a power-of-two\cite[Chapter 2]{specification-RVV-v1.0} and therefore will be a multiple of the capability length.
    \end{itemize}
    \item The memory interface is identical to that used by the scalar processor, where each vector access is split into a set of sequential accesses less than 128 bits.
    \begin{itemize}
        \item Therefore the safety properties for the scalar code still hold.
        \item Example: capabilities can only be stored through a capability with the \code{STORE\_CAP} permission.
    \end{itemize}
    \item Capability memory accesses use \code{SafeTaggedCap} as a unit.
    \begin{itemize}
        \item This means it is impossible to set the tag bit on invalid capabilities.
        \item It also means all capability accesses are 128-bit aligned, and atomic.
    \end{itemize}
    \item The only instruction that can place capabilities in a vector register is 128-bit element loads.
    \begin{itemize}
        \item There are no vectorized capability-to-capability or integer-to-capability instructions, such as \code{CSetBounds} or \code{CFromPtr}.
        \item All other instructions that write to registers (e.g. arithmetic, 64-bit loads, etc.) unset the tag bit.
        \item The only way to place a valid capability in a vector register is to copy a valid capability from memory.
        \item Therefore the Provenance and Monotonicity properties are always upheld.
    \end{itemize}
    \item The only 128-bit element vector load instruction is unit-stride.
    \begin{itemize}
        \item Strided and Indexed accesses could be supported as long as they enforced alignment and atomicity correctly.
        \item Whole-register accesses would need to be updated to always use 128-bit elements, in case capabilities are being accessed.
    \end{itemize}
    \item Capabilities-in-vectors cannot be dereferenced directly.
    \begin{itemize}
        \item Therefore Integrity cannot be violated by vector operations.
    \end{itemize}
\end{itemize}