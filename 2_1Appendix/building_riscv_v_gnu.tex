\chapter{Building the \code{riscv-gnu-toolchain} with vector support}\label{appx:building_rvv_gcc_toolchain}

As of May 2022, the RISC-V GNU toolchain (hosted at \gitrepo{riscv-collab/riscv-gnu-toolchain}{https://github.com/riscv-collab/riscv-gnu-toolchain/}) does not support the vector extension or it's intrinsics.
The \code{rvv-intrinsic} branch of this repository claimed to support vector intrinsics, but it was slightly outdated and has been deleted as of 17th May 2022.
It referenced a repository for \code{glibc} that no longer exists as a submodule, which makes compilation impossible.
I have archived the branch in my own repository (\gitrepo{theturboturnip/riscv-gnu-toolchain}{https://github.com/theturboturnip/riscv-gnu-toolchain}) and fixed that issue.
This appendix describes how to build the toolchain.

To build the full toolchain with intrinsic support, perform the following steps (derived by the author independently, then amended based on macOS instructions from \todocite{https://github.com/riscv-collab/riscv-gcc/issues/323}):
\begin{enumerate}
    \item Clone the repository \emph{without} cloning any submodules.
    % \item Change the \code{.gitmodules} file to point the \code{riscv-glibc} submodule at the upstream \code{glibc} repository \url{https://sourceware.org/git/?p=glibc.git}, which hosts the RISC-V version as of 19 March 2021\footnote{\gitrepo[archive-notify]{riscvarchive/riscv-glibc}{https://github.com/riscvarchive/riscv-glibc/tree/archive-notify}}.
    
    % \todomark{Example}
    % \begin{itemize}
    %     \item Alternatively, this can be pointed at \url{https://github.com/riscvarchive/riscv-glibc.git}, the archived version of the old repository\todocite{macOS GCC instructions}.
    % \end{itemize}
    \item Clone the \code{riscv-gcc} submodule:
    \item[\code{\$}] \code{git submodule update --init --recursive --progress --force ./riscv-gcc}
    \begin{itemize}
        \item On macOS, it may be necessary to disable SSL:
        \item[\code{\$}] \code{git -c http.sslVerify=false submodule ...}
    \end{itemize}
    \item Configure the compiler so it supports all General extensions, Compressed instructions, and Vector extension:
    \item[\code{\$}] \code{./configure --prefix=<output directory> --with-arch=rv64gcv --with-abi=lp64d}
    \item Build the \code{newlib} version to compile for bare-metal platforms:
    \item[\code{\$}] \code{make newlib -j\$(nproc)}
\end{enumerate}
