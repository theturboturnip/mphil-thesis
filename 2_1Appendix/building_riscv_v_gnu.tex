\chapter{Building the \code{riscv-gnu-toolchain} with vector support}\label{appx:building_rvv_gcc_toolchain}

As of May 2022, the RISC-V GNU toolchain (hosted as \gitrepo{riscv-collab/riscv-gnu-toolchain}{https://github.com/riscv-collab/riscv-gnu-toolchain/}) does not support the vector extension or it's intrinsics.
\todomark{verify - do a fresh clone and see}
The \code{rvv-intrinsic} branch of this repository claims to support vector intrinsics, but is slightly outdated.
It references a repository for \code{glibc} that no longer exists as a submodule, which makes compilation impossible.

To build the full toolchain with intrinsic support, perform the following steps (derived by the author independently, then amended based on macOS instructions from \todocite{https://github.com/riscv-collab/riscv-gcc/issues/323}):
\begin{enumerate}
    \item Clone the repository as usual, \emph{without} cloning any submodules.
    \item Change the \code{.gitmodules} file to point the \code{riscv-glibc} submodule at the upstream \code{glibc} repository \url{https://sourceware.org/git/?p=glibc.git}, which hosts the RISC-V version as of 19 March 2021\footnote{\gitrepo[archive-notify]{riscvarchive/riscv-glibc}{https://github.com/riscvarchive/riscv-glibc/tree/archive-notify}}.
    
    \todomark{Example}
    \begin{itemize}
        \item Alternatively, this can be pointed at \url{https://github.com/riscvarchive/riscv-glibc.git}, the archived version of the old repository\todocite{maxOS GCC instructions}.
    \end{itemize}
    
    \item \todomark{clone submodules (macOS guy turns off SSL for this?)}
    \item \todomark{Configure with as many risc-v exts as possible}
    \item Build
\end{enumerate}

\todomark{Does RISC-V GCC 10.2 support vector intrinsics if you compile with v?}