%TC:macro \todomark [ignore]
%TC:macro \todoref [ignore]
%TC:macro \todocite [option:text,ignore]
%TC:macro \redact [text]
%TC:subst \subfile \input
%TC:fileinclude \figinput file

% The above comments tell texcount that some macros should not have their arguments counted as text.
% \todomark, \todoref, and \todocite are all macros used during development - there will be no TODO messages
% in the final release.
% In particular, \todomark is sometimes used with long sentences e.g. \todomark{Should I use a different sentence here?} which I don't want to count.
% The \cref and \crefrange macros are used throughout the thesis, and are counted slightly incorrectly:
% While the actual output of \cref{hyp:cap_in_vec_load_store} = "Hyp. H-8" (2 or 3 words), TeXcount counts it as "hyp cap in vec load store" (6 words).
% This means the actual wordcount is an over-approximation.

% TC:subst tells TeXcount to interpret \subfile as \input for the purposes of word counting.
% TeXcount has native support for the subfiles package, but because this file doesn't have a preamble we can't tell TeXcount that we're using it.

% Any text that is \redact-ed in the anonymous version is still counted.

\subfile{1_10Introduction/content}

\subfile{1_20Background/content}

\subfile{1_25EmulationInvestigation/content}

\subfile{1_30Software/content}

\subfile{1_40CapInVec/content}

\subfile{1_50Evaluation/content}

\subfile{1_60Conclusion/content}
\label{wc:end}