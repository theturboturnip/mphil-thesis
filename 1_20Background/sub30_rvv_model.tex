\section{The RVV vector model}\label{chap:bg:sec:rvv:vector_model}
\emph{Summarizes \cite[Sections 1-6, 17]{specification-RVV-v1.0}}

\figinput[width=0.7\textwidth,pos=h]{1_20Background/figures/fig_RVV_simple_widths}

RVV defines thirty-two vector registers, each of an implementation-defined constant width \code{VLEN}.
These registers can be interpreted as \emph{vectors} of \emph{elements}.
The program can configure the size of elements, and the implementation defines a maximum width \code{ELEN}.
\cref{fig:RVV_simple_widths} shows a simple example.

RVV also adds some state that defines how the vector registers are used (see \cref{fig:RVV_added_state}).
These are stored in RISC-V Control and Status Registers (CSRs), which the program can read.
\code{vtype} (\cref{chap:bg:subsec:vtype}) defines how the vector registers are split into elements.
\code{vstart} and \code{vl} (\cref{chap:bg:subsec:vlvstart}) divides the elements into three disjoint subsets: \emph{prestart}, the \emph{body}, and the \emph{tail}.
Masked accesses (\cref{chap:bg:subsec:rvvmasking}) further divide the \emph{body} into \emph{active} and \emph{inactive} elements.
This section also describes the vector exception model (\cref{chap:bg:subsec:vexceptions}).

\figinput[width=0.7\textwidth,pos=h]{1_20Background/figures/fig_RVV_added_state}

%---------------------------------
\subsection{\code{vtype}}\label{chap:bg:subsec:vtype}
\figinput[width=0.7\textwidth,pos=b]{1_20Background/figures/fig_RVV_LMUL_widening}
The \code{vtype} CSR contains two key fields that describe how vector instructions interpret the contents of vector registers.
The first is the Selected Element Width (\code{SEW}), which is self-explanatory.
It can be 8, 16, 32, or 64.
128-bit elements are referenced a few times throughout but haven't been formally specified (see \cite[p10, p32]{specification-RVV-v1.0}).
% Most instructions\footnote{Except where the width is encoded in the instruction, like bytemask loads.} will split vector registers into elements of this width.

The second field is the Vector Register Group Multiplier (\code{LMUL}).
Vector instructions don't just operate over a single register, but over a register \emph{group} as defined by this field.
For example, if \code{LMUL=8} then each instruction would operate over 8 register's worth of elements.
These groups must use aligned register indices, so if \code{LMUL=4} all vector register operands should be multiples of 4 e.g. \code{v0, v4, v8} etc.
In some implementations this may increase throughput, which by itself is beneficial for applications.

However, the true utility of \code{LMUL} lies in widening/narrowing operations (see \cref{fig:RVV_LMUL_widening}).
For example, an 8-by-8-bit multiplication can produce 16-bit results.
Because the element size doubles, the number of vector registers required to hold the same number of elements also doubles.
Doubling \code{LMUL} after such an operation allows subsequent instructions to handle all the results at once.
At the start of such an operation, fractional \code{LMUL} (1/2, 1/4, or 1/8) can be used to avoid subsequent results using too many registers.

\code{vtype} also encodes two flags: mask-agnostic and tail-agnostic.
If these are set, the implementation is \emph{allowed} to overwrite any masked-out or tail elements with all 1s.

Most vector instructions will interpret their operands using \code{vtype}, but this is not always the case.
Some instructions (such as memory accesses) use different Effective Element Widths (\code{EEW}) and Effective LMULs (\code{EMUL}) for their operands.
In the case of memory accesses, the \code{EEW} is encoded in the instruction bits and the \code{EMUL} is calculated to keep the number of elements consistent.
Another example is widening/narrowing operations, which by definition have to interpret the destination registers differently from the sources.

Programs update \code{vtype} through the \code{vsetvl} family of instructions.
These are designed for a ``stripmining'' paradigm, where each iteration of a loop processes some elements until all elements are processed.
\code{vsetvl} instructions take a requested \code{vtype} and the number of remaining elements to process (the Application Vector Length or \code{AVL}), and return the number of elements that will be processed in this iteration.
This value is saved in a register for the program to use, and also saved in the internal \code{vl} CSR.

%---------------------------------
\subsection{\code{vl} and \code{vstart} --- Prestart, body, tail}\label{chap:bg:subsec:vlvstart}
The first CSR is the Vector Length \code{vl}, which holds the number of elements that could be updated from a vector instruction.
The program updates this value through fault-only-first loads (\cref{chap:bg:sec:rvv:fof}) and more commonly \code{vsetvl} instructions.

In the simple case, \code{vl} is equal to the total available elements (see \cref{fig:RVV_vl_full}).
It can also be fewer (see \cref{fig:RVV_vl_short}), in which case vector instructions will not write to elements in the \enquote{tail} (i.e. elements past \code{vl}).
This eliminates the need for a `cleanup loop' common in fixed-length vector programs.

\figinput[width=0.7\textwidth,pos=t]{1_20Background/figures/fig_RVV_vl}

In a similar vein, \code{vstart} specifies \enquote{the index of the first element to be executed by a vector instruction}.
Elements before \code{vstart} are known as the \emph{pre-start} and are not touched by executed instructions.
It is usually only set by the hardware whenever it is interrupted mid-instruction (see \cref{fig:RVV_vstart_trap} and \cref{chap:bg:subsec:vexceptions}) so that the instruction can be re-executed later without corrupting completed values.
Whenever a vector instruction completes, \code{vstart} is reset to zero.

The program \emph{can} set \code{vstart} manually, but it may not always work.
If an implementation couldn't arrive at the value itself, then it is allowed to reject it.
The specification gives an example where a vector implementation never takes interrupts during an arithmetic instruction, so it would never set \code{vstart} during an arithmetic instruction, so it could raise an exception if \code{vstart} was nonzero for an arithmetic instruction.

%---------------------------------
\subsection{Masking --- Active/inactive elements}\label{chap:bg:subsec:rvvmasking}
Most vector instructions allow for per-element \emph{masking} (see \cref{fig:RVV_mask_example}).
When masking is enabled, register \code{v0} acts as the `mask register', where each bit corresponds to an element in the vector\footnote{A single vector register will always have enough bits for all elements. The maximum element count is found when SEW is minimized (8 bits) and LMUL is maximized (8 registers), and is equal to \code{VLEN * LMUL / SEW = VLEN * 8 / 8 = VLEN}.}.
If the mask bit is 0, that element is \emph{active} and will be used as normal.
If the mask bit is 1, that element will be \emph{inactive} and not written to (or depending on the mask-agnostic setting, overwritten with 1s).
When masking is disabled, all elements are \emph{active}.

\figinput[width=0.7\textwidth,pos=h]{1_20Background/figures/fig_RVV_vstart_trap}
\figinput[width=0.7\textwidth,pos=h]{1_20Background/figures/fig_RVV_mask_example}
\clearpage

\pagebreak
%---------------------------------
%---------------------------------
%---------------------------------
\subsection{Exception handling}\label{chap:bg:subsec:vexceptions}
\emph{Summarizes \cite[Section 17]{specification-RVV-v1.0}}

During the execution of a vector instruction, two events can prevent an instruction from fully completing: a synchronous exception in the instruction itself, or an asynchronous interrupt from another part of the system.
Implementations may choose to wait until an instruction fully completes before handling asynchronous interrupts, making it unnecessary to pause the instruction halfway through, but synchronous exceptions cannot be avoided in this way (particularly where page fault exceptions must be handled transparently).

The RVV specification defines two modes for `trapping' these events, which implementations may choose between depending on the context (e.g. the offending instruction), and notes two further modes which may be used in further extensions.
All modes start by saving the PC of the trapping instruction to a CSR \code{*epc}.

%---------------------------------
\subsubsection{Imprecise vector traps}
Imprecise traps are intended for events that are not recoverable, where \enquote{reporting an error and terminating execution is the appropriate response}.
They do not impose any extra requirements on the implementation.
For example, an implementation that executes instructions out-of-order does not need to guarantee that instructions older than \code{*epc} have completed, and is allowed to have completed instructions newer than \code{*epc}.

If the trap was triggered by a synchronous exception, the \code{vstart} CSR must be updated with the element that caused it.
The specification calls out synchronous exceptions in particular, but does not mention asynchronous interrupts.
It's likely that imprecise traps for asynchronous interrupts should also set \code{vstart}, but this issue has been raised with the authors for further clarification\footnote{\url{https://github.com/riscv/riscv-v-spec/issues/799}}.
The specification also states \enquote{There is no support for imprecise traps in the current standard extensions}, meaning that the other standard RISC-V exceptions do not use and have not considered imprecise traps.

%---------------------------------
\subsubsection{Precise vector traps}
Precise vector traps are intended for instructions that can be resumed after handling the interrupting event.
This means the architectural state (i.e. register values) when starting the trap could be saved and reloaded before continuing execution.
Therefore it must look like instructions were completed in-order, even if the implementation is out-of-order:
\begin{itemize}
    \item Instructions older than \code{*epc} must have completed (committed all results to the architectural state)
    \item Instructions newer than \code{*epc} must \textbf{not} have altered architectural state.
\end{itemize}

On a precise trap, regardless of what caused it, the \code{vstart} CSR must be set to the element index on which the trap was taken.
The save-and-reload expectation then add two constraints on the trapping instruction's execution:
\begin{itemize}
    \item Operations affecting elements preceding \code{vstart} must have committed their results
    \item Operations affecting elements at or following \code{vstart} must either
    \begin{itemize}
        \item not have committed results or otherwise affected architectural state
        \item be \emph{idempotent} i.e. produce exactly the same result when repeated.
    \end{itemize}
\end{itemize}

The idempotency option gives implementations a lot of leeway.
Some instructions, such as  indexed segment loads (\cref{rvv:indexedmem}), are specifically prohibited from overwriting their inputs to make them idempotent.
If an instruction is idempotent, an implementation is even allowed to repeat operations on elements \emph{preceding} \code{vstart}.
However for memory accesses the idempotency depends on the memory being accessed.
For example, reading or writing a memory-mapped I/O region may not be idempotent.

Another memory-specific issue is that of \emph{demand-paging}, where the OS needs to step in and move virtual memory pages into physical memory for an instruction to use.
This use-case is specifically called out by the specification for precise traps.
Usually, this is triggered by some element of a vector memory access raising a synchronous exception, invoking a precise trap, and writing the ``Machine Trap Value'' scalar register with the offending address\cite[Section 3.1.21]{specification-RISCV-vol2-20211203}.
\code{vstart} must be set to an element at (or before\footnote{If the memory region is idempotent, then \code{vstart} could any value where all preceding elements had completed. It could even be zero, in which case all accesses would be retried on resume, as long as it could guarantee forward progress.}) the one that demanded the page, because that element must perform the access after reloading.
If an implementation sets \code{vstart} to the offending element, because operations preceding \code{vstart} must have completed, any elements that could potentially trigger demand-paging \emph{must} wait for the preceding elements to complete.


% \code{vstart} must be set to an element that demanded the page
% Because \code{vstart} must be set to the element that demanded the page, and operations preceding \code{vstart} must have completed, any elements that could potentially trigger demand-paging \emph{must} wait for the preceding elements to complete.
% This always applies, no matter what the instruction's specific ordering guarantees are.

%---------------------------------
\subsubsection{Other modes}
The RVV spec notes two other possible future trap modes.
First is \enquote{Selectable precise/imprecise traps}, where an implementation allows the user to select precise or imprecise traps for e.g. debugging or performance.

The second mode is \enquote{Swappable traps}, where a trap handler could use special instructions to \enquote{save and restore the vector unit microarchitectural state}.
The intent seems to be to support context switching with imprecise traps, which could also require the \emph{opaque} state (i.e. internal state not visible to the program) to be saved and restored.
Right now, it seems that context switching always requires a precise trap.

%---------------------------------
\subsection{Summary}\label{chap:bg:subsec:rvvsummary}
\figinput[width=0.8\textwidth,pos=h]{1_20Background/figures/fig_RVV_examples_combined}
\cref{fig:RVV_examples_combined} shows all of the above features used in a single configuration:
\begin{itemize}
    \item The instruction was previously interrupted with a precise trap and restarted, so \code{vstart=2}
    \item Elements are 16-bit
    \item \code{LMUL=4} to try and increase throughput
    \item Only 29 of the 32 available elements were requested, so \code{vl=29} (3 tail elements)
    \item Some elements are masked out/inactive (in this case seemingly at random)
    \item Overall, 21 elements are active
\end{itemize}
