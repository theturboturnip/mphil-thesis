\chapter{Background}
% PLAN: 4k words

This chapter describes CHERI (\cref{chap:bg:sec:cheri}) and RVV (\cref{chap:bg:sec:rvv}) to the detail required to understand the rest of the dissertation.
Both are described because this dissertation caters to multiple audiences: those who already understand CHERI, but are less familiar with vectors; and those already familiar with vector hardware but not with CHERI.
The final section describes two other vector models in a little more detail.
They are both proprietary, so we cannot learn from their hardware, but they are useful for contextualizing the software side of things.


\section{CHERI}\label{chap:bg:sec:cheri}
% Concept of a capability
In CHERI, addresses/pointers are replaced with capabilities: unforgeable tokens that provide \emph{specific kinds of access} to an \emph{address} within a \emph{range of memory}.
The above statement is enough to understand what capabilities contain\footnote{This is a slight simplification. For the purposes of vector memory accesses the \emph{otype} of a capability can be ignored, as any type other than \code{UNSEALED} cannot be dereferenced anyway.}:
\begin{itemize}
    \item Permission bits, to restrict access
    \item The \emph{cursor}, i.e. the address it currently points to
    \item The \emph{bounds}, i.e. the range of addresses this capability could point to
\end{itemize}
A great deal of work has gone into compressing capabilities down into a reasonable size (see \cite{woodruffCHERIConcentratePractical2019}, \todoref{diagram from TR-941?}), and using the magic of floating-point all of this data has been reduced to just 2x the architectural register size.
For example, on 64-bit RISC-V a standard capability is 128-bits long.
The rest of this dissertation assumes capabilities are 128-bits long for simplicity.

In order to keep track of what is and isn't a valid capability, registers and memory are both tagged.
Each 128-bit register and each aligned 128-bit region of memory has an associated tag bit, which denotes if its data encodes a valid capability\footnote{This has the side-effect that capabilities must be 128-bit aligned in memory.}.
If any non-capability data is written to any part of the region, the tag bit is zeroed out.
\todomark{Can you have a capability with tag=1, but the cursor is outside the bounds?}
As above, significant work has gone into the implementation to reduce the DRAM overhead of this method (see \cite{joannouEfficientTaggedMemory2017} for an example).


\todomark{Explain the three security properties from davisCheriABIEnforcingValid2019?}


\todomark{rewrite above section - just copied over some stuff previously from Emulation Investigation and it doesn't flow}


\subsection{CHERI-RISC-V ISA}
The Cambridge Computer Lab's TR-951 report\todocite{TR-951} describes the latest version of the CHERI architecture (CHERI ISAv8) and proposes applications to MIPS, x86-64, and RISC-V.
CHERI-RISC-V is a mostly straightforward set of additions to basic RISC-V ISAs.
It adds thirty-two general-purpose capability registers, thirty-two Special Capability Registers (SCRs), and many new instructions.

The new general-purpose capability registers are each of size \code{CLEN = 2 * XLEN} plus a tag bit.
These registers store compressed capabilities.
While there is always a logical distinction between the pre-existing \emph{integer} registers \code{x0-x31} and the \emph{capability} registers \code{cx0-cx31}, the architecture may store them in a Split or Merged register file.
A Split register file stores the integer registers separately from capability registers, so programs can manipulate them independently.
A Merged register file stores thirty-two registers of length \code{CLEN}, using the full width for the capability registers, and aliases the integer registers to the bottom \code{XLEN} bits.
Under a merged register file, writing to an integer register makes the capability counterpart invalid, so programs have to be more careful with register usage.

\todomark{diagram for split/merged register file?}

\todomark{All new CHERI instructions specify if their operands are integers or capabilities, and pull from the correct set of logical registers.}
\todomark{Non-CHERI enabled instructions default to integer representations, and in some cases can be switched to use capabilities using the Integer/Capability encoding mode flag}

Many of the new SCRs are intended to support the privileged ISA extensions for e.g. hypervisors or operating systems.
The emulator doesn't use these, so their SCRs are not listed here, but there are two highly relevant SCRs for all modes: the Program Counter Capability and the Default Data Capability.
The PCC supplements \todomark{replaces?} the program counter by adding more metadata, ensuring instruction fetches have the same security as normal loads and stores.


\subsection{Integer/Capability mode}
\todomark{Capability mode flag overrides behaviour of RISC-V }

\subsection{Capability Relocations}
\todomark{talk about ELF cap-relocs table, how they work and why}

%---------------------------------
%---------------------------------
%---------------------------------
\section{A brief history of vector processing}
Many vector implementations (Intel SSE/AVX, Arm's Advanced SIMD and Neon) use fixed-length vectors - e.g. 128-bit vectors which a program interprets as four 32-bit elements.
As the industry's desire for parallelism grew, new implementations had to be designed with longer vectors of more elements.
For example, Intel SSE/SSE2 (both 128-bit) was succeeded by AVX (128 and 256-bit), then AVX2 (entirely 256-bit), then AVX-512 (512-bit).
Programs built for one extension, and hence designed for a specific vector size, could not automatically take advantage of longer vectors.

Scalable vectors address this by not specifying the vector length, and instead calculating it on the fly.
Instead of hardcoding \enquote{this loop iteration uses a single vector of four 32-bit elements}, the program has to ask \enquote{how many 32-bit elements will this iteration use?}.
This gives hardware designers more freedom, letting them select a suitable hardware vector length for their power/timing targets, while guaranteeing consistent execution of programs on arbitrarily-sized vectors.
RVV uses a scalable vector model.

\section{RVV}\label{chap:bg:sec:rvv}
% RISC-V has extensions
RISC-V is an open family of ISAs which defines ``base integer ISAs'' (e.g. all 64-bit RISC-V cores implement the RV64I Base Integer Instruction Set) and extensions (e.g. the ``M'' extension for integer multiplication).
A base instruction set combined with a set of extensions is known as a RISC-V ISA.
Because RISC-V is open, anyone can design, manufacture, and sell chips implementing any RISC-V ISA.
% RVV is the vector one
RVV is the officially ratified vector extension for RISC-V, and going forward all RISC-V chips with vector processing capabilities should implement it instead of designing their own custom vector extensions.
This section summarizes Sections 1-9 and 17 of the RVV Specification v1.0\cite{RISCVVectorExtension2021}.

%---------------------------------
%---------------------------------
%---------------------------------
\subsection{Vector model}\label{chap:bg:sec:rvv:vector_model}
\emph{Summarizes \cite[Sections 1-4]{RISCVVectorExtension2021}}
\todomark{Add prestart/active/inactive/body/tail definitions, would bump up to 1-5}


\figinput[width=0.7\textwidth,pos=h]{1_20Background/figures/fig_RVV_simple_widths}
\figinput[width=0.7\textwidth,pos=h]{1_20Background/figures/fig_RVV_added_state}


RVV defines thirty-two vector registers, each of an implementation-defined width VLEN.
These registers can be interpreted as \emph{vectors} of \emph{elements}.
The program can configure the size of elements, and the implementation defines a maximum width ELEN.
\cref{fig:RVV_simple_widths} shows a simple example of a 128-bit vector, where the maximum element length is 32-bits.


RVV also adds some state that defines how the vector registers are used (see \cref{fig:RVV_added_state}).
These are stored in RISC-V Control and Status Registers (CSRs), which the program can read.

%---------------------------------
\subsubsection{\code{vl} and \code{vstart}}\label{chap:bg:sec:rvv:vstart}
The first CSR is the Vector Length \code{vl}, which holds the number of elements that could be updated from a vector instruction.
The program updates this value through fault-only-first loads (\todoref{fof load}) and more commonly the \code{vsetvl} instruction family (\todoref{vsetvl}).

In the simple case, \code{vl} is equal to the total available elements (see \cref{fig:RVV_vl_full}).
It can also be fewer (see \cref{fig:RVV_vl_short}), in which case vector instructions will not write to elements in the \enquote{tail} (i.e. elements past \code{vl}).
This eliminates the need for a `cleanup loop' common in fixed-length vector programs.

\figinput[width=0.7\textwidth,pos=t]{1_20Background/figures/fig_RVV_vl}

In a similar vein, \code{vstart} specifies \enquote{the index of the first element to be executed by a vector instruction}.
This is usually only set by the hardware whenever it is interrupted mid-instruction (see \cref{fig:RVV_vstart_trap}) so that the instruction can be re-executed later without corrupting completed values.
Whenever a vector instruction completes, \code{vstart} is reset to zero.

The program \emph{can} set \code{vstart} manually, but it may not always work.
If an implementation couldn't arrive at the value itself, then it is allowed to reject it.
The specification gives an example where a vector implementation never takes interrupts during an arithmetic instruction, so it would never set \code{vstart} during an arithmetic instruction, so it could raise an exception if \code{vstart} was nonzero for an arithmetic instruction.

\figinput[width=0.7\textwidth,pos=t]{1_20Background/figures/fig_RVV_vstart_trap}

%---------------------------------
\subsubsection{\code{vtype}}
\code{vtype} contains two key fields that describe how vector instructions interpret the contents of vector registers.
The first is the Selected Element Width (\code{SEW}), which is self-explanatory.
It can be 8, 16, 32, or 64.
Most instructions\footnote{Except where the width is encoded in the instruction, like bytemask loads.} will split vector registers into elements of this width.

The second field is the Vector Register Group Multiplier (\code{LMUL}).
Vector instructions do not necessarily operate over a single register, but over a register \emph{group} as defined by this field.
For example, if \code{LMUL=8} then each instruction would operate over 8 register's worth of elements.
These groups must use aligned register indices, so if \code{LMUL=4} all vector register operands should be multiples of 4 e.g. \code{v0, v4, v8} etc.
In some implementations this may increase throughput, which by itself is beneficial for applications.

\figinput[width=0.7\textwidth,pos=t]{1_20Background/figures/fig_RVV_LMUL_widening}

However, the true utility of \code{LMUL} lies in widening/narrowing operations (see \cref{fig:RVV_LMUL_widening}).
For example, an 8-by-8-bit multiplication can produce 16-bit results.
Because the element size doubles, the number of vector registers required to hold the same number of elements also doubles.
Doubling \code{LMUL} after such an operation allows subsequent instructions to handle all the results at once.

\code{vtype} also encodes two flags: mask-agnostic and tail-agnostic.
If these are set, the implementation is \emph{allowed} to overwrite any masked-out or tail elements with all 1s.

Most vector instructions will interpret their operands using \code{vtype}, but this is not always the case.
Some instructions (such as memory accesses) use different Effective Element Widths (\code{EEW}) and Effective LMULs (\code{EMUL}) for their operands.
In the case of memory accesses, these are encoded in the instruction bits.
Another example is widening/narrowing operations, which by definition have to interpret the destination registers differently from the sources.

\pagebreak
\figinput[width=0.8\textwidth,pos=h]{1_20Background/figures/fig_RVV_mask_example}
%---------------------------------
\subsubsection{Masking}
Most vector instructions allow for per-element \emph{masking} (see \cref{fig:RVV_mask_example}).
When masking is enabled, register \code{v0} acts as the `mask register', where each bit corresponds to an element in the vector.
If the mask bit is 1, that element will be `masked out' and not written to (or depending on the mask-agnostic setting, overwritten with 1s).

\todomark{There will always be enough bits in a single register to mask all elements. maximum element count comes from smallest SEW (8 bits) and largest LMUL (8 registers). Max elements = VLEN * LMUL / SEW = VLEN * 8 / 8 = VLEN elements, there are VLEN bits in a single register.}

\figinput[width=\textwidth,pos=h]{1_20Background/figures/fig_RVV_examples_combined}
%---------------------------------
\subsubsection{Summary}
\cref{fig:RVV_examples_combined} shows all of the above features used in a single configuration:
\begin{itemize}
    \item The instruction was previously interrupted and restarted, so \code{vstart=2}
    \item Elements are 16-bit
    \item \code{LMUL=4} to try and increase throughput
    \item Only 29 of the 32 available elements were requested, so \code{vl=29}
    \item Some elements are masked out (in this case seemingly at random)
\end{itemize}

\pagebreak
%---------------------------------
%---------------------------------
%---------------------------------
\subsection{Strip mining and \code{vsetvl}}
\emph{Summarizes \cite[Section 6]{RISCVVectorExtension2021}}
\todomark{example usage of the above, introduces vsetvli}

\pagebreak
%---------------------------------
%---------------------------------
%---------------------------------
\subsection{Exception handling}
\emph{Summarizes \cite[Section 17]{RISCVVectorExtension2021}}

During the execution of a vector instruction, two events can prevent an instruction from fully completing: a synchronous exception in the instruction itself, or an asynchronous interrupt from another part of the system.
Implementations may choose to wait until an instruction fully completes before handling asynchronous interrupts, making it unnecessary to pause the instruction halfway through, but synchronous exceptions cannot be avoided in this way (particularly for those performing memory accesses).
The RVV specification defines two modes for `trapping' these events, which implementations may choose between depending on the context (e.g. the offending instruction), and notes two further modes which may be used in further extensions.
All modes start by saving the PC of the trapping instruction to a CSR \code{*epc}.

%---------------------------------
\subsubsection{Imprecise vector traps}
Imprecise traps are intended for events that are not recoverable, where \enquote{reporting an error and terminating execution is the appropriate response}.
They do not impose any extra requirements on the implementation.
For example, an implementation that executes instructions out-of-order does not need to guarantee that instructions older than \code{*epc} have completed, and is allowed to have completed instructions newer than \code{*epc}.

If the trap was triggered by a synchronous exception, the \code{vstart} CSR must be updated with the element that caused it.
\todomark{inconsistency in spec - Ch17 first para says "the vstart CSR contains the element index on which the trap was taken", but the imprecise trap section only specifies this for synchronous exceptions}

The specification also states \enquote{There is no support for imprecise traps in the current standard extensions}, \todomark{what does this mean?}

%---------------------------------
\subsubsection{Precise vector traps}
Precise vector traps are intended for instructions that can be resumed after handling the interrupting event.
This means the architectural state (i.e. register values) when starting the trap could be saved and reloaded before continuing execution.
Therefore it must look like instructions were completed in-order, even if the implementation is out-of-order:
\begin{itemize}
    \item Instructions older than \code{*epc} must have completed (committed all results to the architectural state)
    \item Instructions newer than \code{*epc} must \textbf{not} have altered architectural state.
\end{itemize}

On a precise trap, regardless of what caused it, the \code{vstart} CSR must be set to the element index on which the trap was taken.
The save-and-reload expectation then add two constraints on the trapping instruction's execution:
\begin{itemize}
    \item Operations affecting elements preceding \code{vstart} must have committed their results
    \item Operations affecting elements at or following \code{vstart} must either
    \begin{itemize}
        \item not have committed results or otherwise affected architectural state
        \item be \emph{idempotent} i.e. produce exactly the same result when repeated.
    \end{itemize}
\end{itemize}

The idempotency option gives implementations a lot of leeway.
Some instructions \todomark{examples} are specifically prohibited from overwriting their inputs to make them idempotent.
If an instruction is idempotent, an implementation is even allowed to repeat operations on elements \emph{preceding} \code{vstart}.
However for memory accesses the idempotency depends on the memory being accessed.
For example, writing to a memory-mapped I/O region may not be idempotent.

Another memory-specific issue is that of \emph{demand-paging}, where the OS needs to step in and move virtual memory pages into physical memory for an instruction to use.
This use-case is specifically called out by the specification for precise traps.
Usually, this is triggered by some element of a vector memory access raising a synchronous exception, invoking a precise trap, and \todomark{using \code{vstart} to show the OS which address it wants?}
Because \code{vstart} must be set to the element that demanded the page, and operations preceding \code{vstart} must have completed, any elements that could potentially trigger demand-paging \emph{must} wait for the preceding elements to complete.
This always applies, no matter what the instruction's specific ordering guarantees are.

%---------------------------------
\subsubsection{Other modes}
The RVV spec mentions two other potential trap modes.
First is \enquote{Selectable precise/imprecise traps}, where an implementation provides a bit that selects between precise or imprecise traps.
The intent is to allow precise traps to be selected for e.g. debugging purposes, and for imprecise traps to be selected for extra performance.

The second mode is \enquote{Swappable traps}, where a trap handler could use special instructions to \enquote{save and restore the vector unit microarchitectural state}.
The intent seems to be to support context switching with imprecise traps, which could also require the \emph{opaque} state (i.e. internal state not visible to the program) to be saved and restored.
Right now, it seems that context switching always requires a precise trap.

Neither of these modes are actually defined, but they are simply noted as possibilities for the future.

\pagebreak
%---------------------------------
%---------------------------------
%---------------------------------
\subsection{Memory access instructions}
\emph{Summarizes \cite[Sections 7-9]{RISCVVectorExtension2021}}

RVV defines five broad categories of memory access instructions, which this subsection describes.
For the most part, they handle their operands as described in \cref{chap:bg:sec:rvv:vector_model}.
\code{EEW} and \code{EMUL} are usually derived from the instruction encoding, rather than reading the \code{vtype} CSR.
In one case (\todoref{bytemask access}) the Effective Vector Length \code{EVL} is different from the \code{vl} CSR, so for simplicity all instructions are described in terms of \code{EVL}.

\subsubsection{Segment accesses}
Three of the five categories (unit/strided, fault-only-first, and indexed) support \emph{segmented} access.
This is used for unpacking contiguous structures of $n$ \emph{fields} and placing each field in a separate vector.
Rather than 

\pagebreak
\subsubsection{Unit and Strided accesses}


\begin{tabular}{ll}
\toprule
    \code{EEW} & From Instruction \\
    \code{EMUL} & From Instruction \\
    \midrule
    \code{NFIELDS} & From Instruction \\
    \code{EVL} & \code{vl} \\
    \bottomrule
\end{tabular}

\pagebreak
\subsubsection{Fault-only-first unit loads}

\begin{tabular}{ll}
\toprule
    \code{EEW} & From Instruction \\
    \code{EMUL} & From Instruction \\
    \midrule
    \code{NFIELDS} & From Instruction \\
    \code{EVL} & \code{vl} \\
    \bottomrule
\end{tabular}

\pagebreak
\subsubsection{Indexed accesses}
\begin{tabular}{ll}
\toprule
    Element \code{EEW} & \code{vtype.SEW} \\
    Element \code{EMUL} & \code{vtype.LMUL} \\
    \midrule
    Index Width & From Instruction \\
    Index \code{MUL} & From Instruction \\
    \midrule
    \code{NFIELDS} & From Instruction \\
    \code{EVL} & \code{vl} \\
    \bottomrule
\end{tabular}

\pagebreak
\subsubsection{Whole-register accesses}
\begin{tabular}{ll}
\toprule
    \code{EEW} & From Instruction \\
    \midrule
    \code{NFIELDS} & From Instruction \\
    \code{EVL} & \code{NFIELDS * VLEN / EEW} \\
    \bottomrule
\end{tabular}

\pagebreak
\subsubsection{Bytemask accesses}
\begin{tabular}{ll}
\toprule
    \code{EEW} & 8-bits \\
    \code{EMUL} & 1 \\
    \midrule
    \code{EVL} & \code{ceil(vl/8)} \\
    \bottomrule
\end{tabular}

\pagebreak
%---------------------------------
%---------------------------------
%---------------------------------
\subsection{Implementations}

\section{Other vector models}
\subsection{Arm SVE}
\subsection{Intel SSE/AVX}