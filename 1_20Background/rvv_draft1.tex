\section{RVV}\label{chap:bg:sec:rvv}
% RISC-V has extensions
RISC-V defines a vast number of 
% RVV is the vector one

% RVV adds registers
RVV defines thirty-two new vector registers, each of implementation-defined length VLEN.
\begin{itemize}
    \item vtype (SEW + LMUL)
    \item vlen
    \item vstart
\end{itemize}
% registers are scalable
% vsetvl family for getting `vector length' based on `vtype'
% vtype
% Every vector instruction operates on at most vlen elements of SEW width
% `at most' => elements can be excluded based on vstart and masking

% If vector instructions are interrupted after completing X elements, e.g. crossing a page boundary and faulting during a memory access, the implementation may set `vstart=X' and resume the instruction from that element later
% Explored further in \cref{??}
% Thus each instruction operates on elements between vstart and vlen

% masking is common to other fixed-length architectures
% set bits in register v0 - 1 = masked \emph{out}, 0 = performs as normal

\subsection{Motivation for scalable vectors}
\todomark{I originally wrote this for the introduction, but I felt it was taking up too much time. It could be useful for someone familiar with fixed-length vectors but not scalable ones. Should it be here, or move somewhere else?}
Many vector implementations (Intel SSE/AVX, Arm's Advanced SIMD and Neon) use fixed-length vectors - e.g. 128-bit vectors which a program interprets as four 32-bit elements.
As the industry's desire for parallelism grew, new implementations had to be designed with longer vectors of more elements.
For example, Intel SSE/SSE2 (both 128-bit) was succeeded by AVX (128 and 256-bit), then AVX2 (entirely 256-bit), then AVX-512 (512-bit).
Programs built for one extension, and hence designed for a specific vector size, could not automatically take advantage of longer vectors.

Scalable vectors address this by not specifying the vector length, and instead calculating it on the fly.
Instead of hardcoding \enquote{this loop iteration uses a single vector of four 32-bit elements}, the program has to ask \enquote{how many 32-bit elements will this iteration use?}.
This gives hardware designers more freedom, letting them select a suitable hardware vector length for their power/timing targets, while guaranteeing consistent execution of programs on arbitrarily-sized vectors.


\subsection{Memory accesses}
unit/strided,
indexed,
fof,
bytemask,
wholeregister.

\subsection{Interaction with exceptions}

\subsection{Implementations}