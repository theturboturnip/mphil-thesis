\section{RVV}\label{chap:bg:sec:rvv}
% RISC-V has extensions
RISC-V is an open family of ISAs which defines ``base integer ISAs'' (e.g. all 64-bit RISC-V cores implement the RV64I Base Integer Instruction Set) and extensions (e.g. the ``M'' extension for integer multiplication).
A base instruction set combined with a set of extensions is known as a RISC-V ISA.
Because RISC-V is open, anyone can design, manufacture, and sell chips implementing any RISC-V ISA.
% RVV is the vector one
RVV is the officially ratified vector extension for RISC-V, and going forward all RISC-V chips with vector processing capabilities should implement it instead of designing their own custom vector extensions.

% RVV adds registers, vtype, vlen, vstart
RVV defines thirty-two new vector registers, each of implementation-defined length VLEN.
% registers are scalable
VLEN can be different for different chips, so programs have to adapt to the VLEN provided by the processor it's running on.
% \begin{itemize}
%     \item vtype (SEW + LMUL)
%     \item vlen
%     \item vstart
% \end{itemize}
% vsetvl family for getting `vector length' based on `vtype'
This is achieved using the \code{vsetvl} family of instructions.

\subsection{\code{vsetvl} and \code{vtype}}
\code{vsetvl} instructions take in the total number of elements that need processing and a \code{vtype} describing those elements, and returns the number of elements that can be processed per instruction.
A colloquial description is shown in \cref{fig:vsetvl}.

\begin{figure}
    \centering
    \begin{displayquote}
    I have $N$ elements to process overall.
    These elements are $SEW$-bits wide, and I'd like to use $LMUL$ vector register's worth of elements per instruction.
    How many elements will I process in a single instruction?
    \end{displayquote}
    \caption{Summary of \code{vsetvl} behaviour}
    \label{fig:vsetvl}
\end{figure}


% vtype, LMUL
The \code{vtype} passed to a \code{vsetvl} operation is stored by the processor, and has two key components\footnote{The tail and mask-agnostic bits are ignored here for the purposes of simplification}:
\begin{itemize}
    \item Selected Element Width \code{SEW}
    \item Vector Register Group Multiplier \code{LMUL}
\end{itemize}

\todomark{talk about SEW? it's pretty obvious how it works}

The utility of \code{LMUL} is not immediately obvious.
It defines the length of the Vector Register Group, which all instructions operate on.
For example, if $LMUL = 8$ then vectorized operations will run on each element over 8 vector registers.
Applications which benefit from using longer logical vectors, e.g. for higher throughput, can request a large LMUL and leave it at that.
However the intent behind LMUL is to make widening/narrowing calaulations easier.
For example, if a vector multiplication multiplies 8-bit input elements producing 16-bit results, storing those 16-bit results requires twice as many vector registers as for the input elements.
\todomark{diagram explaining that lol}

\subsection{\code{vlen}, \code{vstart}, and masking}
% Every vector instruction operates on at most vlen elements of SEW width
% `at most' => elements can be excluded based on vstart and masking

% If vector instructions are interrupted after completing X elements, e.g. crossing a page boundary and faulting during a memory access, the implementation may set `vstart=X' and resume the instruction from that element later
% Explored further in \cref{??}
% Thus each instruction operates on elements between vstart and vlen

% masking is common to other fixed-length architectures
% set bits in register v0 - 1 = masked \emph{out}, 0 = performs as normal

\subsection{Memory accesses}
% unit/strided,
% indexed,
% fof,
% bytemask,
% wholeregister.

\subsection{Interaction with exceptions}

\subsection{Implementations}