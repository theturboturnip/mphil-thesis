\section{CHERI}\label{chap:bg:sec:cheri}
% Concept of a capability
In CHERI, addresses/pointers are replaced with capabilities: unforgeable tokens that provide \emph{specific kinds of access} to an \emph{address} within a \emph{range of memory}.
The above statement is enough to understand what capabilities contain\footnote{This is a slight simplification. For the purposes of vector memory accesses the \emph{otype} of a capability can be ignored, as any type other than \code{UNSEALED} cannot be dereferenced anyway.}:
\begin{itemize}
    \item Permission bits, to restrict access
    \item The \emph{cursor}, i.e. the address it currently points to
    \item The \emph{bounds}, i.e. the range of addresses this capability could point to
\end{itemize}
A great deal of work has gone into compressing capabilities down into a reasonable size (see \cite{woodruffCHERIConcentratePractical2019}, \todoref{diagram from TR-941?}), and using the magic of floating-point all of this data has been reduced to just 2x the architectural register size.
For example, on 64-bit RISC-V a standard capability is 128-bits long.
The rest of this dissertation assumes capabilities are 128-bits long for simplicity.

In order to keep track of what is and isn't a valid capability, registers and memory are both tagged.
Each 128-bit register and each aligned 128-bit region of memory has an associated tag bit, which denotes if its data encodes a valid capability\footnote{This has the side-effect that capabilities must be 128-bit aligned in memory.}.
If any non-capability data is written to any part of the region, or the capability is manipulated to make it invalid (e.g. moving the cursor out-of-bounds), the tag bit is zeroed out.
As above, significant work has gone into the implementation to reduce the DRAM overhead of this method (see \cite{joannouEfficientTaggedMemory2017} for an example).


\todomark{Explain the three security properties from davisCheriABIEnforcingValid2019?}

\subsection{CHERI-RISC-V ISA}
The Cambridge Computer Lab's TR-951 report\todocite{TR-951} describes the latest version of the CHERI architecture (CHERI ISAv8) and proposes applications to MIPS, x86-64, and RISC-V.
CHERI-RISC-V is a mostly straightforward set of additions to basic RISC-V ISAs.
It adds thirty-two general-purpose capability registers, thirty-two Special Capability Registers (SCRs), and many new instructions.

The new general-purpose capability registers are each of size \code{CLEN = 2 * XLEN} plus a tag bit.
These registers store compressed capabilities.
While there is always a logical distinction between the pre-existing \emph{integer} registers \code{x0-x31} and the \emph{capability} registers \code{cx0-cx31}, the architecture may store them in a Split or Merged register file.
A Split register file stores the integer registers separately from capability registers, so programs can manipulate them independently.
A Merged register file stores thirty-two registers of length \code{CLEN}, using the full width for the capability registers, and aliases the integer registers to the bottom \code{XLEN} bits.
Under a merged register file, writing to an integer register makes the capability counterpart invalid, so programs have to be more careful with register usage.

\todomark{diagram for split/merged register file?}

\todomark{All new CHERI instructions specify if their operands are integers or capabilities, and pull from the correct set of logical registers.}
\todomark{Non-CHERI enabled instructions default to integer representations, and in some cases can be switched to use capabilities using the Integer/Capability encoding mode flag}

Many of the new SCRs are intended to support the privileged ISA extensions for e.g. hypervisors or operating systems.
The emulator doesn't use these, so their SCRs are not listed here, but there are two highly relevant SCRs for all modes: the Program Counter Capability and the Default Data Capability.
The PCC supplements \todomark{replaces?} the program counter by adding more metadata, ensuring instruction fetches have the same security as normal loads and stores.


\subsection{Integer/Capability mode}
\todomark{Capability mode flag overrides behaviour of RISC-V }

\subsection{Compilation - Pure-capability and legacy modes}

\subsection{Capability Relocations}
\todomark{talk about ELF cap-relocs table, how they work and why}

\subsection{Instruction reference}
\todomark{TR-951 contains a complete reference of all added instructions in section blah, and notes legacy instructions modified by capability mode in section blah. Reproduce short list? Do so in appendix?}