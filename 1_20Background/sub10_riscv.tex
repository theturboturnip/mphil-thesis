\section{RISC-V}\label{chap:bg:sec:rvv}
% RISC-V has extensions
RISC-V is an open family of ISAs which defines ``base integer ISAs'' (e.g. all 64-bit RISC-V cores implement the RV64I Base Integer Instruction Set) and extensions (e.g. the ``M'' extension for integer multiplication).
A base instruction set combined with a set of extensions (\todomark{example of a architecture/feature string}) is known as a RISC-V ISA.
Because RISC-V is open, anyone can design, manufacture, and sell chips implementing any RISC-V ISA.

Each RISC-V implementation has a set of constant parameters.
The most common example is \code{XLEN}, the length of an integer register in bits, which is tied to the base integer ISA (e.g. 64-bit ISA implies \code{XLEN=64}).
Other constant parameters include \code{CLEN}, the length of a capability in bits, defined by CHERI relative to \code{XLEN}; and \code{VLEN} and \code{ELEN}, which are used by RVV and entirely implementation-defined.

The extensions of most relevance to this project are the ``V'' vector extension (RVV) and the CHERI extension.
RVV recently became the officially ratified vector extension for RISC-V, which all RISC-V vector processing chips should implement going forward.
The following sections summarize the vector extension, how it accesses memory, and previous implementations in academia.
% \todomark{finish this para}

% RVV is the officially ratified vector extension for RISC-V, and going forward all RISC-V chips with vector processing capabilities should implement it instead of designing their own custom vector extensions.