\chapter{The CHERI-RVV software stack}
While building hardware that can execute vector instructions is important, generating those vector instructions in software quickly and easily is equally important.
This chapter explores the current state of the so-called ``CHERI-RVV software stack'': mainstream compiler support for vanilla RVV (\todoref{4.1}), and the modifications required to bring support to CHERI-Clang (\todoref{4.2}).
The software hypotheses are tested with this knowledge (\todoref{4.3}), and we recommend a set of changes to bring CHERI-Clang support to par with other compilers (\todoref{4.4}).
% First, the available compiler support for vanilla RVV is explored and briefly compared to Arm SVE .
% that translates to CHERI-Clang and what changes were necessary to implement capabilities, and 

\section{Compiling vector code}
Modern compilers provide many ways to generate vectorized code.
While this support is very advanced for well established vector models, like x86-64 AVX, newer vector models like RVV don't have as many options.
It can even be difficult to get the compiler to generate any vector instructions at all.
This section examines support across the Clang and GCC compilers for various vectorization methods on RVV.

\subsection{Required command-line options}
Before you can compile vector code, the compiler must be told to use a vector ISA.
This is fortunately quite easy, only requiring an addition to the architecture feature string \todomark{is it called that?} in most cases.
On Clang 13 and other LLVM-13-based compilers, version 0.1 of the vector specification is supported as an experimental extension, so an extra command.
Clang/LLVM~14 and up support RVV v1.0.
GCC is an interesting case - there is a branch of \code{riscv-gcc-toolchain} that supports RVV v0.1, based on RISC-V GCC 10.1, but it hasn't been touched for more than a year.
See \cref{appx:building_rvv_gcc_toolchain} for more information on finding and building this version.

\begin{table}[]
    \centering
\begin{tabular}{llp{0.55\linewidth}}
    \toprule
    Compiler & Required Arguments & Notes \\ 
    \midrule
    Clang-13 & \code{-march=rv64gv0p10}  & Supports intrinsics, inline assembly for RVV v0.1 \\
    & \code{-menable-experimental-extensions} & \\
    Clang-14+ & \code{-march=rv64gv} & Supports intrinsics, inline assembly for RVV v1.0 \\
    GCC 10.1 & \code{-march=rv64g_v} & Requires special toolchain (see \cref{appx:building_rvv_gcc_toolchain}) and has incomplete support (see \todoref{bit from testing where we talk about GCC being iffy}) \\
    \bottomrule
\end{tabular}
    \caption{Command-line arguments for compiling RVV code on various compilers\\(assuming the base ISA is \code{rv64g})}
    \label{tab:rvv_cmdline_nocheri}
\end{table}


\subsection{Automatic vectorization}
\todomark{Figure page showing generated ASM for increment loop - see godbolt links MD}
Compilers with auto-vectorization can automatically create vectorized code from a scalar program.
For example, a scalar loop over an array that increments each element could be converted to a vectorized loop that increments multiple elements at once.
Although this is simple in some cases, auto-vectorization can take significant effort and time to implement (for example, GCC started implementing it for x86 in 2003 and only turned on basic support in 2007 \todocite{https://gcc.gnu.org/projects/tree-ssa/vectorization.html}).
Currently there is no support for RVV auto-vectorization in Clang or GCC.
Both compilers have support for Arm SVE auto-vectorization, explored further in \cref{chap:soft:compiling:armsve}.

\subsection{Vector intrinsics}
\enquote{Intrinsics} are functions built in to a compiler that can invoke low-level functionality and instructions directly for a specific architecture.
When automatic vectorization is not available, intrinsics are the next best thing - they present a familiar high-level interface (function calls), that gives the programmer fine-grained control over which instructions to execute, typically providing an intrinsic for each vector instruction.
The compiler then handles low-level decisions like register allocation under the hood, and sometimes may provide extra functionality for ease of use.
%the compiler still allocates registers 

% A prime example is RVV's vector intrinsics\todocite{RISCVVectorIntrinsicsv0.1}, implemented in GCC~?\todocite{} and Clang~13+\todocite{}.
% All instructions that rely on the current vector length \code{vl} take it as an argument for their respective intrinsics.
% The programmer is strongly encouraged to use a \code{vsetvl} intrinsic to generate the length first, but could also pass the 
% which allow (and strongly suggest) the programmer to get the vector length with a \code{vsetvl} function, then pass it to an 

RVV has a comprehensive set of vector intrinsics\todocite{RISCVVectorIntrinsicsv0.1}, implemented in the aforementioned special version of GCC and Clang~13+.
With these, the general strip-mining loop is easy to construct:
\todomark{example based on \url{https://github.com/riscv-non-isa/rvv-intrinsic-doc/blob/master/examples/rvv_memcpy.c}}
\begin{enumerate}
    \item Use a \code{vsetvl} intrinsic to get the vector length for this iteration.
    \item Allocate vector registers by declaring variables with vector types (e.g. \code{vuint32m8_t} represents 8 registers worth of 32-bit unsigned integers).
    \item Pass the vector length to the computation/memory intrinsics, which operate on the vector variables.
\end{enumerate}

\todomark{Hammer home that intrinsics aren't reusable across instruction sets? e.g. AVX intrinsics don't work with RISC-V}

\subsection{Inline assembly}
If a compiler doesn't supply complete intrinsics, or if the programmer desires extremely fine-grained control, inline assembly may be used.
The programmer gives a string of handwritten assembly code to the compiler, which is parsed and directly inserted into the output code at that point.
The compiler still has to interpret the instruction and understand it correctly, but as long as it knows the instruction this method does not depend on any intrinsics being present (or functional\footnote{For example, CHERI-Clang could not compile code with vector intrinsics, so had to use inline assembly for all vector instructions.}).

Inline assembly can interact with C code and variables through a template syntax.
The programmer inserts a placeholder in the assembly code with a corresponding expression, noting how the expression is stored using a \enquote{constraint}.
For our purposes, constraints enforce that a value is either in a register or in memory (see \cref{tab:inline_asm_constraint}).
As an example, writing to a memory address stored in a variable could use a constraint \code{"m"(*addr)} - i.e. \enquote{the value pointed to by \code{addr} is stored in memory}. \todoref{Example} \todomark{the previous sentence kinda sucks at getting the point across. Using a constraint forces the compiler to move the value into a register/memory}

Using the constraint, the compiler determines how the expression's value is stored, and inserts a reference to it in the assembly string.
Because this is done before the assembly string is parsed, and isn't immediately type-checked against the assembly instruction, it can lead to some difficult errors.

Clang and GCC support inline assembly for RVV quite well, and even allows the intrinsic vector types to be referenced by assembly templates (thus making the compiler do register allocation instead of the programmer).
The only caveat is that \emph{memory} constraints are not supported by RVV memory accesses.
None of the vector memory access instructions support address offsets, unlike their scalar counterparts.
Clang always treats the \emph{memory} constraint as an offset access, even when that offset is zero, so it adds an offset to the assembly string \todoref{example}, making it invalid.
To get around this, one must use the pointer itself with a \emph{register} constraint - effectively saying \enquote{find the register this pointer is in, and use that as the base address for the memory access} \todoref{example}.
On CHERI platforms, because pointers must be stored in capability registers, the \emph{capability register} constraint must be used instead (see \todoref{CHERI-agnostic inline assembly}).

Broadly speaking, inline assembly supports more RVV instructions than intrinsics do.
It is used extensively in the testbench code for the evaluation (\todoref{evaluation}) alongside intrinsics where possible.

% the compiler determines where the value of the expression is/should be stored, and inserts a reference to that location in the assembly string.
% The programmer may also control where the expression is stored by setting a \enquote{constraint}.
% This is an extremely useful feature, particularly if 
% Because this is done before the assembly string is parsed, it can lead to some difficult-to-understand errors.

\begin{table}[]
    \centering
    \begin{tabular}{c|c}
       "="  & Output - the old value is overwritten. Can be combined with other constraints. \\
        "r" & Store in a register \\
        "vr" & Store in a vector register (RVV only) \\
        "C" & Store in a capability register (CHERI-Clang only) \\
        "m" & Store in memory \\
    \end{tabular}
    \caption{Inline assembly constraints \todocite{\url{https://gcc.gnu.org/onlinedocs/gcc/Extended-Asm.html}}}
    \label{tab:inline_asm_constraints}
    \todomark{Beautify this table}
\end{table}

\subsection{vs. Arm SVE}\label{chap:soft:compiling:armsve}
\emph{Summarizes \todocite{armltdArmCompilerScalable2019}}

Arm SVE uses a similar model to RVV, where the vector length may scale between 128 and 2048\footnote{RVV slightly differs here, as it allows VLEN smaller than 128.} and the instructions are designed to be totally agnostic across different platforms\todocite{stephensARMScalableVector2017}.
Arm have released a C language extension to support SVE development (\todocite{armltdARMLanguageExtensions2020}), supported by the Arm Compiler for Embedded\todocite{\url{https://developer.arm.com/Tools\%20and\%20Software/Arm\%20Compiler\%20for\%20Embedded}}, Clang, and GCC.
It supports all of the previously examined vectorization types.

Auto-vectorization is supported, and the main focus of the user guide is helping the compiler decide whether to auto-vectorize \todocite{armltdArmCompilerScalable2019}.
Intrinsics are also supported, and seem to cover all of the SVE instructions, but take a slightly different approach to RVV.
Arm SVE intrinsics do not directly map to available instructions, but aim to \enquote{provide a regular interface and leave the compiler to pick the best mapping to SVE instructions}, while RVV intrinsics (at least for memory) tend to map 1:1 to existing instructions.
Arm's approach gives more flexibility for future extensions, as the same intrinsics could be compiled to new instructions with newer compilers.

Arm SVE also supports inline assembly, but the experience is notably worse than for RVV.
The two standout issues are a lack of register allocation and the use of condition code flags for branching.
Unlike RVV, the intrinsic types for vector values cannot be referenced in inline assembly\todocite{stephensARMScalableVector2017}, so all vector registers must be allocated and tracked by the programmer.
Arm SVE's equivalent of \code{vsetvl}, the \code{while} family\todocite{armltdARMLanguageExtensions2020}, do not return the number of updated elements, and instead set the condition flags based on how many elements are updated.
Because there is no way to branch based on the condition flags in C, the programmer must manually insert a label for the top of the loop, and a branch to that label (see \url{https://godbolt.org/z/zoWh9jq3o}), which is more error prone than the RVV method.
Examples of Arm SVE code with auto-vectorization, intrinsics, and inline ASM can be found in \todomark{appendix based on \url{https://godbolt.org/z/zoWh9jq3o}}.


\pagebreak
\section{Compiling vector code with CHERI-Clang}
Current CHERI compiler work is done on CHERI-Clang, a fork of Clang and other LLVM tools that supports capabilities.
It's based on LLVM~13, so it includes support for vanilla RVV v0.1, but none of the vector-extension related code had been updated to work with capabilities.
This section outlines the changes one has to make to CHERI-Clang and vector code to compile programs for CHERI-RVV.

\subsection{Required command-line options}
\begin{table}[]
    \centering
\begin{tabular}{llp{0.55\linewidth}}
    \toprule
    Compiler & Required Arguments & Notes \\ 
    \midrule
    CHERI-Clang  & \code{-march=rv64gv0p10xcheri}  & Supports intrinsics, inline assembly for RVV v0.1. \\
    \multicolumn{1}{c}{(LLVM-13)} & \code{-menable-experimental-extensions} & LLVM 13 RVV support is experimental. \\
    & \code{-mabi=l64pc128} & ABI string required to set capability width. \\
    & \code{-mno-relax} & Linker relaxations must be disabled. \\
    \bottomrule
\end{tabular}
    \caption{Command-line arguments for compiling CHERI-RVV code\\(assuming the base ISA is \code{rv64g})}
    \label{tab:rvv_cmdline_cheri}
\end{table}

By default CHERI-Clang doesn't actually compile capability-enabled code.
The documentation on enabling capabilities is unfortunately sparse and outdated.
In particular, the CHERI-Clang help menu states that \code{--cheri} will \enquote{Enable CHERI support with the default capability size}, but this has no effect (at least on RISC-V).
To find up-to-date answers, the CHERIbuild (\todocite{https://github.com/CTSRD-CHERI/cheribuild}) build tool was consulted.

CHERIbuild's code\footnote{\url{https://github.com/CTSRD-CHERI/cheribuild/blob/ba3a0b6388436224968c906192c61d2ccbdd7616/pycheribuild/config/compilation_targets.py\#L176}} revealed three requirements:
\begin{itemize}
    \item The architecture string must contain \code{xcheri}
    \item The capability length must be set using the ABI string
    \begin{itemize}
        \item In pure-capability mode, pointers and capabilities are \code{CLEN} long
        \begin{itemize}
            \item Example string: \code{l64pc128}
            \item Integer width (\code{long}, or \code{l}) = XLEN = 64-bits
            \item Pointer width (\code{p}) = Capability width (\code{p}) = CLEN = 128-bits
        \end{itemize}
        \item For hybrid mode, pointers remain \code{XLEN} long and capability length is not specified
        \begin{itemize}
            \item Example string: \code{lp32}
            \item Integer width (\code{l}) = XLEN = Pointer width (\code{p}) = 32-bits
        \end{itemize}
    \end{itemize}
    \item ``Linker relaxations'', where function calls are converted to short jumps\cite{chenCompilerSupportLinker2019}, must be disabled.
    \begin{itemize}
        \item This is likely because CHERI requires function calls to go through capabilities
        \item However the code that adds this option wasn't documented, so there may be more to it
    \end{itemize}
\end{itemize}

Once the above options are set, plain CHERI-RISC-V code compiles without a hitch.
Changes to CHERI-Clang itself are required to compile vectors.

\subsection{Adapting vector assembly instructions to CHERI}
LLVM uses a domain-specific language to describe the instructions it can emit for a given target.
The RISC-V target describes multiple register sets that RISC-V instructions can use.
Vanilla RVV vector memory accesses use the General Purpose Registers (GPR) to store the base address of each access.
CHERI-Clang added a GPCR set, i.e. the General Purpose Capability Registers.
As noted in \cref{chap:emu:rvv_int_mode} CHERI-RVV requires the vector memory accesses to support integer \emph{and} capability mode, therefore two versions of the vector accesses must be created: versions which take a capability base address, only available in CHERI/Capability mode, and versions which take integer base addresses for Integer mode.

\todomark{Appendix on how this was done?}
With the above changes, inline assembly could be used to insert capability-enabled vector instructions\todomark{Example}.
However, as this requires using a capability register constraint for the base address, inline assembly code written for CHERI-RVV is not inherently compatible with vanilla RVV.
For un-annotated pointers (e.g. \code{int*}), which are only capabilities in pure-capability code and integers in legacy or hybrid code, a conditional macro can be used to insert the correct constraint: \todoref{example}.
However, this falls apart in hybrid code for manually annotated pointers (e.g. \code{int* \_\_capability}) because the macro cannot detect the annotation.


\subsection{Adapting vector intrinsics to CHERI}
Vector intrinsics are another story entirely.
When compiling for pure-capability libraries, all attempts to use vector intrinsics crash CHERI-Clang \todoref{example of error message}.
This is due to a similar issue to inline assembly: the intrinsics (both the Clang intrinsic functions and the underlying LLVM IR intrinsics) were designed to take regular pointers and cannot handle it when capabilities are used instead.
\todoref{Appendix which covers what I know so far about this problem?}
Unfortunately the code for generating the intrinsics on both levels is spread across many files, and there's no simple way to change the associated pointer type (much less changing it for pure capability vs. hybrid mode).

It seems that significant compiler development work is required to bring vector intrinsics up to scratch on CHERI-Clang.
We did experiment with creating replacement wrapper functions, where each function tried to mimic an intrinsic using inline asssembly.
These were rejected for two reasons: the increased overhead of a function call on every vector instruction\footnote{This could have been eliminated by using preprocessor macros instead of real functions, but they are difficult to program and do not easily support returning values like intrinsics do.}, and the lack of support for passing vector types as arguments or return values.
The RISC-V ABI treats all vector registers as temporary and explicitly states that \enquote{vector registers are not used for passing arguments or return values}\todocite{RISCV-ABI}, and CHERI has its own issues with saving vector registers on the stack.

\todomark{segue into saving registers on stack doesn't make sense - doesn't explain why someone would want to}

\subsection{Storing scalable vectors on the stack}
If a program uses more data than can fit in registers, or calls a function which may overwrite important register values, the compiler will save those register values to memory on the stack.
Because vector registers are temporary, and thus may be overwritten by called functions, they must also be saved/restored from the stack\todomark{Example https://godbolt.org/z/KPTW7rcvY}.
This also applies to multiprocessing systems where a process can be paused, have the state saved, and resume later.
RVV provides the whole-register memory access instructions explicitly to make this process easy.

CHERI-Clang contains an LLVM IR pass\footnote{llvm/lib/CodeGen/CheriBoundAllocas.cpp} which enforces strict bounds on so-called ``stack capabilities'' (capabilities pointing to stack-allocated data), which by definition requires knowing the size of the data ahead of time.
This pass assumes all stack-allocated data has a static size, and crashes when dynamically-sized types e.g. scalable vectors are allocated.
It is therefore impossible (for now) to save vectors on the stack in CHERI-Clang, although it's clear that it's theoretically possible.
For example, the length of the required vector allocations could be calculated based on \code{VLEN} before each stack allocation is performed, or if performance is a concern stack bounds for those allocations could potentially be ignored altogether.
These possibilities are investigated further in the next section.

\todomark{Note: Arm Language C extensions https://developer.arm.com/documentation/100987/0000/ defines the concept of a sizeless type, which may be stored on the stack. Would be a good base for RVV?}

\todocite{LLVM IR pass investigated in TR-949 \$3.8.2, couldn't find earlier reference to it}

\pagebreak
\section{Evaluating hypotheses}\label{chap:software:sec:hypotheses}

\hypsubsection{hyp:sw_vec_legacy}{Compiling/running legacy code in integer mode}
This is true for CHERI-RVV, when running the compiled programs in integer mode, as long as the programs only access memory within the DDC.

All vanilla RVV instructions have counterparts with identical encodings and behaviour in CHERI-RVV integer mode, assuming the accessed addresses are all accessible through the DDC.
There are no changes to instruction behaviour that require the compiler's handling of them to change, so a non-CHERI compiler and an integer-mode-CHERI compiler can always produce the same vector instructions from the same code.
This does not apply to capability-mode-CHERI, because integer addressing is not supported in capability-mode-CHERI-RVV.

All legacy vector programs should produce equivalent binaries when compiled for integer-mode-CHERI.
On top of that, all binaries compiled for non-CHERI RVV platforms should produce the same results when run on an equivalent integer-mode-CHERI RVV platform.
Both claims assume the program doesn't perform accesses that \todomark{violate? exceed?} the DDC.

\hypsubsection{hyp:sw_pure_compat}{Converting legacy code to pure-capability code}
This is true for CHERI-RVV, but cannot be done in practice yet.
Some engineering effort is required to support this in CHERI-Clang.
Because this argument concerns source code, all three ways to generate CHERI-RVV instructions must be examined.

\subsubsection*{Inline Assembly --- Unlikely}
For GCC-style inline assembly, it is currently impossible for integer-addressed RVV source code to be recompiled in pure-capability mode without modification.
Integer-addressed RVV assembly uses general-purpose registers for the base address, but the pure-capability instructions require capability registers instead.
The base address register can either be specified directly, so must be changed to a capability register; or specified using template syntax and an ``r'' constraint, which must be changed to a capability ``C'' constraint (\cref{fig:inlineasm,fig:inlineasmcheri}).
Using a preprocessor macro in the template syntax (e.g. \cref{subfig:inline_asm_vector_portable}) could make code portable between non-CHERI and CHERI, but adding it still requires source code changes.

In theory, one could change the behaviour of inline assembly to automatically convert general purpose registers/constraints to capability versions in specific circumstances.
However, this can have wide-reaching ramifications, potentially making code more difficult to understand, or even breaking existing code.

\subsubsection*{Intrinsics --- Yes}
The current specification for RVV intrinsics uses pointer types for all base addresses\cite{specification-RVV-intrinsics}.
In pure-capability compilers all pointers should be treated as capabilities instead of integers, including those in intrinsics.
All RVV memory intrinsics have equivalent RVV instructions, which all use capabilities in pure-capability mode, so changing the intrinsics to match is valid.

Assuming all base address pointers are created in a valid manner (e.g. through \code{malloc} or monotonic decrease, and not through integer literals), the conversion to pure-capability should make them all valid capabilities which are compatible with the intrinsics.
Therefore well-behaved code using RVV intrinsics should be compilable in pure-capability mode without changes.

This is not currently the case for CHERI-Clang, as RVV memory access intrinsics are broken, but this can be fixed with engineering effort.

\subsubsection*{Auto-vectorization --- Yes}
All vanilla RVV instructions have counterparts with identical encodings and behaviour in CHERI-RVV pure-capability mode, assuming the base addresses can be converted to valid capabilities.
Any scalar code that can be 
\begin{enumerate*}[label=\alph*)]
    \item compiled in scalar pure-capability mode\footnote{This ensures all memory accesses use valid capabilities.}, and
    \item auto-vectorized by a legacy RVV compiler,
\end{enumerate*}
must have an equivalent pure-capability vectorized form.
This form could be acquired by performing the auto-vectorization in legacy mode, ensuring all base addresses are available as capabilities, then making the vector instructions use those capabilities.
Therefore a pure-capability compiler can always auto-vectorize CHERI-compliant scalar code if some legacy compiler can also auto-vectorize it.

This is not currently possible for CHERI-Clang, as RVV auto-vectorization is not implemented yet.
Similar models (e.g. Arm SVE) already have auto-vectorization, so RVV auto-vectorization (and thus CHERI-RVV auto-vectorization) should be possible.

\hypsubsection{hyp:sw_stack_vectors}{Saving vectors on the stack}
% \todomark{Yes vectors can be stored on the stack}
This is true in theory, but not yet supported by CHERI-Clang in practice.
Placing variable-length structures on the stack is possible as long as the length can be known at runtime (and as long as the stack has space, of course).
This isn't exclusive to CHERI --- to push and pop values on the stack, the stack pointer must be incremented or decremented by the size of the value.
Because the length already has to be measured, and CHERI-RISC-V supports setting capability bounds from runtime-computed values, it's entirely possible to correctly set tight bounds for capabilities pointing to variable-length vectors on the stack.

% Arm SVE explicitly defines a ``sizeless type'' in the C language extensions \cite{} that may be stored on the stack, so this will need to be implemented for 
% \todomark{Note: Arm Language C extensions https://developer.arm.com/documentation/100987/0000/ defines the concept of a sizeless type, which may be stored on the stack. Would be a good base for RVV?}

A minor complication is presented by a note in TR-949\cite[Section 3.8.2]{TR-949} concerning ``re-materializing bounded stack variables''.
This section implies LLVM can try to re-create a pointer-to-stack at any time with minimal cost, but this may not be able to apply to vectors.
Measuring the bounds requires measuring \code{VLMAX} by changing \code{vl}, which could then require saving/restoring the old value.
This is only a performance issue, and in the worst case we can just say pointers-to-stack-vectors are not re-materializable, so it isn't a dealbreaker.
Further investigation of this issue is left as future work.
% LLVM allows pointers to be non-re-materializable, so this isn't a dealbreaker, but it should be investigated in the future.
% Further investigation of this issue is left as future work.

\hypsubsection{hyp:sw_multiproc}{Running CHERI-RVV code in a multiprocessing system}
% \todomark{H-D: Yes IFF vectors can be stored on the stack. May require a dynamic stack bounds calculation based on VLEN.}
This requires two conditions: an OS must be able to save and restore vector state, and the vector hardware must support resuming from an interrupted state.
The first condition is easy to fulfil by extending the previous hypothesis. 
If it is possible to save variable-length vectors on the stack, given their length is known at runtime, it must also be possible to save their data on the heap.
Some OSs might need to make changes to their ``current process state'' structure to support variable-length data, and they would also need to allocate space for the \code{vtype} value, but it is certainly possible.

The second condition can be upheld in two ways.
First, if the OS only context switches and services interrupts while the vector hardware is in a complete state (i.e. not partially executing an instruction), then context switches and interrupts are completely transparent to the vector hardware and no changes need to be made.
Secondly, if context switches and interrupts can actually interrupt vector instructions partway through, then they can only be cleanly resumed if the vector hardware supports precise traps for the exact instruction being executed.

\subsection{Verification and testing: \code{vector\_memcpy}}\label{chap:software:eval}
To verify the behaviour of the CHERI-RVV instructions, we developed a self-checking test program for the emulator to execute.
At first it was hand-written, but in order to test a wide range of \code{vtype}s we now generate it with a Python script.
It consists of fifty-seven tests of different vector memory access archetypes under various configurations (\cref{tab:vectormemcpyschemes}).

% The test code is compatible with integer-mode and capability-mode CHERI (and non-CHERI) through preprocessor macros, as shown in .
\todomark{Put the compatibility preprocessor stuff in the compiler appendix}
The test code will use intrinsics wherever the compiler supports them (see \cref{compilerdifferences}), and falls back to inline assembly otherwise.
Inline assembly uses the preprocessor macro from \cref{subfig:inline_asm_vector_portable} to handle CHERI and non-CHERI platforms.
% On CHERI platforms, which don't support intrinsics, the preprocessor macro from \todoref{preprocessor macro} is used to support inline ASM.
% Where the compiler supports intrinsics, the test code will use them, and otherwiuse
% Based on the compiler version, the test code runs operations with intrinsics where
% To make up for inconsistent RVV support, the test code checks the compiler version and implements operations with inline assembly or intrinsics as supported.
% Because the compilers don't have consistent RVV support (see \cref{compilerdifferences})

These tests are run under \emph{harnesses}, which provide setup and self-checking code for common cases:
The Vanilla harness tests a simple \code{memcpy} between two arrays;
Masked tests that every other element is copied, not all of them;
Segmented tests a \code{memcpy} into four separate output arrays, each a different field of a four-field structure.
There is also a special test for fault-only-first: FoF loads are performed at the edge of mapped memory, and the test verifies that out-of-bounds exceptions are swallowed and \code{vl} is reduced accordingly.
All tests were successful when they ran, but some testbenches could not be built with some compilers.
The full set of test results is available in \cref{chap:fullresults}.

\begin{table}[h]
    \centering
    \begin{tabular}{lcc}
    \toprule
        Test Scheme & Harness & Compilers \\
    \midrule
        Unit Stride & Vanilla & All \\
        Strided & Vanilla & All \\
        Indexed & Vanilla & All \\
        Whole Register & Vanilla & All \\
        Fault-only-First & Vanilla & All \\
        
        Unit Stride (Masked) & Masked & All \\
        Bytemask Load & Masked & \code{llvm-15} only \\
        
        Unit Stride (Segmented) & Segmented & All \\

        Fault-only-First Boundary & --- & All \\
    \bottomrule
    \end{tabular}
    \caption{\code{vector\_memcpy} test schemes and harnesses}
    \label{tab:vectormemcpyschemes}
\end{table}

\section{Recommended changes for CHERI-Clang}
\begin{itemize}
    \item Build on current work to make all RVV memory access instructions and pseudoinstructions CHERI-compatible.
    \item Make RVV memory access intrinsics take capabilities as arguments when compiled in pure-capability mode.
    \item Consider options on handling stack bounds for scalable vectors, and implement in the CheriBoundAllocas IR pass.
\end{itemize}
% \todomark{where to put something on how CHERI-Clang should change?}
