\chapter{The CHERI-RVV software stack}
\todomark{Introduction}

\section{Compiling vector code}
Modern compilers provide many ways to generate vectorized code.
While this support is very advanced for well established vector models, like x86-64 AVX, newer vector models like RVV don't have as many options.
It can even be difficult to get the compiler to generate any vector instructions at all.
This section examines support across the Clang and GCC compilers for various vectorization methods on RVV.

\subsection{Required command-line options}
Before you can compile vector code, the compiler must be told to use a vector ISA.
This is fortunately quite easy, only requiring an addition to the architecture feature string \todomark{is it called that?} in most cases.
On Clang 13 and other LLVM-13-based compilers, version 0.1 of the vector specification is supported as an experimental extension, so an extra command.
Clang/LLVM~14 and up support RVV v1.0.
GCC is an interesting case - there is a branch of \code{riscv-gcc-toolchain} that supports RVV v0.1, based on RISC-V GCC 10.1, but it hasn't been touched for more than a year.
See \cref{appx:building_rvv_gcc_toolchain} for more information on finding and building this version.

\begin{table}[]
    \centering
\begin{tabular}{llp{0.55\linewidth}}
    \toprule
    Compiler & Required Arguments & Notes \\ 
    \midrule
    Clang-13 & \code{-march=rv64gv0p10}  & Supports intrinsics, inline assembly for RVV v0.1 \\
    & \code{-menable-experimental-extensions} & \\
    Clang-14+ & \code{-march=rv64gv} & Supports intrinsics, inline assembly for RVV v1.0 \\
    GCC 10.1 & \code{-march=rv64g_v} & Requires special toolchain (see \cref{appx:building_rvv_gcc_toolchain}) and has incomplete support (see \todoref{bit from testing where we talk about GCC being iffy}) \\
    \bottomrule
\end{tabular}
    \caption{Command-line arguments for compiling RVV code on various compilers\\(assuming the base ISA is \code{rv64g})}
    \label{tab:rvv_cmdline_nocheri}
\end{table}


\subsection{Automatic vectorization}
\todomark{Figure page showing generated ASM for increment loop - see godbolt links MD}
Compilers with auto-vectorization can automatically create vectorized code from a scalar program.
For example, a scalar loop over an array that increments each element could be converted to a vectorized loop that increments multiple elements at once.
Although this is simple in some cases, auto-vectorization can take significant effort and time to implement (for example, GCC started implementing it for x86 in 2003 and only turned on basic support in 2007 \todocite{https://gcc.gnu.org/projects/tree-ssa/vectorization.html}).
Currently there is no support for RVV auto-vectorization in Clang or GCC.
Both compilers have support for Arm SVE auto-vectorization, explored further in \cref{chap:soft:compiling:armsve}.

\subsection{Vector intrinsics}
\enquote{Intrinsics} are functions built in to a compiler that can invoke low-level functionality and instructions directly for a specific architecture.
When automatic vectorization is not available, intrinsics are the next best thing - they present a familiar high-level interface (function calls), that gives the programmer fine-grained control over which instructions to execute, typically providing an intrinsic for each vector instruction.
The compiler then handles low-level decisions like register allocation under the hood, and sometimes may provide extra functionality for ease of use.
%the compiler still allocates registers 

% A prime example is RVV's vector intrinsics\todocite{RISCVVectorIntrinsicsv0.1}, implemented in GCC~?\todocite{} and Clang~13+\todocite{}.
% All instructions that rely on the current vector length \code{vl} take it as an argument for their respective intrinsics.
% The programmer is strongly encouraged to use a \code{vsetvl} intrinsic to generate the length first, but could also pass the 
% which allow (and strongly suggest) the programmer to get the vector length with a \code{vsetvl} function, then pass it to an 

RVV has a comprehensive set of vector intrinsics\todocite{RISCVVectorIntrinsicsv0.1}, implemented in the aforementioned special version of GCC and Clang~13+.
With these, the general strip-mining loop is easy to construct:
\todomark{example based on \url{https://github.com/riscv-non-isa/rvv-intrinsic-doc/blob/master/examples/rvv_memcpy.c}}
\begin{enumerate}
    \item Use a \code{vsetvl} intrinsic to get the vector length for this iteration.
    \item Allocate vector registers by declaring variables with vector types (e.g. \code{vuint32m8_t} represents 8 registers worth of 32-bit unsigned integers).
    \item Pass the vector length to the computation/memory intrinsics, which operate on the vector variables.
\end{enumerate}

\todomark{Hammer home that intrinsics aren't reusable across instruction sets? e.g. AVX intrinsics don't work with RISC-V}

\subsection{Inline assembly}
If a compiler doesn't supply complete intrinsics, or if the programmer desires extremely fine-grained control, inline assembly may be used.
The programmer gives a string of handwritten assembly code to the compiler, which is parsed and directly inserted into the output code at that point.
The compiler still has to interpret the instruction and understand it correctly, but as long as it knows the instruction this method does not depend on any intrinsics being present (or functional\footnote{For example, CHERI-Clang could not compile code with vector intrinsics, so had to use inline assembly for all vector instructions.}).

Inline assembly can interact with C code and variables through a template syntax.
The programmer inserts a placeholder in the assembly code with a corresponding expression, noting how the expression is stored using a \enquote{constraint}.
For our purposes, constraints enforce that a value is either in a register or in memory (see \cref{tab:inline_asm_constraint}).
As an example, writing to a memory address stored in a variable could use a constraint \code{"m"(*addr)} - i.e. \enquote{the value pointed to by \code{addr} is stored in memory}. \todoref{Example} \todomark{the previous sentence kinda sucks at getting the point across. Using a constraint forces the compiler to move the value into a register/memory}

Using the constraint, the compiler determines how the expression's value is stored, and inserts a reference to it in the assembly string.
Because this is done before the assembly string is parsed, and isn't immediately type-checked against the assembly instruction, it can lead to some difficult errors.

Clang and GCC support inline assembly for RVV quite well, and even allows the intrinsic vector types to be referenced by assembly templates (thus making the compiler do register allocation instead of the programmer).
The only caveat is that \emph{memory} constraints are not supported by RVV memory accesses.
None of the vector memory access instructions support address offsets, unlike their scalar counterparts.
Clang always treats the \emph{memory} constraint as an offset access, even when that offset is zero, so it adds an offset to the assembly string \todoref{example}, making it invalid.
To get around this, one must use the pointer itself with a \emph{register} constraint - effectively saying \enquote{find the register this pointer is in, and use that as the base address for the memory access} \todoref{example}.
On CHERI platforms, because pointers must be stored in capability registers, the \emph{capability register} constraint must be used instead (see \todoref{CHERI-agnostic inline assembly}).

Broadly speaking, inline assembly supports more RVV instructions than intrinsics do.
It is used extensively in the testbench code for the evaluation (\todoref{evaluation}) alongside intrinsics where possible.

% the compiler determines where the value of the expression is/should be stored, and inserts a reference to that location in the assembly string.
% The programmer may also control where the expression is stored by setting a \enquote{constraint}.
% This is an extremely useful feature, particularly if 
% Because this is done before the assembly string is parsed, it can lead to some difficult-to-understand errors.

\begin{table}[]
    \centering
    \begin{tabular}{c|c}
       "="  & Output - the old value is overwritten. Can be combined with other constraints. \\
        "r" & Store in a register \\
        "vr" & Store in a vector register (RVV only) \\
        "C" & Store in a capability register (CHERI-Clang only) \\
        "m" & Store in memory \\
    \end{tabular}
    \caption{Inline assembly constraints \todocite{That one GCC manual page about inline asm constraints}}
    \label{tab:inline_asm_constraints}
    \todomark{Beautify this table}
\end{table}

\subsection{vs. Arm SVE}\label{chap:soft:compiling:armsve}
\todomark{Arm SVE has extensive auto-vectorization, RISC-V V does not.}
\todomark{Would it be easy to bring this auto-vectorization code to RISC-V V?}
\todomark{Arm SVE and RVV have comparable(?) intrinsics}
\todomark{Arm SVE and RVV have comparable(?) abilities, and can both achieve these through inline assembly?}
\todomark{Can you use Arm SVE intrinsic types with inline ASM?}

\pagebreak
\section{Compiling vector code with CHERI-Clang}
Current CHERI compiler work is done on CHERI-Clang, a fork of Clang and other LLVM tools that supports capabilities.
It's based on LLVM~13 so supports RVV v0.1, but none of the vector-extension related code had been updated to work with capabilities.
This section outlines the changes one has to make to CHERI-Clang and vector code to compile programs for CHERI-RVV.

\subsection{Required command-line options}
\begin{table}[]
    \centering
\begin{tabular}{llp{0.55\linewidth}}
    \toprule
    Compiler & Required Arguments & Notes \\ 
    \midrule
    CHERI-Clang  & \code{-march=rv64gv0p10xcheri}  & Supports intrinsics, inline assembly for RVV v0.1. \\
    \multicolumn{1}{c}{(LLVM-13)} & \code{-menable-experimental-extensions} & LLVM 13 RVV support is experimental. \\
    & \code{-mabi=l64pc128} & ABI string required to set capability width. \\
    & \code{-mno-relax} & Linker relaxations must be disabled. \\
    \bottomrule
\end{tabular}
    \caption{Command-line arguments for compiling CHERI-RVV code\\(assuming the base ISA is \code{rv64g})}
    \label{tab:rvv_cmdline_cheri}
\end{table}

By default CHERI-Clang doesn't actually compile capability-enabled code.
The documentation on enabling capabilities is unfortunately sparse and outdated.
In particular, the CHERI-Clang help menu states that \code{--cheri} will \enquote{Enable CHERI support with the default capability size}, but this has no effect (at least on RISC-V).
To find up-to-date answers, the CHERIbuild (\todocite{https://github.com/CTSRD-CHERI/cheribuild}) build tool was consulted.

CHERIbuild's code\footnote{\url{https://github.com/CTSRD-CHERI/cheribuild/blob/ba3a0b6388436224968c906192c61d2ccbdd7616/pycheribuild/config/compilation_targets.py\#L176}} revealed three requirements:
\begin{itemize}
    \item The architecture string must contain \code{xcheri}
    \item The capability length must be set using the ABI string
    \begin{itemize}
        \item In pure-capability mode, pointers and capabilities are \code{CLEN} long
        \begin{itemize}
            \item Example string: \code{l64pc128}
            \item Integer width (\code{long}, or \code{l}) = XLEN = 64-bits
            \item Pointer width (\code{p}) = Capability width (\code{p}) = CLEN = 128-bits
        \end{itemize}
        \item For hybrid mode, pointers remain \code{XLEN} long and capability length is not specified
        \begin{itemize}
            \item Example string: \code{lp32}
            \item Integer width (\code{l}) = XLEN = Pointer width (\code{p}) = 32-bits
        \end{itemize}
    \end{itemize}
    \item ``Linker relaxations'', where function calls are converted to short jumps\cite{chenCompilerSupportLinker2019}, must be disabled.
    \begin{itemize}
        \item This is likely because CHERI requires function calls to go through capabilities
        \item However the code that adds this option wasn't documented, so there may be more to it
    \end{itemize}
\end{itemize}

Once the above options are set, plain CHERI-RISC-V code compiles without a hitch.
Changes to CHERI-Clang itself are required to compile vectors.

\subsection{Adapting vector assembly instructions to CHERI}
LLVM uses a domain-specific language to describe the instructions it can emit for a given target.
The RISC-V target describes multiple register sets that RISC-V instructions can use.
Vanilla RVV vector memory accesses use the General Purpose Registers (GPR) to store the base address of each access.
CHERI-Clang added a GPCR set, i.e. the General Purpose Capability Registers.
As noted in \cref{chap:emu:rvv_int_mode} CHERI-RVV requires the vector memory accesses to support integer \emph{and} capability mode, therefore two versions of the vector accesses must be created: versions which take a capability base address, only available in CHERI/Capability mode, and versions which take integer base addresses for Integer mode.

\todomark{Appendix on how this was done?}
With the above changes, inline assembly could be used to insert capability-enabled vector instructions\todomark{Example}.
However, as this requires using a capability register constraint for the base address, inline assembly code written for CHERI-RVV is not inherently compatible with vanilla RVV.
For un-annotated pointers (e.g. \code{int*}), which are only capabilities in pure-capability code and integers in legacy or hybrid code, a conditional macro can be used to insert the correct constraint: \todoref{example}.
However, this falls apart in hybrid code for manually annotated pointers (e.g. \code{int* \_\_capability}) because the macro cannot detect the annotation.


\subsection{Adapting vector intrinsics to CHERI}
Vector intrinsics are another story entirely.
When compiling for pure-capability libraries, all attempts to use vector intrinsics crash CHERI-Clang \todoref{example of error message}.
This is due to a similar issue to inline assembly: the intrinsics (both the Clang intrinsic functions and the underlying LLVM IR intrinsics) were designed to take regular pointers and cannot handle it when capabilities are used instead.
\todoref{Appendix which covers what I know so far about this problem?}
Unfortunately the code for generating the intrinsics on both levels is spread across many files, and there's no simple way to change the associated pointer type (much less changing it for pure capability vs. hybrid mode).

It seems that significant compiler development work is required to bring vector intrinsics up to scratch on CHERI-Clang.
We did experiment with creating replacement wrapper functions, where each function tried to mimic an intrinsic using inline asssembly.
These were rejected for two reasons: the increased overhead of a function call on every vector instruction\footnote{This could have been eliminated by using preprocessor macros instead of real functions, but they are difficult to program and do not easily support returning values like intrinsics do.}, and the lack of support for passing vector types as arguments or return values.
The RISC-V ABI treats all vector registers as temporary and explicitly states that \enquote{vector registers are not used for passing arguments or return values}\todocite{RISCV-ABI}, and CHERI has its own issues with saving vector registers on the stack.

\todomark{segue into saving registers on stack doesn't make sense - doesn't explain why someone would want to}

\subsection{Storing scalable vectors on the stack}
If a program uses more data than can fit in registers, or calls a function which may overwrite important register values, the compiler will save those register values to memory on the stack.
Because vector registers are temporary, and thus may be overwritten by called functions, they must also be saved/restored from the stack\todomark{Example https://godbolt.org/z/KPTW7rcvY}.
RVV provides the whole-register memory access instructions explicitly to make this process easy.

CHERI-Clang contains an LLVM IR pass\footnote{llvm/lib/CodeGen/CheriBoundAllocas.cpp} which enforces strict bounds on so-called ``stack capabilities'' (capabilities pointing to stack-allocated data), which by definition requires knowing the size of the data ahead of time.
This pass assumes all stack-allocated data has a static size, and crashes when dynamically-sized types e.g. scalable vectors are allocated.
It is therefore impossible (for now) to save vectors on the stack in CHERI-Clang, although it's clear that it's theoretically possible.
For example, the length of the required vector allocations could be calculated based on \code{VLEN} before each stack allocation is performed, or if performance is a concern stack bounds for those allocations could potentially be ignored altogether.
These possibilities are investigated further in the next section.

\todocite{LLVM IR pass investigated in https://www.cl.cam.ac.uk/techreports/UCAM-CL-TR-949.pdf \$3.8.2, couldn't find earlier reference to it}

\pagebreak
\section{Testing hypotheses}

\todomark{H-C: At machine level, yes - a program compiled for pure-cap CHERI-RVV will have the exact same vector instruction encodings as a no-capability RVV.
Although if it were recompiled for integer-mode-CHERI it may end up using different registers for the capabilities.
Practically:
Yes under auto-vectorization once added, Yes under vector intrinsics IFF it gets ported, Yes under inline-assembly WITH a simple macro}

\todomark{H-D: Yes IFF vectors can be stored on the stack. May require a dynamic stack bounds calculation based on VLEN.}

\todomark{where to put something on how CHERI-Clang should change?}

\todomark{``unforeseen consequences'' of making instructions compatible with Capability and Integer mode, intrinsics have to be compatible too?}