\documentclass[../thesis]{subfiles}
\begin{document}

\chapter{Conclusion}

This project demonstrated the viability of integrating CHERI with scalable vector models by producing an example CHERI-RVV implementation.
This required both research effort in studying the related specifications (\cite{TR-951,specification-RVV-v1.0}), demonstrated in \cref{chap:background}, and a substantial implementation effort demonstrated in \cref{chap:hardware,chap:software,chap:capinvec}.
% During this project, we developed a large Rust and C codebase (9,120 lines of code) for a helper library (860 LoC), a RISC-V emulator (5,320 LoC), and a set of emulator test programs (2,940 LoC\footnote{\code{vector\_memcpy} uses an autogenerated test program, the generator is included but the extra 2,470 lines of generated code are not.}).
We produced four software artifacts: a Rust wrapper for the \code{cheri-compressed-cap} C library (900 lines of code), a RISC-V emulator supporting multiple architecture extensions (5,300 LoC), a fork of CHERI-Clang supporting CHERI-RVV (400 changed LoC), and test programs for the emulator (3,000 LoC\footnote{This doesn't include automatically generated code.}).
% This includes the \code{vector\_memcpy} generator codebase but not the generated program.
% These are hosted online (see \cref{appx:artifacts}) and were submitted alongside this dissertation.
Developing these artifacts provided enough information to make conclusions for the initial hypotheses (\cref{tab:hypotheses}).

\section{Evaluating hypotheses}

\cref{hyp:hw_cap_as_vec_mem_ref} showed that all memory references can be replaced with capabilities in all RVV instructions while maintaining functionality.
\cref{hyp:hw_cap_bounds_checks_amortized} then alleviated performance concerns by showing it was possible to combine the required capability checks for all vector accesses, amortizing the overall cost of checking, although with varying practical benefit.

On the software side \cref{hyp:sw_vec_legacy,hyp:sw_pure_compat} showed that non-CHERI vectorized code could be run on CHERI systems, and even recompiled for pure-capability platforms with no source code changes, but that CHERI-Clang's current state adds some practical limitations.
We developed the \code{vector\_memcpy} test program to show that despite those limitations, it's possible to write correct CHERI-RVV code on current compilers.
% Despite those limitations, the \code{vector\_memcpy} testbench proves it's possible to write correct CHERI-RVV code on current compilers 
\cref{hyp:sw_stack_vectors,hyp:sw_multiproc} address the pausing and resuming of vector code, specifically saving and restoring variable-length architectural state, concluding that it is entirely possible but requires software adjustments.

Through a limited investigation of capabilities-in-vectors, \cref{hyp:cap_in_vec_storage,hyp:cap_in_vec_load_store,hyp:cap_in_vec_manip} showed that a highly constrained implementation could enable a fully-functional vectorized \code{memcpy}, as demonstrated in the \code{vector\_memcpy\_pointers} test program, without violating CHERI security principles.
It should be possible to extend the CHERI-RVV ISA with vector equivalents of existing CHERI scalar instructions, but we did not investigate this further.

Clearly, scalable vector models can be adapted to CHERI without significant loss of functionality.
Most of the hypotheses are general enough to cover other scalable models, e.g. Arm SVE, but any differences from RVV's model will require careful examination.
Given the importance of vector processing to modern computing, and thus its importance to CHERI, we hope that this research paves the way for future vector-enabled CHERI processors.

\section{Future work}
The stated purpose of this project was to enable future implementations of CHERI-RVV and CHERI Arm SVE.
We've shown this is feasible, and we believe our research is enough to create an initial CHERI-RVV specification, but both could benefit from more research on capabilities-in-vectors.

All architectures may benefit from more advanced vectorized capability manipulation.
Because these processes are still evolving, it may be wise to standardize the first version of CHERI-RVV based on this dissertation and only add new instructions as required.
Once created, the standard can be implemented in CHERI-Clang\footnote{See \cref{chap:software:sec:chericlangchanges} for the other required changes to CHERI-Clang.} and added to existing CHERI-RISC-V processors\footnote{\url{https://www.cl.cam.ac.uk/research/security/ctsrd/cheri/cheri-risc-v.html}}.

More theoretically, other vector models could benefit from \emph{dereferencing} capabilities-in-vectors.
Arm SVE has addressing modes that directly use vector elements as memory references, as do its predecessors and contemporaries.
A draft specification of CHERI-x86 is in the works\cite[Chapter~6]{TR-951}, and existing x86 vector models like AVX have similar features.
This may prove impractical, but this could be mitigated by e.g. replacing these addressing modes with variants of RVV's ``indexed'' mode.
Once this problem is solved, CHERI will be able to match the memory access abilities of any vector ISA it needs to, making it that much easier for industry to adopt CHERI in the long term.
% Once this problem is solved, CHERI support may be extended to any vector models where it would help,
% Solving this problem may be the key to extending CHERI support to 
% Regardless of the exact solution, we believe it is feasible to extend CHERI-support

\end{document}