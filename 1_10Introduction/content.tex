\documentclass[../thesis]{subfiles}
\begin{document}

\chapter{Introduction\label{wc:start}\label{chap:intro}}
% PLAN: 1k words
% 2-3 pages

% \todomark{para on what capabilities are}
Since 2010, the Cambridge Computer Lab (in association with SRI) has been developing the CHERI\footnote{Capability Hardware Enhanced RISC Instructions} architecture extension, which improves the security of any given architecture by checking all memory accesses in hardware.
The core impact of CHERI, on a hardware level, is that memory can no longer be accessed directly through raw addresses, but must pass through a \emph{capability}\cite{TR-941}.
Capabilities are unforgeable tokens that grant fine-grained access to ranges of memory.
Instead of generating them from scratch, capabilities must be \emph{derived} from another capability with greater permissions.
For example, a capability giving read-write access to an array of structures can be used to create a sub-capability granting read-only access to a single element.
This vastly reduces the scope of security violations through spatial errors (e.g. buffer overflows\cite{szekeresSoKEternalWar2013}), and creates interesting opportunities for software compartmentalization\cite{watsonCHERIHybridCapabilitySystem2015}.

Industry leaders have recognized the value CHERI provides.
Arm Inc have manufactured the Morello System-on-Chip, based on their Neoverse N1 CPU, which incorporates CHERI capabilities into the Armv8.2 ISA.
While this represents a great step forward, there are still elements on the SoC that haven't fully embraced CHERI (e.g. the GPU), and architecture extensions that haven't been investigated in the context of CHERI.
One such example is Arm's Scalable Vector Extension (introduced in Armv8.2 but not included in Neoverse N1), which is designed to remain in use well into the future\cite{stephensARMScalableVector2017}.
Supporting this and other scalable vector ISAs in CHERI is essential to CHERI's long-term relevance.

% \todomark{para on what vectors are}
In the context of modern computer architecture, vector processing is the practice of dividing a large hardware register into a \emph{vector} of multiple \emph{elements} and executing the same operation on each element in a single instruction\footnote{This is a SIMD (Single Instruction Multiple Data) paradigm.}.
This data-level parallelism can drastically increase throughput, particularly for arithmetic-heavy programs.
However, before computing arithmetic, the vectors must be populated with data.
% This is typically done using specialized vector memory accesses, which access a whole vector's worth of memory in a single instruction.

% This was the initial motivation for investigating the interactions between CHERI and vector processing.

% \todomark{above paras should explain why the interactions between the two are interesting}

\section{Motivation}
% \todomark{vectorized memory accesses required for vectorized arithmetic}
Modern vector implementations all provide vector load/store instructions to access a whole vector's worth of memory.
These range from simple contiguous accesses (where all elements are next to each other), to complex indexed accesses (where each element loads from a different location based on another vector).
They can also have per-element semantics, e.g. ``elements must be loaded in order, so if one element fails the preceding elements are still valid''\cite[Section~7.7]{specification-RVV-v1.0}.
% \todomark{Necessary for CHERI support in order for CHERI performance to be competitive}
If CHERI CPUs want to benefit from vector processing's increased performance and throughput, they must support those instructions at some level.
But adding CHERI's bounds-checking to the mix may affect these semantics, and could impact performance (e.g. checking each element's access in turn may be slow).

% \todomark{if this were the only priority, vectorized memory access doesn't need to be that fast relative to arithmetic?}
\label{vectorized_memcpy}Vector memory access performance is more critical than one may initially assume, because vectors are used for more than just computation.
% \todomark{memcpy is a very common operation, vectorized on many platforms (cite)}
A prime example is \code{memcpy}: for \code{x86\_64}, \code{glibc} includes multiple versions of the function\footnote{It appears memcpy is implemented as a copy of memmove.} taking advantage of vector platforms, then selects one to use at runtime\footnote{\gitfile{sysdeps/x86_64/multiarch/ifunc-memmove.h}{bminor/glibc}{https://github.com/bminor/glibc/blob/7b1cfba79ee54221ffa7d7879433b7ee1728cd76/sysdeps/x86_64/multiarch/ifunc-memmove.h}}.
% \todomark{thus memory accesses themselves should be optimized}
These implementations are written in assembly and heavily optimized.
If the memory accesses are hitting the cache, a few extra cycles of bounds-checking for each access could actually make a noticeable difference.

% \todomark{also raises the question of storing capabilities within vectors, which could be useful for revocation strategies as proposed in XYZ}
\code{memcpy} also raises the important question of how the vector model interacts with capabilities.
In non-CHERI processors, \code{memcpy} will copy pointers around in memory without fuss.
For a CHERI-enabled vector processor to support this, it would need to be able to load/store capabilities from vectors without violating any security guarantees.
This may add more constraints --- for example, each vector register likely needs to be at least as large as a single capability.

% \todomark{rvv introduces vectors for RVV}
% \todomark{rvv is scalable vector}
% \todomark{why rvv and not just arm}
% \todomark{CHERI has a RISC-V ISA + various implementations, wants to support vectors}
% \todomark{scalable vector support useful for future Arm support}
% \todomark{Current Morello doesn't support Arm SVE/NEON (cite?) despite the base Neoverse N1 supporting it (cite??)}
To explore this topic, we chose to focus on the RISC-V Vector extension\cite{specification-RVV-v1.0} (shortened to RVV throughout).
As of November 2021 this has been ratified by RISC-V International\footnote{\url{https://wiki.riscv.org/display/HOME/Recently+Ratified+Extensions}}, and will be RISC-V's standard vector instruction set moving forward.
This has two key benefits.
Studying RVV will allow reference \enquote{CHERI-RVV} implementations to be built for the CHERI project's open-source RISC-V cores\footnote{\url{https://www.cl.cam.ac.uk/research/security/ctsrd/cheri/cheri-risc-v.html}}, which don't currently support vector processing.
Secondly, RVV is a \emph{scalable} vector model, where the length of each vector is implementation-dependent.
This has more potential roadblocks than a fixed-length vector model, and investigating them here will make life easier if Arm wish to integrate their Scalable Vector Extension with CHERI later down the road.

% \section{Aims}
% The aim of this project is to investigate the impact of, and the roadblocks for, integrating a scalable vector architecture with CHERI's memory protection system.
% In particular, we focus on integrating RVV with the CHERI-RISC-V ISA, with the aim of enabling a future CHERI-RVV implementation and informing the approach for a future CHERI Arm SVE implementation.

% % Investigate the reprecussions of adding scalable vector support to CHERI systems.
% % \todomark{Identify where potential roadblocks could be, and where they couldn't be.}
% % What are potential performance/compatibility roadblocks?
% This project does not try to implement CHERI-RVV in a hardware description language.
% Instead, it uses an emulator written in Rust to experiment with potential integration methods.
% A higher-level approach allows us to evaluate what's plausible for hardware without getting stuck in the weeds of implementing it in full.

% % \todomark{Investigate from both hardware and software perspectives}
% Software and the user story for writing CHERI-RVV programs is also considered. 
% Where possible, vector code is generated from C source code using the CHERI-Clang compiler, rather than hand-writing assembly.
% This reveals many problems CHERI-Clang has with vector intrinsics, which we later propose solutions to.

% % \todomark{Investigate from both (adding CHERI to vector) and (adding vectors to CHERI) perspectives}
% As scalar CHERI processors and vector non-CHERI processors both already exist, we consider integration from both sides: how does vector processing change when adding CHERI, and how does CHERI change when adding vectors?
% Examples of these perspectives include the following:
% \begin{itemize}
%     \item When adding CHERI to vectors, do vectors need to be able to hold capabilities? If so, what changes are necessary?
%     \item When adding vectors to scalar CHERI, what are the benefits/drawbacks of using a separate hardware vector unit vs. an integrated scalar+vector pipeline?
% \end{itemize}

% % How does CHERI impact vector implementations (i.e. storing capabilities inside vectors for generic memcpy)
% % How do vectors impact CHERI implementations (particularly scalable vectors?)

\section{Hypotheses and Aims}
The goal of this project is to investigate the impact of, and the roadblocks for, integrating a scalable vector architecture with CHERI's memory protection system.
In particular, we focus on integrating RVV with the CHERI-RISC-V ISA, with the aim of enabling a future CHERI-RVV implementation and informing the approach for a future CHERI Arm SVE implementation.

The investigation was carried out by designing and testing a CHERI-RVV emulator written in Rust, but that is only a single implementation.
To show that RVV can be integrated with CHERI-RISC-V for a wide range of processors, we use information gathered from the emulator to check nine hypotheses (\cref{tab:hypotheses}).
\cref{hyp:hw_cap_as_vec_mem_ref,hyp:hw_cap_bounds_checks_amortized} consider basic feasibility and potential hardware performance issues.
\cref{hyp:sw_vec_legacy,hyp:sw_pure_compat,hyp:sw_stack_vectors,hyp:sw_multiproc} ensure that vector software is compatible with potential CHERI-RVV software stacks.
\cref{hyp:cap_in_vec_storage,hyp:cap_in_vec_load_store,hyp:cap_in_vec_manip} considers capabilities-in-vectors: the conditions under which vector registers can hold capabilities, and vectorized instructions can manipulate them.

This document also examines current practical realities, such as the limitations of the current CHERI-RVV software stack, and explains how to compile/execute CHERI-RVV code on bare-metal platforms.
The RVV and CHERI-RISC-V specifications are succinctly explained in \cref{chap:background}.

% Along with testing these hypotheses, this document serves as a practical explanation of the following:
% \begin{itemize}
%     \item The changes to RISC-V required by CHERI-RISC-V
%     \item The RISC-V-V vector model, and all memory-related instructions
%     \item How to compile CHERI-RVV code for a bare-metal platform (with no operating system)
%     \item How to execute CHERI-RVV code compiled for a bare-metal platform
%     \item The limitations of the current CHERI-RVV software stack
% \end{itemize}
% \todomark{fill out this para more, have an actual end sentence}

\vfill\null

% Count hypothesis words - normally they shouldn't be counted, because they're inside a table,
% but we can't easily count when they're reproduced in later sections.
% Count them once here, and not after, which is equivalent to not counting them here but counting after.
%TC:macro \definehyp [ignore,text]
%TC:group table 0 1
%TC:group tabular 1 1
\newcommand{\definehyp}[2]{{\sethyptext{#1}{#2}}\newhyp{}\label{#1} & #2}
\begin{table}[h]
    \centering
    \begin{tabular}{lp{0.9\linewidth}}
    \toprule
    \multicolumn{2}{l}{\emph{Hardware Hypotheses --- \cref{chap:hardware}}} \\
    \midrule
        \definehyp{hyp:hw_cap_as_vec_mem_ref}%
            {It is possible to use CHERI capabilities as memory references in all vector instructions.} \\
        \definehyp{hyp:hw_cap_bounds_checks_amortized}%
            {The capability bounds checks for vector elements within a known range (e.g. a cache line) can be performed in a single check, amortizing the cost.} \\
    \midrule
    \multicolumn{2}{l}{\emph{Software Hypotheses --- \cref{chap:software}}} \\
    \midrule
        \definehyp{hyp:sw_vec_legacy}%
            {Vector code can be compiled in legacy forms (with integer addressing) and function correctly on CHERI with no source code changes.} \\
        \definehyp{hyp:sw_pure_compat}%
            {Legacy vector code can be compiled into a pure-capability form with no changes.} \\
        \definehyp{hyp:sw_stack_vectors}%
            {Vector code that saves/restores variable-length vectors to/from the stack can be compiled on CHERI-RVV with no source code changes.} \\
        \definehyp{hyp:sw_multiproc}%
            {CHERI-vector code can run correctly in multiprocessing systems, where execution may be paused and resumed on interrupts or context switches.} \\
    \midrule
    \multicolumn{2}{l}{\emph{Capabilities-in-Vectors --- \cref{chap:capinvec}}} \\
    \midrule
        \definehyp{hyp:cap_in_vec_storage}%
            {It is possible for vector registers to hold capabilities to enable copying without violating CHERI security principles.} \\
        \definehyp{hyp:cap_in_vec_load_store}%
            {It is possible for vector memory accesses to load and store capabilities from vector registers without violating CHERI security principles.} \\
        \definehyp{hyp:cap_in_vec_manip}%
            {It is possible for vector instructions to manipulate capabilities in vector registers without violating CHERI security principles.} \\
    \bottomrule
    \end{tabular}
    \caption{Project Hypotheses}%
    \label{tab:hypotheses}
\end{table}
% This table was an exception, usually don't count tables
%TC:group table 0 0
%TC:group tabular 1 0

\end{document}