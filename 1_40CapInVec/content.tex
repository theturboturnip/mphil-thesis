\chapter{Capabilities-in-vectors\label{chap:capinvec}}
Implementing \code{memcpy} correctly for CHERI systems requires copying the tag bits as well as the data.
As it stands, any vectorized \code{memcpy} compiled and executed on the systems described in \cref{chap:software,chap:hardware} will not copy the tag bits, because the vector registers cannot store the tag bits and indeed cannot store valid capabilities.
This chapter examines the changes made to the emulator to support storing capabilities-in-vectors, and determines the conditions required for the related hypotheses to be true.
\cref{appx:emucapinvec} lists all the relevant properties of the emulator that allow storing capabilities in vectors.

\section{Changing the emulator}
This part of the project aimed to investigate \cref{hyp:cap_in_vec_storage,hyp:cap_in_vec_load_store,hyp:cap_in_vec_manip}, which led to the following goals:
\begin{itemize}
    \item (\cref{hyp:cap_in_vec_storage}) Vector registers should be able to hold capabilities 
    \item (\cref{hyp:cap_in_vec_load_store}) At least one vector memory operation should be able to load/store capabilities from vectors
    \begin{itemize}
        \item Because \code{memcpy} should copy both integer and capability data, the vector memory operations should be able to handle both
    \end{itemize}
    \item (\cref{hyp:cap_in_vec_manip}) Vector instructions should be able to affect capabilities in some way
    \begin{itemize}
        \item Clearing the tag bit on a vector register counts as manipulation
    \end{itemize}
\end{itemize}

First, we had to consider the impact of vectors on the theoretical vector model.
We decided that any operation with element widths less than \code{CLEN} cannot output valid capabilities under any circumstances\footnote{This avoids edge cases with masking, where one part of a capability could be modified while the other parts are left alone.}.
This, of course, means a new element width equal to \code{CLEN} must be introduced.
We set \code{ELEN = VLEN = CLEN = 128}\footnote{The tag bits are implicitly instead of explicitly included here because \code{VLEN,ELEN} must be powers of two.} for our vector unit.

Two new memory access instructions were created to take advantage of this new element width, and the \code{vsetvl} family were adjusted to support 128-bit values.
Similar to the CHERI-RISC-V \code{LC/SC} instructions, we implemented 128-bit unit-stride vector loads and stores, which took over officially-reserved encodings\footnote{The RVV spec mentions, but does not specify, potential encodings for 128-bit element widths and instructions (\cite[p10, p32]{specification-RVV-v1.0}, \cref{tab:capinvec:accesswidth}).} we expected official versions to use.
The memory instructions had to be added to CHERI-Clang manually, and Clang already has support for setting \code{SEW=128} in the \code{vsetvl} family (\cref{tab:capinvec:vtypewidth}).
These instruction changes affected inline assembly only, rather than adding vector intrinsics, because CHERI-Clang only supports inline assembly anyway.

\begin{table}[]
    \centering
    \begin{tabular}{l|ccc}
        \multicolumn{1}{c}{SEW} & \multicolumn{3}{c}{\code{vsew[2:0]}} \\
        8 & 0 & 0 & 0 \\
        16 & 0 & 0 & 1 \\
        32 & 0 & 1 & 0 \\
        64 & 0 & 1 & 1 \\
        128$^{\text{new}}$ & 1 & 0 & 0 \\
    \end{tabular}
    \caption{Selected element width encoding}
    \label{tab:capinvec:vtypewidth}
\end{table}
\begin{table}[]
    \centering
    \begin{tabular}{l|cccc}
        & \code{mew} & \multicolumn{3}{c}{\code{width[2:0]}} \\
        Vector 8b element & 0 & 0 & 0 & 0 \\
        Vector 16b element & 0 & 1 & 0 & 1 \\
        Vector 32b element & 0 & 1 & 1 & 0 \\
        Vector 64b element & 0 & 1 & 1 & 1 \\
        Vector 128b element & 1 & 0 & 0 & 0 \\
    \end{tabular}
    \caption{Width encoding for vector loads and stores}
    \todomark{tabularx for evenly spacing columns here}
    \label{tab:capinvec:accesswidth}
\end{table}

The next step was to add capability support to the vector register file.
Our approach to capabilities-in-vectors is similar in concept to the Merged scalar register file for CHERI-RISC-V (\cref{chap:bg:subsec:cherimergedreg}), in that the same bits of a register can be accessed in two contexts: an integer context, zeroing the tag, or a capability context which maintains the current tag.
The only instructions which can access data in a capability context are the aforementioned 128-bit memory accesses\footnote{The encoding mode (\cref{chap:bg:subsec:cheriencodingmode}) does not affect register usage: when using the Integer encoding mode, instructions can still access the vector register in a capability context. This is just like how scalar capability registers are still accessible in Integer encoding mode.}.
All other instructions will read out untagged integer data, and clear tag bits when writing data.
A new CHERI-specific vector register file was created, where each register is a \code{SafeTaggedCap} \todoref{place in software chapter where I discuss that :D} i.e. either zero-tagged integer data or a valid tagged capability.

Using \code{SafeTaggedCap} had a few major consequences.
Firstly, \code{SafeTaggedCap} enforces the Provenance security property within the vector unit as described in \todoref{}.
Secondly, it reuses the code path for accessing scalar capabilities in memory, so all related security properties are maintained (e.g. accesses must be 128-bit aligned, and are atomic).

\subsection{Testing}
The above emulator changes allowed a minimal \code{memcpy} example to be constructed, which could copy both capabilities and integer data mixed together.
This, along with another test to ensure arithmetic accesses the register file in an integer context, is described in the evaluation (\cref{chap:eval}).

\section{Testing hypotheses}

\hypsubsection{hyp:cap_in_vec_storage}{Holding capabilities in vectors}
\hypsubsection{hyp:cap_in_vec_load_store}{Sending capabilities between vectors and memory}
\hypsubsection{hyp:cap_in_vec_manip}{Manipulating capabilities in vectors}
