\documentclass[../thesis]{subfiles}
\begin{document}

\chapter{Capabilities-in-vectors\label{chap:capinvec}}
Implementing \code{memcpy} correctly for CHERI systems requires copying the tag bits as well as the data.
As it stands, any vectorized \code{memcpy} compiled and executed on the systems described in \cref{chap:software,chap:hardware} will not copy the tag bits, because the vector registers cannot store the tag bits and indeed cannot store valid capabilities.
This chapter examines the changes made to the emulator to support storing capabilities-in-vectors, and determines the conditions required for the related hypotheses to be true.
\cref{appx:capinvec} lists the changes made and all the relevant properties of the emulator that allow storing capabilities in vectors.

\section{Changing the emulator}
This part of the project aimed to investigate \cref{hyp:cap_in_vec_storage,hyp:cap_in_vec_load_store,hyp:cap_in_vec_manip}, which led to the following goals:
\begin{itemize}
    \item (\cref{hyp:cap_in_vec_storage}) Vector registers should be able to hold capabilities 
    \item (\cref{hyp:cap_in_vec_load_store}) At least one vector memory operation should be able to load/store capabilities from vectors
    \begin{itemize}
        \item Because \code{memcpy} should copy both integer and capability data, the vector memory operations should be able to handle both
    \end{itemize}
    \item (\cref{hyp:cap_in_vec_manip}) Vector instructions should be able to affect capabilities in some way
    \begin{itemize}
        \item Clearing the tag bit on a vector register counts as manipulation
    \end{itemize}
\end{itemize}

First, we had to consider the impact of vectors on the theoretical vector model.
We decided that any operation with element widths less than \code{CLEN} cannot output valid capabilities under any circumstances\footnote{This avoids edge cases with masking, where one part of a capability could be modified while the other parts are left alone.}.
This, of course, means a new element width equal to \code{CLEN} must be introduced.
We set \code{ELEN = VLEN = CLEN = 128}\footnote{The tag bits are implicitly instead of explicitly included here because \code{VLEN,ELEN} must be powers of two.} for our vector unit.

Two new memory access instructions were created to take advantage of this new element width, and the \code{vsetvl} family were adjusted to support 128-bit values.
Similar to the CHERI-RISC-V \code{LC/SC} instructions, we implemented 128-bit unit-stride vector loads and stores, which took over officially-reserved encodings\footnote{The RVV spec mentions, but does not specify, potential encodings for 128-bit element widths and instructions (\cite[p10, p32]{specification-RVV-v1.0}, \cref{tab:capinvec:accesswidth}).} we expected official versions to use.
The memory instructions had to be added to CHERI-Clang manually, and Clang already has support for setting \code{SEW=128} in the \code{vsetvl} family (\cref{tab:capinvec:vtypewidth}).
These instruction changes affected inline assembly only, rather than adding vector intrinsics, because CHERI-Clang only supports inline assembly anyway.

% \begin{table}[h]
    \centering
    \begin{minipage}[c]{.49\textwidth}
        \centering
    \begin{tabular}{lx{0.5cm}x{0.5cm}x{0.5cm}}
        \multicolumn{1}{c}{SEW} & \multicolumn{3}{c}{\code{vsew[2:0]}} \\
        \cmidrule(lr){2-4}
        8 & 0 & 0 & 0 \\
        16 & 0 & 0 & 1 \\
        32 & 0 & 1 & 0 \\
        64 & 0 & 1 & 1 \\
        128$^{\text{new}}$ & 1 & 0 & 0 \\
    \end{tabular}
    \captionof{table}{Selected element width encoding}
    \label{tab:capinvec:vtypewidth}
\end{minipage}\hfill
\begin{minipage}[c]{.49\textwidth}
    \centering
    \begin{tabular}{lcx{0.5cm}x{0.5cm}x{0.5cm}}
        Access Type & \code{mew} & \multicolumn{3}{c}{\code{width[2:0]}} \\
        \cmidrule(lr){2-2} \cmidrule(lr){3-5}
        Vector(8) & 0 & 0 & 0 & 0 \\
        Vector(16) & 0 & 1 & 0 & 1 \\
        Vector(32) & 0 & 1 & 1 & 0 \\
        Vector(64) & 0 & 1 & 1 & 1 \\
        Vector(128)$^{\text{new}}$ & 1 & 0 & 0 & 0 \\
    \end{tabular}
    \captionof{table}{Width encoding for vector loads and stores}\label{tab:capinvec:accesswidth}
\end{minipage}
\end{table}


The next step was to add capability support to the vector register file.
Our approach to capabilities-in-vectors is similar in concept to the Merged scalar register file for CHERI-RISC-V (\cref{chap:bg:subsec:cherimergedreg}), in that the same bits of a register can be accessed in two contexts: an integer context, zeroing the tag, or a capability context which maintains the current tag.
The only instructions which can access data in a capability context are the aforementioned 128-bit memory accesses\footnote{The encoding mode (\cref{chap:bg:subsec:cheriencodingmode}) does not affect register usage: when using the Integer encoding mode, instructions can still access the vector register in a capability context. This is just like how scalar capability registers are still accessible in Integer encoding mode.}.
All other instructions will read out untagged integer data, and clear tag bits when writing data.
A new CHERI-specific vector register file was created, where each register is a \code{SafeTaggedCap} (\cref{safetaggedcap}) i.e. either zero-tagged integer data or a valid tagged capability.

\todomark{This paragraph is bad. Plan: split the above paragraph into two, incorporate impact on Provenance and required memory accesses into second half.}
Using \code{SafeTaggedCap} had a few major consequences.
Firstly, \code{SafeTaggedCap} enforces the Provenance security property within the vector unit.
Secondly, it reuses the code path for accessing scalar capabilities in memory, so all related security properties are maintained (e.g. accesses must be 128-bit aligned, and are atomic).

\subsection{Testing}
The above emulator changes allowed a minimal \code{memcpy} example to be constructed, which could copy both capabilities and integer data mixed together.
This, along with another test to ensure arithmetic accesses the register file in an integer context, is described in the evaluation (\cref{chap:eval}).

\section{Testing hypotheses}

\hypsubsection{hyp:cap_in_vec_storage}{Holding capabilities in vectors}
It is possible for a single vector register to hold a capability (and differentiate a capability from integer data) as long as \code{VLEN = CLEN}.
\code{VLEN} could also be larger, and a compliant implementation must then have \code{VLEN} be an integer multiple of \code{CLEN}.
In theory, one could also describe a scheme where capabilities must be held by multiple registers together (e.g. \code{VLEN = CLEN/2} with one tag bit for every two registers), but this would complicate matters.

If an implementation decides, as we did, that elements of width \code{CLEN} are required to produce capabilities, then $\code{VLEN} \ge \code{ELEN}$ therefore $\code{VLEN} \ge \code{CLEN}$.
If a short \code{VLEN} is absolutely essential, one could place precise guarantees on a specific set of instructions to enable it (e.g. \code{SEW=64, LMUL=2} unit-stride unmasked loads could guarantee atomic capability transfers) but the emulator does not consider this.
The CHERI security properties also impose some conditions.

\subsubsection*{Provenance \& Monotonicity}
The tag bit must be protected such that capabilities cannot be forged from integer data.
The emulator's integer/capability context approach, where the tag bit may only be set on copying a valid capability from memory, and the output tag bit is zeroed on all other accesses, enforces this correctly.

\subsubsection*{Integrity}
Integrity is not affected by how a capability is stored, as long as the other properties are maintained.

%%%%%%%%%%%%%%%%%%%%%%%%%%%%%%%%%%%%%%%%%%%%%%%%%%%%%%%%%%%%%%%%%%%%%%%%%%%%%%%%%%%%%%%
\hypsubsection{hyp:cap_in_vec_load_store}{Sending capabilities between vectors and memory}\label{chap:capinvec:hyp_load_store}
For this to be the case, the instructions which can load/store capabilities must fulfil certain alignment and atomicity requirements.
They must require all accesses be \code{CLEN}-aligned, or at least only load valid caps from aligned addresses, because tag bits only apply to \code{CLEN}-aligned regions.
TR-951 states that capability memory accesses must be atomic\cite[Section 11.3]{TR-951}.
This applies to vectors, even in ways that don't apply to scalar accesses.

Individual element accesses for a vector access must be atomic relative to each other.
This is relevant for e.g. a strided store uses a zero or unaligned stride, such that one element writes a valid capability and another element overwrites part of that address range.
In that case, either the second element should ``win'' and clear the tag bit, or the first element should ``win'' and write the full capability atomically.
For the emulator this condition is easily met.

\subsubsection*{Provenance}
Provenance requires the accesses be atomic as described above, and require that tag bits are copied correctly: the output tag bit must only be set if the input had a valid tag bit.
These conditions also apply to scalar accesses.

\subsubsection*{Monotonicity}
These loads/stores do not attempt to manipulate capabilities, so have no relevance to monotonicity.

\subsubsection*{Integrity}
The same conditions for scalar and other vector accesses apply to maintain Integrity: namely that the base capability for each access should be checked to ensure it is valid.
The emulator doesn't allow capabilities-in-vectors to be dereferenced directly, but if an implementation allows it those capabilities would also need to be checked.

%%%%%%%%%%%%%%%%%%%%%%%%%%%%%%%%%%%%%%%%%%%%%%%%%%%%%%%%%
\hypsubsection{hyp:cap_in_vec_manip}{Manipulating capabilities in vectors}
The emulator limits all manipulation to clearing the tag bit, achieved by writing data to the register in an integer context.
In theory, it's possible to do more complex transformations, which can be proven by implementing each vector manipulation on vector elements as sequential scalar manipulations on scalar elements.
With this method, all pre-existing scalar capability manipulations can become vector manipulations, but the utility seems limited.
For example, instructions for creating capabilities or manipulating bounds en masse don't have an obvious use case.
If more transformations are added they should be considered carefully, rather than creating vector equivalents for all scalar manipulations.

\subsubsection*{Provenance \& Monotonicity}
Because the only possible manipulations clear the tag bit, it's impossible to create or change capabilities, so Provenance and Monotonicity cannot be violated.
Any manipulations that create capabilities, or potentially any manipulations that transfer capabilities from vector registers directly to scalar registers, would require more scrutiny.

\subsubsection*{Integrity}
As stated before, capabilities-in-vectors cannot be dereferenced directly, so there is no impact on Integrity.


\end{document}