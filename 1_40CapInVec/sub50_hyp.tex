\section{Testing hypotheses}

\hypsubsection{hyp:cap_in_vec_storage}{Holding capabilities in vectors}
It is possible for a single vector register to hold a capability (and differentiate a capability from integer data) as long as \code{VLEN = CLEN}.
\code{VLEN} could also be larger, and a compliant implementation must then have \code{VLEN} be an integer multiple of \code{CLEN}.
In theory, one could also describe a scheme where capabilities must be held by multiple registers together (e.g. \code{VLEN = CLEN/2} with one tag bit for every two registers), but this would complicate matters.

If an implementation decides, as we did, that elements of width \code{CLEN} are required to produce capabilities, then $\code{VLEN} \ge \code{ELEN}$ therefore $\code{VLEN} \ge \code{CLEN}$.
If a short \code{VLEN} is absolutely essential, one could place precise guarantees on a specific set of instructions to enable it (e.g. \code{SEW=64, LMUL=2} unit-stride unmasked loads could guarantee atomic capability transfers) but the emulator does not consider this.
The CHERI security properties also impose some conditions.

\subsubsection*{Provenance \& Monotonicity}
The tag bit must be protected such that capabilities cannot be forged from integer data.
The emulator's integer/capability context approach, where the tag bit may only be set on copying a valid capability from memory, and the output tag bit is zeroed on all other accesses, enforces this correctly.

\subsubsection*{Integrity}
Integrity is not affected by how a capability is stored, as long as the other properties are maintained.

%%%%%%%%%%%%%%%%%%%%%%%%%%%%%%%%%%%%%%%%%%%%%%%%%%%%%%%%%%%%%%%%%%%%%%%%%%%%%%%%%%%%%%%
\hypsubsection{hyp:cap_in_vec_load_store}{Sending capabilities between vectors and memory}
For this to be the case, the instructions which can load/store capabilities must fulfil certain alignment and atomicity requirements.
They must require all accesses be \code{CLEN}-aligned, or at least only load valid caps from aligned addresses, because tag bits only apply to \code{CLEN}-aligned regions.
TR-951 states that capability memory accesses must be atomic\cite[Section 11.3]{TR-951}.
This applies to vectors, even in ways that don't apply to scalar accesses.

Individual element accesses for a vector access must be atomic relative to each other.
This is relevant for e.g. a strided store uses a zero or unaligned stride, such that one element writes a valid capability and another element overwrites part of that address range.
In that case, either the second element should ``win'' and clear the tag bit, or the first element should ``win'' and write the full capability atomically.
For the emulator this condition is easily met.

\subsubsection*{Provenance}
Provenance requires the accesses be atomic as described above, and require that tag bits are copied correctly: the output tag bit must only be set if the input had a valid tag bit.
These conditions also apply to scalar accesses.

\subsubsection*{Monotonicity}
These loads/stores do not attempt to manipulate capabilities, so have no relevance to monotonicity.

\subsubsection*{Integrity}
The same conditions for scalar and other vector accesses apply to maintain Integrity: namely that the base capability for each access should be checked to ensure it is valid.
The emulator doesn't allow capabilities-in-vectors to be dereferenced directly, but if an implementation allows it those capabilities would also need to be checked.

%%%%%%%%%%%%%%%%%%%%%%%%%%%%%%%%%%%%%%%%%%%%%%%%%%%%%%%%%
\hypsubsection{hyp:cap_in_vec_manip}{Manipulating capabilities in vectors}
The emulator limits all manipulation to clearing the tag bit, achieved by writing data to the register in an integer context.
In theory, it's possible to do more complex transformations, which can be proven by implementing each vector manipulation on vector elements as sequential scalar manipulations on scalar elements.
With this method, all pre-existing scalar capability manipulations can become vector manipulations, but the utility seems limited.
For example, instructions for creating capabilities or manipulating bounds en masse don't have an obvious use case.
If more transformations are added they should be considered carefully, rather than creating vector equivalents for all scalar manipulations.

\subsubsection*{Provenance \& Monotonicity}
Because the only possible manipulations clear the tag bit, it's impossible to create or change capabilities, so Provenance and Monotonicity cannot be violated.
Any manipulations that create capabilities, or potentially any manipulations that transfer capabilities from vector registers directly to scalar registers, would require more scrutiny.

\subsubsection*{Integrity}
As stated before, capabilities-in-vectors cannot be dereferenced directly, so there is no impact on Integrity.
