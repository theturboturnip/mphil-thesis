\chapter{Hardware emulation investigation}
In order to experiment with integrating CHERI and RVV, we implemented a RISC-V emulator in the Rust programming language named \enquote{riscv-v-lite}.
The emulator can emulate four unprivileged\footnote{i.e. entirely bare-metal without privilege levels for OSs or hypervisors.} RISC-V ISAs (\cref{tab:emu_arches}), and was also used as the base for capabilities-in-vectors research (\todoref{capinvec}).
This chapter explores the development of the emulator, the implementation of CHERI support (including supplementary libraries), the addition of vector support, and the conclusions drawn about the integration of CHERI and RVV (referred to as CHERI-RVV throughout).

\begin{table}[h]
    \centering
    \begin{tabular}{cll}
    \toprule
    \multicolumn{2}{c}{Architecture} & Extensions \\
    \midrule
    32-bit & \code{rv32imv} & Multiply, CSR, Vector  \\
    64-bit & \code{rv64imv} & Multiply, CSR, Vector  \\
    64-bit & \code{rv64imvxcheri} & Multiply, CSR, Vector, CHERI  \\
    64-bit & \code{rv64imvxcheri-int} & Multiply, CSR, Vector, CHERI (Integer)  \\
    \bottomrule
    \end{tabular}
    \caption{\code{riscv-v-lite} supported architectures}
    \label{tab:emu_arches}
\end{table}

\section{Developing the emulator}\label{chap:software:sec:emu}

The emulator for each architecture follows a similar pattern.
A \code{Processor} struct stores the register file and the available RAM.
A separate \code{ProcessorModules} struct holds all ISA modules the processor can execute (e.g. the base RV64 Integer ISA, the Multiply extension, and the Vector extension).

The gap between the \code{Processor} and the ISA modules is bridged by a module-specific ``connector'' struct, which holds references to data in the \code{Processor} that is required by the ISA module.
For example, the RV64 Integer ISA's connector contains the current PC, a virtual reference to a register file, and a virtual reference to memory.
This allows different \code{Processor} structs (e.g. a normal RV64 and a CHERI-enabled RV64) to reuse the same ISA modules despite using different register file implementations.

Each \code{Processor} implements a single stage pipeline.
Instructions are fetched, decoded with a common decoder function\footnote{The decoder, and therefore all emulated processors, doesn't support RISC-V Compressed instructions.}, and executed.
The processor asks each ISA module in turn if it wants to handle the instruction, and uses the first module to say yes.
If the ISA module returns a new PC value it is immediately applied, otherwise it is automatically incremented.
This structure easily represents basic RISC-V architectures, and can scale up to support many different new modules.

\subsection{Emulating CHERI}

Manipulating CHERI capabilities securely and correctly is a must for any CHERI-enabled emulator.
Capability encoding logic is not trivial by any means, so the \code{cheri-compressed-cap} C library was re-used rather than implementing it from scratch.
Rust has generally decent interoperability with C, but some of the particulars of this library caused issues.

\subsubsection{\code{rust-cheri-compressed-cap}}
\code{cheri-compressed-cap} provides two versions of the library by default, for 64-bit and 128-bit capabilities, which are generated from a common source through extensive use of the preprocessor.
Each variant defines a set of preprocessor macros (e.g. the widths of various fields) before including two common header files \code{cheri\_compressed\_cap\_macros.h} and \code{cheri\_compressed\_cap\_common.h}.
The latter then defines every relevant structure or function based on those preprocessor macros.
For example, a function \code{compute_base_top} is generated twice, once as  \code{cc64\_decompress\_mem} returning \code{cc64\_cap\_t} and another time as \code{cc128\_decompress\_mem} returning \code{cc128\_cap\_t}.
Elegantly capturing both sets was the main challenge for the Rust wrapper.

One of Rust's core language elements is the Trait --- a set of functions and \enquote{associated types} that can be \emph{implemented} for any type.
This gives a simple way to define a consistent interface: define a trait \code{CompressedCapability} with all of the functions from the common headers, and implement it for two empty structures \code{Cc64} and \code{Cc128}.
In the future, this would allow the Morello versions of capabilities to be added easily.
A struct \code{CcxCap<T>} is also defined which uses specific types for addresses and lengths pulled from a \code{CompressedCapability}.
For example, the 64-bit capability structure holds a 32-bit address, and the 128-bit capability a 64-bit address.

128-bit capabilities can cover a 64-bit address range, and thus can have a length of $2^{64}$.
Storing this length requires 65-bits, so all math in \code{cheri-compressed-cap} uses 128-bit length values.
C doesn't have any standardized 128-bit types, but GCC and LLVM provide so-called ``extension types'' which are used instead.
Although the x86-64 ABI does specify how 128-bit values should be stored and passed as arguments\cite{specification-x86-psABI-v1.0}, these rules do not seem consistently applied\footnote{See \url{https://godbolt.org/z/qj43jssr6} for an example.}.
This causes great pain to anyone who needs to pass them across a language boundary.

Rust explicitly warns against passing 128-bit values across language boundaries, and the Clang User's Manual even states that passing \code{i128} by value is incompatible with the Microsoft x64 calling convention\footnote{\gitfile[release/13.x]{clang/docs/UsersManual.rst:3384}{llvm/llvm-project}{https://github.com/llvm/llvm-project/blob/release/13.x/clang/docs/UsersManual.rst\#x86}}.
This could be resolved through careful examination: for example, on LLVM 128-bit values are passed to functions in two 64-bit registers\footnote{\gitfile[release/13.x]{clang/lib/CodeGen/TargetInfo.cpp:2811}{llvm/llvm-project}{https://github.com/llvm/llvm-project/blob/75e33f71c2dae584b13a7d1186ae0a038ba98838/clang/lib/CodeGen/TargetInfo.cpp\#L2811}}, which could be replicated in Rust by passing two 64-bit values.
For convenience, we instead rely on the Rust and Clang compilers using compatible LLVM versions and having identical 128-bit semantics.

The CHERI-RISC-V documentation contains formal specifications of all the new CHERI instructions, expressed in the Sail architecture definition  language\footnote{\gitrepo{rems-project/sail}{https://github.com/rems-project/sail}}.
These definitions are used in the CHERI-RISC-V formal model\footnote{\gitrepo{CTSRD-CHERI/sail-cheri-riscv}{https://github.com/CTSRD-CHERI/sail-cheri-riscv}}, and require a few helper functions (see~\cite[Chapter~8.2]{TR-951}).
To make it easier to port the formal definitions directly into the emulator the \code{rust-cheri-compressed-cap} library also defines those helper functions.

The above work is available online\footnote{\redact{\gitrepo{theturboturnip/cheri-compressed-cap}{https://github.com/theturboturnip/cheri-compressed-cap}}}, and includes documentation for all C functions (which is not documented in the main repository).
That documentation is also available online\footnote{\redact{\url{https://theturboturnip.github.io/files/doc/rust_cheri_compressed_cap/}}} and partially reproduced in \cref{appx:docs:rustcherilib}.

\subsubsection{Integrating into the emulator}
% i.e. using MemoryOf trait to make all memory addressable only by capabilities
Integrating capabilities into the emulator was relatively simple thanks to the modular emulator structure.
A capability-addressed memory type was created, which wraps a simple integer-addressed memory in logic which performs the relevant capability checks.
For integer encoding mode, a further integer-addressed memory type was created where integer addresses are bundled with the DDC before passing through to a capability-addressed memory (see \cref{fig:emulatormemory}).
% For integer encoding mode, a further integer-addressed memory type was created which wraps the capability addressed mode, where all integer addresses are bundled with the DDC before passing through to the capability-addressed memory.
Similarly, a merged capability register file type was created that exposed integer-mode and capability-mode accesses.
This layered approach meant code for basic RV64I operations did not need to be modified to handle CHERI at all --- simply passing the integer-mode memory and register file would perform all relevant checks.

% i.e. isn't the module system nice for overriding specific behaviour like Capability-mode RV64I :)
Integrating capability instructions was also simple.
Two new ISA modules were created: \code{XCheri64} for the new CHERI instructions, and \code{Rv64imCapabilityMode} to override the behaviour of legacy instructions in capability-encoding-mode (see \cref{fig:module_algorithm}).
The actual Processor structure was left mostly unchanged.
Integer addresses were changed to capabilities throughout,
memory and register file types were changed as described above, and the PCC/DDC were added.
\pagebreak

\begin{figure}
    \centering
    \begin{minipage}[c]{.4\textwidth}
      \centering
      \includegraphics[width=\linewidth]{Figures/cheri_memory.pdf}
      \captionof{figure}{Emulator memory structure}
      \label{fig:emulatormemory}
    \end{minipage}\hfill%
    \begin{minipage}[c]{7.5cm}
        \centering
        {

        \small
      \begin{algorithmic}
        \If{new CHERI instruction}
            \State handle with \code{XCheri64}
        \ElsIf{basic \code{rv64} instruction}
            \If{in capability encoding mode}
                \State handle with \code{Rv64imCapabilityMode}
            \Else{}
                \State wrap memory with DDC-relative
                \State handle with \code{Rv64im}
            \EndIf{}
        \ElsIf{vector instruction}
            \If{in capability encoding mode}
                \State handle with vector unit
            \Else{}
                \State wrap memory in DDC-relative
                \State handle with vector unit
            \EndIf{}
        % \ElsIf{CSR instruction}
        %     \State handle with CSR module
        \EndIf{}
    \end{algorithmic}
        }
    \captionof{figure}{Example algorithm for emulating \code{rv64imvxcheri}}\label{fig:module_algorithm}
    \end{minipage}
\end{figure}

The capability model presented by the C/Rust library has one flaw.\label{safetaggedcap}
Each \code{CcxCap} instance stores capability metadata (e.g. the uncompressed bounds) as well as the compressed encoding.
This makes it potentially error-prone to represent untagged integer data with \code{CcxCap}, as the compressed and uncompressed data may not be kept in sync and cause inconsistencies later down the line.
\code{CcxCap} also provides a simple interface to set the tag bit, without checking whether that is valid.
The emulator introduced the \code{SafeTaggedCap} to resolve this: a sum type which represents either a \code{CcxCap} with the tag bit set, or raw data with the tag bit unset.
This adds type safety, as the Rust compiler forces every usage of \code{SafeTaggedCap} to consider both options, preventing raw data from being interpreted as a capability by accident and enforcing Provenance.

% i.e. doing capability relocation
The final hurdle was the capability relocations outlined in \cref{chap:bg:subsec:cherirelocs}.
Because we're emulating a bare-metal platform, there is no operating system to do this step for us.
A bare-metal C function has been written to perform the relocations\footnote{\gitfile{src/crt_init_globals.c}{CTSRD-CHERI/device-model}{https://github.com/CTSRD-CHERI/device-model/blob/88e5e8e744d57b88b0dbb8e3456ee0e69afc143b/src/crt_init_globals.c}}, which could be compiled into the emulated program.
We decided it would be quicker to implement this in the simulator, but
in the future we should be able to perform the relocations entirely in bare-metal C.

\subsection{Emulating vectors}
% i.e. using addr, provenance split to write agnostic code?

Vector instructions are executed by a Vector ISA module, which stores all registers and other state.
\code{VLEN} is hardcoded as 128-bits, chosen because it's the largest integer primitive provided by Rust that's large enough to hold a capability.
\code{ELEN} is also 128-bits, which isn't supported by the specification, but is required for capabilities-in-vectors (\cref{chap:capinvec}).
Scaling \code{VLEN} and \code{ELEN} any higher would require the creation and integration of new types that were more than 128-bits long.

To support both CHERI and non-CHERI execution pointers are separated into an address and a \emph{provenance}\footnote{The ``original allocation the pointer is derived from''\cite{memarianExploringSemanticsPointer2019}, or in CHERI terms the bounds within which the pointer is valid.}.
The vector unit retrieves an address + provenance pair from the base register, generates a stream of addresses to access, then rejoins each address with the provenance to access memory.
When using capabilities, provenance is defined in terms of the base register e.g. \enquote{the provenance is provided by capability register X}, or defined by the DDC in integer mode\footnote{See \cref{chap:emu:rvv_int_mode} for the reasoning behind this decision.}.
On non-CHERI platforms the vector unit doesn't check provenance.

Arithmetic and configuration instructions are generally simple to implement, so aren't covered here.
The emulator splits vector memory accesses into three phases: decoding, checking, and execution.
A separate decoding stage may technically not be necessary in hardware (especially the parts checking for errors and reserved instruction encodings, which a hardware platform could simply assume won't happen), but it allows each memory access instruction to be classified into one of the five archetypes outlined in \cref{chap:bg:sec:rvvmemory}.
It is then easy to define the checking and execution phases separately for each archetype, as the hardware would need to do.

\subsubsection{Decoding phase}\label{chap:hardware:subsec:decoding}
Decoding is split into two steps: finding the encoded \paramt{nf} and \paramt{eew} values, then interpreting them based on the encoded archetype.
% Vector memory accesses are encoded under the F extension's Load and Store major opcodes, which already encodes an element width.
% Vector memory access instructions are encoded similarly to the F extension's floating-point load/store instructions, which include an ``element width''.
Vector memory accesses reuse instruction encodings from the F extension's floating-point load/store instructions, which include an ``element width''.
The vector extension adds four extra ``element width'' values which imply the access is vectorized.
% This element width is used to differentiate between vector accesses and scalar floating-point accesses, where a vector access can have one of four widths (8, 16, 32, 64).
If any of these values are found, the instruction is interpreted as a vector access and \paramt{nf} is extracted.
% \paramt{nf} is encoded consistently in all vector memory access instructions

Once the generic parameters are extracted, the addressing method is determined from \paramt{mop} (Unit, Strided, Indexed-Ordered, or Indexed-Unordered).
If a unit access is selected, the second argument field \paramt{umop} selects a unit-stride archetype (normal access, fault-only-first, whole register, or bytemask).
% Strided and indexed accesses just use their dedicated archetypes.
Extra archetype-specific calculations are performed (e.g. computing \code{EVL = ceil(vl/8)} for bytemask accesses), and the relevant information is returned as a \code{DecodedMemOp} enum.

% \todomark{diagram of floating point ld/st vs. vector ld/st}
% \todomark{table of floating point ld/st, vector ld/st instructions}
\todomark{appx - Decision tree for operation decoding?}

\newcommand{\mcb}[2]{\multicolumn{#1}{|c|}{#2}}
\begin{table}
    % \begin{tabularx}{\textwidth}{@{}*{32}{p{\dimexpr(\textwidth-64\tabcolsep)/32\relax}}@{}}
    \small\begin{tabularx}{\textwidth}{XXXX XXXX XXXX XXXX XXXX XXXX XXXX XXXX}
    % \small\begin{tabularx}{\textwidth}{p{}}
                % \begin{tabular}{*{32}{p{3cm}}}
            % 31&&29&28&27&26&25&24&&&&20&19&&&&15&14&&12&11&&&&7&6&&&&&&0 \\
        \hline
        \mcb{3}{nf} & mew & \mcb{2}{\paramt{mop}} & vm & \mcb{5}{\paramt{umop}/rs2/vs2} & \mcb{5}{rs1} & \mcb{3}{width} & \mcb{5}{vd} & \mcb{7}{Opcode} \\
        \hline
    \end{tabularx}
    \caption{Floating-point and Vector load/store encoding}
    \todomark{fix}
\end{table}

\subsubsection{Fast-path checking phase}\label{chap:hardware:subsec:checking}
The initial motivation for this project was investigating the impact of capability checks on performance.
Rather than check each element's access invidually, we determine a set of \enquote{fast-path} checks which count as checks for multiple elements at once.
In the emulator, this is done by computing the \enquote{tight bounds} for each access, i.e. the exact range of bytes that will be accessed, and doing a single capability check with that bounds.
\cref{chap:hardware:sec:fastpath} describes methods for calculating the \enquote{tight bounds} for each access type, and ways that architectural complexity can be traded off to calculate \emph{wider} bounds.

% We investigate an approach using a \enquote{fast-path}, where certain instructions could check their whole access range against a capability immediately rather than check each individual element.
% Other approaches, such as optimizing a parallel checker for $n$ elements, were too hardware-specific and couldn't be modelled in software as well.
% \cref{chap:hardware:sec:fastpath} describes methods for calculating the \enquote{tight bounds} for each access type, i.e. the minimum range of bytes that must be accessible, and ways that architectural complexity can be traded off to calculate \emph{wider} bounds.

If the tight bounds don't pass the capability check, the emulator raises an imprecise trap and stops immediately.
In the case of fault-only-first loads, where synchronous exceptions (e.g. capability checks) are explicitly handled, the access continues regardless and elements are checked individually.
This is also the expected behaviour if a capability check for \emph{wider} bounds fails.
The emulator deviates from the spec in that \code{vstart} is \emph{not} set when the tight bounds check fails, as it does not know exactly which element would have triggered the exception.
As noted in \cref{chap:hardware:sec:fastpath}, a fully compliant machine must check each access to find \code{vstart} in these cases.

\subsubsection{Execution phase}\label{chap:hardware:subsec:execution}
If the fast-path check deems it appropriate, the emulator continues execution of the instruction in two phases.
First, the mapping of vector elements to accessed memory addresses is found.
The code for this step is independent of the access direction, and an effective description of how each type of access works.
It can be found in \cref{appx:code:vector_mem_access}.
The previously computed tight bounds are sanity-checked against these accesses, and the accesses are actually performed.

\subsubsection{Integer vs. Capability encoding mode\label{chap:emu:rvv_int_mode}}
As noted in \cref{chap:bg:subsec:cheriencodingmode} CHERI-RISC-V defines two execution modes that the program can switch between.
In Integer mode \enquote{address operands to existing RISC-V load and store opcodes contain integer addresses} which are implicitly dereferenced relative to the default data capability, and in Capability mode those opcodes are modified to use capability operands.
Integer mode was included in the interests of maintaining compatibility with legacy code that hasn't been adapted to capabilities.
As similar vector code may also exist, CHERI-RVV treats vector memory access instructions as \enquote{existing RISC-V load and store opcodes} and requires that they respect integer/capability mode.
\section{Fast-path calculations\label{chap:hardware:sec:fastpath}}
A fast-path check can be performed over various sets of elements.
The emulator chooses to perform a single fast-path check for each vector access, calculating the tight bounds before starting the actual access, but in hardware this may introduce prohibitive latency.
This section describes the general principles surrounding fast-paths for CHERI-RVV, notes the areas where whole-access fast-paths are difficult to calculate, and describes possible approaches for hardware.

\subsection{Possible fast-path outcomes}
In some cases, a failed address range check may not mean the access fails.
The obvious case is fault-only-first loads, where capability exceptions may be handled without triggering a trap.
Implementations may also choose to calculate wider bounds than accessed for the sake of simplicity, or even forego a fast-path check altogether.
Thus, a fast-path check can have four outcomes depending on the circumstances.

\begin{figure}[t]
\begin{subtable}{0.5\textwidth}
    \begin{tabular}{ll}
        \toprule
        Success & All accesses will succeed \\
        \midrule
        Failure & At least one access \emph{will}\\& raise an exception \\
        \midrule
        Likely-Failure & At least one access \emph{may} \\
        \emph{or} Unchecked & raise an exception \\
        \bottomrule
    \end{tabular}
    \caption{Possible fast-path outcomes}
\end{subtable}%
\hfill%
\begin{subfigure}{0.45\textwidth}
    \begin{algorithmic}
        \If{can calculate range}
            \If{range is within capability}
                \State Success
            \ElsIf{range is wide}
                \State Likely-Failure
            \ElsIf{fault-only-first}
                \State Likely-Failure
            \Else{}
                \State Failure
            \EndIf{}
        \Else{}
            \State Unchecked
        \EndIf{}
    \end{algorithmic}
    \caption{Algorithm}
\end{subfigure}
\caption{Fast-path outcomes}
\end{figure}

A Success means no per-access capability checks are required.
Likely-Failure and Unchecked results mean each access must be checked, to see if any of them actually raise an exception.
Unfortunately, accesses still need to be checked under Failure, because both precise and imprecise traps need to report the offending element in \code{vstart}\footnote{In very particular cases, e.g. unmasked unit-strided accesses where \code{nf = 1}, the capability bounds could be used to calculate what the offending element must have been. We believe this is too niche of a use case to investigate further, particularly given the complexity of the resulting hardware.}.

Because all archetypes may have Failure or Likely-Failure outcomes, hardware must provide a fallback slow-path for each archetype which checks/performs each access in turn.
In theory, a CHERI-RVV specification could relax the \code{vstart} requirement for imprecise traps, and state that all capability exceptions trigger imprecise traps.
In this case, only archetypes that produce Likely-Failure outcomes need the slow-path.
However, it is likely that for complexity reasons all masked accesses will use wide ranges, thus producing Likely-Failure outcomes and requiring slow-paths for all archetypes anyway.
Because the Likely-Failure and Failure cases require the slow-path anyway, computing the fast-path can only be worthwhile if Success is the common case.

\subsection{Whole-access fast-paths}\label{chap:hardware:subsec:wholeaccessfastpath}
It is technically possible to calculate a fast-path for the entirety of an access (see \cref{appx:fastpathfull}), but for some situations it may be equally/more expensive than checking each access.
For example, the bounds for masked accesses depend on finding the minimum and maximum active indices, which in hardware may require a linear scan.
Indexed accesses require finding the minimum/maximum offset values, which likely requires an expensive parallel reduction over all/some elements.
In these cases hardware implementations could defer to the slow-path on all masked/indexed accesses, or for masked accesses use the wider, unmasked bounds and generate Likely-Failure outcomes.
Unit and strided accesses are much easier to handle.

Arbitrarily strided accesses (which may have positive, negative, or zero-valued strides) are relatively simple to calculate.
After calculating the segment width (i.e. $\text{number of fields} * \text{element width}$) the full bounds just depends on the sign of the stride (\cref{eq:tightboundsstrided}).
Unit-stride accesses simplify this further, because the stride is equal to the segment width and guaranteed to be positive (\cref{eq:tightboundsunit}).

\newcommand{\vstart}{\code{vstart}}
% \newcommand{\vstartactive}{\code{vstart}_{\mathit{active}}}
\newcommand{\vstartactive}{\code{vstart}}
\newcommand{\evl}{\code{evl}}
% \newcommand{\evlactive}{\code{evl}_{\mathit{active}}}
\newcommand{\evlactive}{\code{evl}}
\newcommand{\baseaddr}{\code{base}}

\begin{mycapequ}[!ht]

\begin{equation}\label{eq:tightboundsstrided}
\baseaddr{}\ +\ \begin{cases}
        \code{[\vstartactive{} * \code{stride}, (\evlactive{} - 1) * \code{stride} + \code{nf} * \code{eew})} & \code{stride} \ge 0 \\
            
        \code{[(\evlactive{} - 1) * \code{stride}, \vstartactive{} * \code{stride} + \code{nf} * \code{eew})} & \code{stride} < 0
    \end{cases}
\end{equation}
\caption{Tight bounds for strided access}
\end{mycapequ}

\begin{mycapequ}[!ht]
\begin{equation}\label{eq:tightboundsunit}
    \baseaddr{}\ +\ \code{[\vstartactive{} * \code{nf} * \code{eew}, \evlactive{} * \code{nf} * \code{eew})}
\end{equation}
\caption{Tight bounds for unit-stride access}
\end{mycapequ}

Ultimately, the potential up-front latency seemed like a dealbreaker for this approach.
We turned our attention to fast-pathing smaller groups of elements.

\subsection{$m$-element known-range fast-paths}
A hardware implementation of a vector unit may be able to issue $m$ requests within a set range in parallel.
For example, elements in the same cache line may be accessible all at once.
In these cases, checking elements individually would either require $m$ parallel bounds checks, $m$ checks' worth of latency, or something inbetween.
In this subsection we consider a fast-path check for $m$ elements. 

% \todomark{pre-check: permissions}
Capability checks can be split into two steps: address-agnostic (e.g. permissions checks, bounds decoding) and address-dependent (e.g. bounds checks).
Address-agnostic steps can be performed before any bounds checking, and should add minimal start-up latency (bounds decoding must complete before the checks anyway, and permission checks can be performed in parallel).
Once the bounds are decoded the actual checks consist of minimal logic\footnote{Likely requires two arithmetic operations per element, for checking against the top and bottom bounds.}, so a fast-path must have very minimal logic to compete.

We first consider unit and strided accesses, and note two approaches.
First, one could amortize the checking logic cost over multiple sets of $m$ elements by operating in terms of cache lines.
Iterating through all accessed cache lines, and then iterating over the elements inside, allows the fast-path to hardcode the bounds width and do one check for multiple cycles of work (if cache lines contain more than $m$ elements).
Cache-line-aligned allocations benefit here, as all fast-path checks will be in-bounds i.e. Successful, but misaligned data is guaranteed to create at least one Likely-Failure outcome per access (requiring a slow-path check).
Calculating tight bounds for the $m$ accessed elements per cycle could address this.

% \todomark{unit/strided: by $m$ elements?}
For unit and strided accesses, the bounds occupied by $m$ elements is straightforward to calculate, as the addresses can be generated in order.
The minimum and maximum can then be picked easily to generate tight bounds.
An $m$-way multiplexer is still required for taking the minimum and maximum, because \code{evl} and \code{vstart} may not be $m$-aligned.
If $m$ is small, this also neatly extends to handle masked/inactive elements.
This may use less logic overall than $m$ parallel bounds checks, depending on the hardware platform\footnote{e.g. on FPGAs multiplexers can be relatively cheap.}, but it definitely uses more logic than the cache-line approach.
Clearly, there's a trade-off to be made.

Indexed fast-paths are more complicated, because the addresses are unsorted.
The two approaches above have different advantages for indexed accesses.
If the offsets/indices are spatially close, just not sorted, cache line checks may efficiently cover all elements.
An implementation could potentially cache the results, and refer back for each access, instead of trying to iterate through cache lines in order.
Otherwise a $m$-way parallel reduction could be performed to find the min and max, but that would likely take up more logic than $m$ comparisons.
This may be a moot point depending on the cache implementation though - if the $m$ accesses per cycle must be in the same cache line, and the addresses are spread out, you're limited to one access and therefore one check per cycle regardless.

In summary, there are fast-path checks that consume less logic than $m$ parallel checks in certain circumstances.
Even though a slow-path is always necessary, it can be implemented in a slow way (e.g. doing one check per cycle) to save on logic.
Particularly if other parts of the system rely on constraining the addresses accessed in each cycle, a fast-path check can take advantage of those constraints. 
\section{Going beyond the emulator}
The emulator is a single example of a conformant CHERI-RVV implementation, and does not exercise every part of the specification.
Four properties stand out:
\begin{itemize}
    \item The emulator assumes all element accesses are naturally aligned, but the spec allows misaligned accesses.
    \item The emulator doesn't consider multiple hardware threads, essentially assuming all accesses are atomic.
    \item Segments/elements are always accessed in order, despite the spec not enforcing ordering
    \item Imprecise traps are used for all exceptions - precise trap behaviour is not explored.
\end{itemize}
This section notes how relaxed access ordering and precise exceptions may affect the hardware in ways not previously explored.

\subsection{Misaligned accesses}
Implementations are allowed to handle vector accesses that are not aligned to the size of the element.
This support is independent of misaligned scalar access support, so if e.g. misaligned 64-bit scalar accesses are allowed, misaligned vector accesses of 64-bit elements do \emph{not} have to be allowed.

Changing the emulator to allow misaligned accesses of integer data would not have any impact on CHERI correctness.
Capability loads/stores must be aligned to \code{CLEN}\cite[Section~3.5.2]{TR-951}, and an implementation cannot change this.
Writing misaligned integer values across a \code{CLEN} boundary would need to make sure to zero the tag bit on both regions, but this applies to scalar implementations as much as vector ones.
Alignment only impacts CHERI-RVV to the extent that it impacts capabilities-in-vectors (\cref{chap:capinvec:hyp_load_store}).

\subsection{Atomicity of accesses/General memory model}
Vector memory instructions are specified to follow the RISC-V Weak Memory Ordering model\cite{specification-RVV-v1.0}\footnote{Behaviour under the Total Store Ordering extension hasn't been defined.}, although this model hasn't been fully explained in terms of vectors yet.
RVWMO defines a global order of \enquote{memory operations}: atomic operations that are either loads, stores, or both\cite[Chapter~14]{specification-RISCV-vol1-20191213}.
The RVWMO spec assumes all memory instructions create exactly one memory operation but calls out that once the vector model is formalized, vector accesses may be defined to create multiple operations.

The RVV spec states \enquote{vector misaligned memory accesses follow the same rules for atomicity as scalar misaligned memory accesses}, i.e. that misaligned accesses may be decomposed into multiple memory operations of any granularity\footnote{e.g. each byte could be written in a separate access.}.
This is the only mention of atomicity in that document.

Again, atomicity of integer data doesn't really impact the fusion of CHERI and RVV, as long as tag bits are correctly zeroed on all integer writes.
However, it does impact capabilities-in-vectors (\cref{chap:capinvec:hyp_load_store}).

\subsection{Relaxed access ordering and precise traps}
Ordering is only enforced insofar as it is observable.
The only instructions that are forced to perform their accesses in order are indexed-ordered accesses, which can be used to write to e.g. I/O regions where order matters, and instructions that trigger precise traps.
Precise traps require \code{vstart} to be set to a value such that all elements before \code{vstart} have completed their accesses, and all accesses on/after \code{vstart} have not completed or are idempotent.

If a vector memory access instruction is 1. not indexed-ordered and 2. guaranteed not to trigger a precise trap\footnote{Even instructions that \emph{would} trigger precise traps but are guaranteed not to throw an exception or respond to asynchronous interrupt may execute out of order.} then it may execute out of order.
This does not affect CHERI-RVV in any way.
\section{Testing hypotheses}

\hypsubsection{hyp:hw_cap_as_vec_mem_ref}{Feasibility}
% \subsubsection{In theory?}
This is true.
All vector memory access instructions index the scalar general-purpose register file to read the base address, and CHERI-RVV implementations can simply use this index for the scalar capability register file instead.
% rather than a general-purpose register.
This can be considered through the lens of adding CHERI to any RISC-V processor, and in particular adding Capability mode to adjust the behaviour of legacy instructions.
RVV instructions can have their behaviour adjusted in exactly the same way as the scalar memory access instructions.

That approach then scales to other base architectures that have CHERI variants.
For example, on Morello scalar Arm instructions were modified to use CHERI capabilities as memory references\todocite{There's definitely an Arm CHERI v8 document somewhere}, so one may simply try to apply those modifications to e.g. Arm SVE instructions.
This only works where Arm SVE accesses memory references in the same way as scalar Arm instructions did i.e. through a scalar register file.
Arm SVE has other addressing modes like \code{u64base}, which uses a vector as a set of 64-bit addresses\todocite{https://developer.arm.com/documentation/100891/0612/coding-considerations/using-sve-intrinsics-directly-in-your-c-code}, which require more specific attention.

% \subsubsection{In practice?}
% Yes.

% The practical hurdle for CHERI-RVV is defining the behaviour of vector instructions under Integer and Capability modes.
% The 

\hypsubsection{hyp:hw_cap_bounds_checks_amortized}{Fast-path checks}

\enquote{Cost} can be defined in multiple ways.
As with all concepts in hardware, implementing fast-path capability checks requires a trade-off between competing interests, which are each considered here.
As mentioned above, it is also assumed that successful accesses (i.e. those without any capability violations) are the common case.
Overall, it seems the key benefit of fast-path checks is power consumption.

\subsubsection*{Power - Better}
If the fast-path check succeeds, then no power needs to be wasted on capability checks for the remaining cycles of the access.
If the vector unit has its own dedicated capability check logic, it could even be clock gated to completely eliminate dynamic power.
This shows a clear benefit as long as the extra logic for bounds calculations uses less power than $n - 1$ capability checks.
Implementations which use large vectors or make careful use of the simplifications laid out in \todoref{fast-path} should fulfil this condition easily.

\begin{equation}
\begin{array}{lr}
    \mathit{slow-path} =& n\ \mathit{checks} \\
&\\
    \mathit{fast-path} =& 1\ \mathit{check} \\
     &+ \mathit{bounds}\ \mathit{logic} \\

\end{array}
\end{equation}

\subsubsection*{Area - Worse or negligible change}
No matter how you slice it, slow-path circuitry is \emph{always} necessary for a fully conforming implementation.
If the slow-path is always required, and always takes up area, then adding any circuitry for fast-path must require more area.
Elements from the slow-path, e.g. capability decoding units, may be shared with the fast-path, and any spare space in a slow-path unit could also be shared with the fast-path, so it may have a \emph{negligible} impact.
Crucially, adding a fast-path will never \emph{decrease} the area of a conformant design.

\subsubsection*{Throughput - No change}
Assuming the bounds calculations can be pipelined to the same clock speed as a capability check, putting them before a set of pipelined accesses should not affect the throughput of those accesses.
If the fast-path check is subdivided between different registers in a group, the fast-path check for one register should be performed in parallel with the accesses for other registers for maximum throughput.
There is currently no reason to believe the bounds calculations have to significantly decrease clock speed or throughput.

\subsubsection*{Latency - Worse}
% If the actual accesses require some pre-processing, e.g. an implementation decided to calculate all the addresses it would access before actually accessing the
If the fast-path check could be done in parallel with other memory accesses, then it would not affect latency.
Unfortunately, performing a memory access without checking if it's allowed first completely undermines the security model!
Under very particular circumstances, it could be tolerable: if the access is a read with no side-effects, and the read data would be thrown away on a capability violation, and side-channel attacks were impossible (i.e. no caches were present) or ignored, then an unauthorized read out-of-bounds \emph{technically} has no impact on security; but it's implausible that any architect on a CHERI design would accept this.

Unfortunately, the fast-path check must always block the memory access that depends on it, so the fast-path will always increase latency (unless a separate memory access is performed in parallel).

\todomark{where to note that capability checks count as a synchronous exception? centralized point for "here are the differences between RVV and CHERI-RVV?}
\todomark{I wrote a type-safe wrapper for capabilities}