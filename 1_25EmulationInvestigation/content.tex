\chapter{Hardware emulation investigation}
% In order to experiment with integrating CHERI and RVV, we implemented a RISC-V emulator in the Rust programming language named \enquote{riscv-v-lite}.
% This emulator can emulate three RISC-V ISAs: 
% \begin{itemize}
%     \item \code{rv32imv} - 32-bit + Multiply and Vector extensions
%     \item \code{rv64imv} - 64-bit + Multiply and Vector extensions
%     \item \code{rv64imvxcheri} - 64-bit + CHERI, Multiply, and Vector extensions
% \end{itemize}
% Each ISA's vector behaviour is tested with a set of \code{memcpy}-esque testbench programs.
% These are written in C, using RISC-V vector intrinsics where possible, and compiled with a variety of C compilers to examine compatibility.
% An experimental capabilities-in-vectors concept is also tested.

% \cref{chap:software:sec:emu} describes the emulator, focusing on how vector instructions are emulated.
% \cref{chap:software:sec:testbench} describes the testbench, the various techniques used to emit vector instructions on different compilers, and the results of each testbench for each architecture and compiler.
% \cref{chap:software:sec:capinvec} explains the experiments done with capabilities-in-vectors.

\todomark{introduction}

\section{Developing the emulator}\label{chap:software:sec:emu}
Rust was selected due to its memory safety and runtime speed, and provides language constructs that suit themselves well to emulation.
Rust Enums are equivalent to \enquote{variant} types in other languages like C++, which \todomark{}.
\todomark{something something match statements}.

\subsection{Emulating CHERI}

\subsubsection{\code{rust-cheri-compressed-cap}}
\todomark{128-bit values interop}

\subsubsection{CHERI-RISC-V ISA}

\subsubsection{Capability Relocations}

\subsubsection{Integer/Capability mode}

\subsubsection{Integrating into the emulator}
% i.e. using MemoryOf trait to make all memory addressable only by capabilities
% i.e. using addr, provenance split to understand

\subsection{Emulating vectors}

\subsubsection{Decoding phase}

\subsubsection{Fast-path check phase}

\subsubsection{Execution phase}


% \section{Developing the testbench}\label{chap:software:sec:testbench}

\subsection{Vectorized memcpy}

\subsubsection{Strip-mining paradigm?}

\subsection{Emitting vector instructions}
\subsubsection{Auto-vectorization}
\subsubsection{Intrinsics}
\subsubsection{Inline ASM}
\subsubsection{Comparison vs. Arm SVE}

\subsection{CHERI-Clang issues}
\subsubsection{Using CHERI-Clang from scratch}
% i.e. the things we had to do because CHERIbuild wasn't setting the arguments for us
\subsubsection{Updating RVV instructions to use capabilities}
\subsubsection{Allocating vectors on the stack}
\subsubsection{Experiments with ASM wrappers}

\subsection{Final testbench structure}

\subsection{Results}
% \section{Experimenting with capabilities-in-vectors}\label{chap:software:sec:capinvec}

\section{Going beyond the emulator}
The emulator is a single example of a conformant CHERI-RVV implementation, and does not exercise every part of the specification.
Two properties stand out:
\begin{itemize}
    \item Segments/elements are always accessed in order, despite the spec not enforcing ordering
    \item Imprecise traps are used for all exceptions - precise exception behaviour is not explored.
\end{itemize}
This section notes how relaxed access ordering and precise exceptions may affect the hardware in ways not previously explored.

\subsection{Precise exceptions}
\todomark{precise exceptions}

\subsection{Relaxed access ordering}
\todomark{relaxed access ordering}

\section{Testing hypotheses}
\todomark{Move the fast-path stuff here?}

\todomark{H-A - yes}
\todomark{H-B - yes, depending on hardware constraints. A fast-path that checks a single super-range may always be fallible, so hardware must always be prepared to do a slow per-element check}