\subsection{Emulating CHERI}

Manipulating CHERI capabilities securely and correctly is a must for any CHERI-enabled emulator.
Capability encoding logic is not trivial by any means, so the \code{cheri-compressed-cap} C library was re-used rather than implementing it from scratch.
Rust has generally decent interoperability with C, but some of the particulars of this library caused issues.

\subsubsection{\code{rust-cheri-compressed-cap}}
\code{cheri-compressed-cap} provides two versions of the library by default, for 64-bit and 128-bit capabilities, which are generated from a common source through extensive use of the preprocessor.
Each variant defines a set of preprocessor macros (e.g. the widths of various fields) before including two common header files \code{cheri\_compressed\_cap\_macros.h} and \code{cheri\_compressed\_cap\_common.h}.
The latter then defines every relevant structure or function based on those preprocessor macros.
For example, a function \code{compute_base_top} is generated twice, once as  \code{cc64\_decompress\_mem} returning \code{cc64\_cap\_t} and another time as \code{cc128\_decompress\_mem} returning \code{cc128\_cap\_t}.
Elegantly capturing both sets was the main challenge for the Rust wrapper.

\todomark{table of relevant structures/functions?}

One of Rust's core language elements is the Trait - a set of functions and \enquote{associated types} that can be \emph{implemented} for any type.
This gives a simple way to define a consistent interface: define a trait \code{CompressedCapability} with all of the functions from \code{cheri\_compressed\_cap\_common.h}, and implement it for two empty structures \code{Cc64} and \code{Cc128}.
In the future, this would allow the Morello versions of capabilities to be added easily.
A struct \code{CcxCap<T>} is also defined which uses specific types for addresses and lengths pulled from a \code{CompressedCapability}.
For example, the 64-bit capability structure holds a 32-bit address, and the 128-bit capability a 64-bit address.

128-bit capabilities can cover a 64-bit address range, and thus can have a length of $2^{64}$.
Storing this length requires 65-bits, so all math in \code{cheri\_compressed\_cap\_common.h} uses 128-bit length values.
C doesn't have any standardized 128-bit types, but GCC and LLVM provide so-called ``extension types'' which are used instead.
Although the x86-64 ABI does specify how 128-bit values should be stored and passed as arguments\cite{specification-x86-psABI-v1.0}, these rules do not seem consistently applied\footnote{See \url{https://godbolt.org/z/qj43jssr6} for an example.}.
This causes great pain to anyone who needs to pass them across a language boundary.

Rust explicitly warns against passing 128-bit values across FFI, and the Clang User's Manual even states that passing \code{i128} by value is incompatible with the Microsoft x64 calling convention\footnote{\gitfile[release/13.x]{clang/docs/UsersManual.rst:3384}{llvm/llvm-project}{https://github.com/llvm/llvm-project/blob/release/13.x/clang/docs/UsersManual.rst\#x86}}.
This could be resolved through careful examination: for example, on LLVM 128-bit values are passed to functions in two 64-bit registers\footnote{\gitfile[release/13.x]{clang/lib/CodeGen/TargetInfo.cpp:2811}{llvm/llvm-project}{https://github.com/llvm/llvm-project/blob/75e33f71c2dae584b13a7d1186ae0a038ba98838/clang/lib/CodeGen/TargetInfo.cpp\#L2811}}, which could be replicated in Rust by passing two 64-bit values.
For convenience, we instead rely on the Rust and Clang compilers using compatible LLVM versions and having identical 128-bit semantics.

\todomark{C code wasn't documented, Rust will be}
\todomark{move markdown documentation into rust :)}

The CHERI-RISC-V documentation contains formal specifications of all the new CHERI instructions, expressed in the Sail architecture definition  language\footnote{\gitrepo{rems-project/sail}{https://github.com/rems-project/sail}}.
These definitions are used in the CHERI-RISC-V formal model (\footnote{\gitrepo{CTSRD-CHERI/sail-cheri-riscv}{https://github.com/CTSRD-CHERI/sail-cheri-riscv}}), and require a few helper functions (see \cite[Chapter 8.2]{TR-951}).
To make it easier to port the formal definitions directly into the emulator the \code{rust-cheri-compressed-cap} library also provides those helper functions through a wrapper trait.

\todomark{documentation should be made available somewhere}

\subsubsection{Integrating into the emulator}
% i.e. using MemoryOf trait to make all memory addressable only by capabilities
Integrating capabilities into the emulator was relatively simple thanks to the modular emulator structure.
A capability-addressed memory type was created, which wraps a simple integer-addressed memory in logic which performs the relevant capability checks.
For integer encoding mode, a further integer-addressed memory type was created which wraps the capability addressed mode, where all integer addresses are bundled with the DDC before passing through to the capability-addressed memory.
Similarly, a merged capability register file type was created that exposed integer-mode and capability-mode accesses.
This layered approach meant code for basic RV64I operations did not need to be modified to handle CHERI at all - simply passing the integer-mode memory and register file would perform all relevant checks.

\todomark{diagram}

% i.e. isn't the module system nice for overriding specific behaviour like Capability-mode RV64I :)
Integrating capability instructions was also simple.
Two new ISA modules were created: \code{XCheri64} for the new CHERI instructions, and \code{Rv64imCapabilityMode} to override the behaviour of legacy instructions in capability-encoding-mode.
\todomark{show the program flow for using modules}
The actual Processor structure was left mostly unchanged.
Integer addresses were changed to capabilities throughout,
memory and register file types were changed as described above, and the PCC/DDC were added.

The capability model presented by the C/Rust library has one flaw.
Each \code{CcxCap} instance stores capability metadata (e.g. the uncompressed bounds) as well as the compressed encoding.
This makes it potentially error-prone to represent untagged integer data with \code{CcxCap}, as the compressed and uncompressed data may not be kept in sync and cause inconsistencies later down the line.
\code{CcxCap} also provides a simple interface to set the tag bit, without checking whether that is valid.
The emulator introduced the \code{SafeTaggedCap} to resolve this: a sum type which represents either a \code{CcxCap} with the tag bit set, or raw data with the tag bit unset.
This adds type safety, as the Rust compiler forces every usage of \code{SafeTaggedCap} to consider both options, preventing raw data from being interpreted as a capability by accident.

% i.e. doing capability relocation
The final hurdle was the capability relocations outlined in \cref{chap:bg:subsec:cherirelocs}.
Because we're emulating a bare-metal platform, there is no operating system to do this step for us.
A bare-metal C function has been written to perform the relocations\footnote{\gitfile{src/crt_init_globals.c}{CTSRD-CHERI/device-model}{https://github.com/CTSRD-CHERI/device-model/blob/88e5e8e744d57b88b0dbb8e3456ee0e69afc143b/src/crt_init_globals.c}}, which could be compiled into the emulated program.
However, I wasn't sure how to find the addresses of the generated relocations in C, so I performed the relocations in Rust by examining the compiled ELF file before starting emulation.
In future research it should be doable to perform the relocations entirely in bare-metal C.
