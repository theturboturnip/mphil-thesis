\section{Fast-path calculations\label{chap:hardware:sec:fastpath}}
Because CHERI, and indeed the vector extension, target all levels of computer architecture from embedded systems to cloud servers, it's important for fast-paths to be scalable and adjust to the implementation complexity.
To that end, we propose a method of generating the address range for accesses of each archetype, noting where architectural complexity can be traded off for tighter coverage.

\todomark{fast-path could be split up? i.e. for LMUL = 8, could execute a fast-path for each register in the group rather than all 8 at once}

\subsection{Possible fast-path outcomes}
In some cases, a failed address range check may not mean the access fails.
The obvious case is fault-only-first loads, where capability exceptions may be handled without triggering a trap.
Implementations may also choose to calculate wider bounds for the sake of simplicity, or even forego a fast-path check altogether.
Thus, a fast-path check can have three outcomes depending on the circumstances:
\begin{itemize}
    \item Success - All accesses will succeed
    \item Likely-Failure - At least one access \emph{may} raise an exception
    \item Failure - At least one access \emph{must} raise an exception
    \item Unchecked
\end{itemize}

\todomark{if an address range is calculated then: if capability contains it: SUCCESS else if it was wide or FoF: Likely-Failure else: FAILURE}
\todomark{if an address range isn't calculated: Unchecked}
\todomark{Put the above into an algorithm format}

A Success means no per-access capability checks are required.
Likely-Failure and Unchecked results mean each access must be checked, to see if any of them actually raise an exception.
Unfortunately, accesses still need to be checked under Failure, because both precise and imprecise traps need to report the offending element in \code{vstart}\footnote{In very particular cases, e.g. unmasked unit-strided accesses where \code{nf = 1}, the capability bounds could be used to calculate what the offending element must have been. We believe this is too niche of a use case to investigate further, particularly given the complexity of the resulting hardware.}.

Because all archetypes may have Failure or Likely-Failure outcomes, hardware must provide a fallback slow-path for each archetype which checks/performs each access in turn.
In theory, a CHERI-RVV specification could relax the \code{vstart} requirement for imprecise traps, and state that all capability exceptions trigger imprecise traps.
In this case, only archetypes that produce Likely-Failure outcomes need the slow-path.
However, it is likely that for complexity reasons all masked accesses will use wide ranges, thus producing Likely-Failure outcomes and requiring slow-paths for all archetypes anyway.
Because the Likely-Failure and Failure cases require the slow-path anyway, computing the fast-path can only be worthwhile if Success is the common case.

\newcommand{\vstart}{\code{vstart}}
\newcommand{\vstartactive}{\code{vstart}_{\mathit{active}}}
\newcommand{\evl}{\code{evl}}
\newcommand{\evlactive}{\code{evl}_{\mathit{active}}}
\newcommand{\baseaddr}{\mathit{base}}

\subsection{Masked accesses}
For all masked accesses, masked-out/inactive segments should not trigger capability exceptions.
Therefore, a tight bounds must include only the smallest and largest active segments.
These segments can be found by inspecting the mask vector: either checking each bit in turn or using parallel logic to find the lowest/highest set bits.
Care must be taken with these checks to ensure elements outside the range $[\textit{vstart}, \textit{evl})$ are not counted.

\begin{align}
    \vstartactive{} &= \min(i\ \forall\ \vstart{} \le i < \evl{}\ \text{where}\ \mathit{mask}[i] = 1) \\
    \evlactive{}    &= \max(i\ \forall\ \vstart{} \le i < \evl{}\ \text{where}\ \mathit{mask}[i] = 1) + 1
\end{align}

\subsubsection*{Tradeoffs}
If using parallel logic to find the lowest/highest bits, it could be difficult to account for $[\textit{vstart}, \textit{evl})$.
An implementation could choose to only calculate tight bounds when the mask is fully utilized, i.e. $\textit{vstart} = 0, \textit{evl} = \textit{VLEN}$, and assume wider bounds otherwise.

Accounting for masked accesses at all may not be worth the extra complexity.
Only elements masked off on the edges make any difference, and it may be uncommon for long runs of edge elements to be masked off.
Thus, an implementation could choose to ignore masking entirely when computing the ranges.
This does mean that all failures become Likely-Failure when masking is enabled, because all elements outside the capability bounds may be masked off.

\todomark{iterating over elements may still be more energy-efficient than doing individual capability checks?}

\subsection{Unit accesses}
For unit segmented accesses, which includes fault-only first, the tight address range for an access is simple to calculate.
Whole register and bytemask accesses can simplify this by fixing \code{nf = 1} and \code{eew = 8}.

\begin{equation}
    \baseaddr{} + [\vstartactive{} * \code{nf} * \code{eew}, \evlactive{} * \code{nf} * \code{eew})
\end{equation}
\todomark{Note that the equation is a simplification of strided for \code{stride = nf * eew}}

\subsubsection*{Tradeoffs}
\code{nf} is not guaranteed to be a power of two (except for the whole-register case), so calculating the `tight' address range would require a multiplication by an arbitrary four-bit value between 1 and 8.
If this multiplication is too expensive, implementations could choose to classify all $\code{nf} > 1$ cases as Unchecked.
% \todomark{if nf == 1; as normal; if nf != 1 Unchecked}

Unless extra restrictions are placed on \code{vstart}, calculating the start of this range requires another arbitrary multiplication.
To avoid this one could assume \code{vstart = 0} and treat failures as Likely-Failure for other cases.
%for the calculation, perform the capability check and treat failures as Likely-Failure (because the failure could be due to an element before \code{vstart}).
% \todomark{make it clear that in this case vstart = 0 would still be a complete failure}
Once could also classify all nonzero \code{vstart} accesses as Unchecked.
% \todomark{if vstart == 0 failure = failure; if vstart != 0 all = Likely-Failure}

Even if the previous two optimizations are applied, the final range still requires a multiplication $\evl{} * \code{eew}$.
Thankfully, because \code{eew} may only be one of four powers-of-two, this can be encoded as a simple shift.

\subsection{Strided accesses}
Strided accesses bring further complication, especially as the stride may be negative.

\begin{equation}
\mathit{bounds}(\code{stride}) = \baseaddr{} + \left\{
    \begin{array}{ll}
          [\vstartactive{} * \code{stride}, (\evlactive{} - 1) * \code{stride} + \code{nf} * \code{eew}) & \code{stride} \ge 0 \\
          
          [(\evlactive{} - 1) * \code{stride}, \vstartactive{} * \code{stride} + \code{nf} * \code{eew}) & \code{stride} < 0
    \end{array} 
\right.
\end{equation}

This is formed of three components:
\begin{itemize}
    \item $\vstartactive{} * \code{stride}$, the start of the first segment. This can be simplified to 0, just like for unit accesses, to avoid an arbitrary multiplication.
    \item $(\evlactive{} - 1) * \code{stride}$, the start of the final segment. This requires an arbitrary multiplication, unless strided accesses are all Unchecked.
    \item $\code{nf} * \code{eew}$, the length of a segment, which can be implemented with a shift.
\end{itemize}
% Because the stride is not guaranteed to be larger than a segment (or even larger than a single element)

% \todomark{stride may be negative}

\subsection{Indexed accesses}
This is the most complicated access of the bunch, because the addresses cannot be computed without reading the index register.

\begin{align}
    [&\baseaddr{}\ +\ \min(\code{offsets}[\vstartactive{}..\evlactive{}]), \\
    &\baseaddr{}\ +\ \max(\code{offsets}[\vstartactive{}..\evlactive{}])\ +\ \code{nf} * \code{eew})
\end{align}

The most expensive components here are of course $min,\,max$ of the offsets.
These could be calculated in hardware through parallel reductions, making it slightly more efficient than looping over each element.
A low-hanging optimization could be to remove the $\vstartactive{}..\evlactive{}$ range condition, performing the reduction over the whole register group, which would make failures Likely-Failure where $\vstartactive{}\ !=\ 0\ ||\ \evlactive{}\ !=\ \code{VLMAX}$.
This calculation could also be restricted to certain register configurations to reduce the amount of required hardware.
Indeed, the amount of hardware could be reduced to zero by simply classifying all indexed accesses as Unchecked.