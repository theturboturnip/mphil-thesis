\section{Fast-path calculations}
Because CHERI, and indeed the vector extension, target all levels of computer architecture from embedded systems to cloud servers, it's important for fast-paths to be scalable and adjust to the implementation complexity.
To that end, we propose a method of generating the address range for accesses of each archetype, noting where architectural complexity can be traded off for tighter coverage.

\subsection{Possible fast-path outcomes}
In some cases, a failed address range check may not mean the access fails.
The obvious case is fault-only-first loads, where synchronous exceptions (such as capability checks) are explicitly handled.
The address range calculations ignore masking in all cases\footnote{Shrinking the address range to remove masked-out elements would likely require a linear search through all elements, which makes the fast-path redundant.} so even if the address range falls outside the capability bounds, the individual accesses outside those bounds could all be masked out.
Therefore a fast-path check can have three outcomes:
\begin{itemize}
    \item Success - All accesses will succeed
    \begin{itemize}
        \item The fast-path address range is entirely contained by the capability
    \end{itemize}
    \item Indeterminate - At least one access \emph{may} fail
    \begin{itemize}
        \item The fast-path address range could not be calculated, or is not entirely contained by the capability for fault-only-first accesses.
    \end{itemize}
    \item Failure - At least one access \emph{must} fail, triggering a synchronous exception.
    \begin{itemize}
        \item The fast-path address range is not entirely contained by the capability
        \item Not a fault-only-first access
    \end{itemize}
\end{itemize}

\todomark{tight vs superset range?}

In the case of indeterminate failure, accesses need to be individually checked to see if failure actually occurs.
If precise exceptions are used, even the failure case needs to perform each individual access to ensure accesses before the offending element are completed.
Because all access archetypes may be masked, therefore the fast-path check may always be indeterminate, implementations must always contain a slow-path fallback option which does each access in turn for every archetype.
% This slow-path 
% In cases where precise exceptions are used, this slow-path must also respect the necessary ordering guarantees.
% \todomark{this isn't a good place to talk about precise exceptions/ordering}

% If masking is enabled, any elements that cause the address range check to fail could be masked out.
% In all cases masking is ignored - masked-out elements should not trigger capability checks or capability failure \todomark{why?} but checking 

\noindent\emph{Unit accesses}\\
\noindent For unit segmented accesses, which includes fault-only first, the tight address range for an access is
$base + [vstart * nf * eew, evl * nf * eew)$.
Unit unsegmented accesses where $nf = 1$, which includes whole register and bytemask accesses, can simplify this by fixing \code{nf} and \code{eew}.

\code{nf} is not guaranteed to be a power of two (except for the whole-register case), so calculating the `tight' address range would require a multiplication by an arbitrary four-bit value between 1 and 8.
If this multiplication is too expensive, implementations could choose to classify all $nf > 1$ cases as indeterminate.
\todomark{if nf == 1; as normal; if nf != 1 indeterminate}

Unless extra restrictions are placed on \code{vstart} calculating the start of this range requires another arbitrary multiplication.
To avoid this, one could assume \code{vstart} is zero for the calculation, perform the capability check and treat failures as indeterminate (because the failure could be due to an element before \code{vstart}).
\todomark{make it clear that in this case vstart = 0 would still be a complete failure}
Once could also classify all nonzero \code{vstart} accesses as indeterminate immediately, without performing any calculations.
\todomark{if vstart == 0 failure = failure; if vstart != 0 all = indeterminate}

Even if the previous two optimizations are applied, the final range still requires a multiplication $evl * eew$.
Thankfully, because \paramt{eew} may only be one of four powers-of-two, this can be encoded as a simple shift.


\noindent\emph{Strided accesses}\\
\noindent Strided accesses bring further complication, especially as the stride may be negative.
For a $stride >= 0$ : $base + [vstart * stride, (evl - 1) * stride + nf * eew)$
For $stride < 0$: $base + [(evl - 1) * stride, vstart * stride + nf * eew)$

This is formed of three components:
\begin{itemize}
    \item $vstart * stride$, the start of the first segment. This can be simplified as 0, just like for unit accesses, to avoid an arbitrary multiplication.
    \item $(evl - 1) * stride$, the start of the final segment. This requires an arbitrary multiplication, unless strided accesses are not fast-pathed at all.
    \item $nf * eew$, the length of a segment, which can be implemented with a shift.
\end{itemize}
% Because the stride is not guaranteed to be larger than a segment (or even larger than a single element)

% \todomark{stride may be negative}

\noindent\emph{Indexed accesses}\\
\noindent This is the most complicated access of the bunch, because the addresses cannot be computed without reading the index register.

$base + [min(offsets[vstart..evl]), max(offsets[vstart..evl]) + nf * eew)$

The most expensive components here are of course $min,max$ of the offsets.
These could be calculated in hardware through parallel reductions, making it slightly more efficient than looping over each element.
A low-hanging optimization could be to remove the $vstart..evl$ range condition, performing the reduction over the whole register group, which would make failures indeterminate where $vstart != 0 || evl != VLMAX$.
This calculation could also be restricted to certain register configurations to reduce the amount of required hardware.
Indeed, the amount of hardware could be reduced to zero by simply classifying all indexed accesses as indeterminate.