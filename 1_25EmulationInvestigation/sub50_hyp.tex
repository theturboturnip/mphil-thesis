\section{Testing and evaluation}

We tested the emulator using a set of test programs described in \cref{chap:software:eval,chap:capinvec:eval}, and found that all instructions were implemented correctly.

\hypsubsection{hyp:hw_cap_as_vec_mem_ref}{Feasibility}
This is true.
All vector memory access instructions index the scalar general-purpose register file to read the base address, and CHERI-RVV implementations can simply use this index for the scalar capability register file instead.
This can be considered through the lens of adding CHERI to any RISC-V processor, and in particular adding Capability mode to adjust the behaviour of legacy instructions.
RVV instructions can have their behaviour adjusted in exactly the same way as the scalar memory access instructions.

That approach then scales to other base architectures that have CHERI variants.
For example, Morello's scalar Arm instructions were modified to use CHERI capabilities as memory references\cite[Section 1.3]{armltdMorelloArchitectureReference2021}, so one may simply try to apply those modifications to e.g. Arm SVE instructions.
This only works where Arm SVE accesses memory references in the same way as scalar Arm instructions did i.e. through a scalar register file.

Arm SVE has some addressing modes like \code{u64base}, which uses a vector as a set of 64-bit integer addresses\cite{armltdArmCompilerScalable2019}.
This has more complications, because simply dereferencing integer addresses without a capability is insecure.
Would a CHERI version convert this mode to use capabilities-in-vectors, breaking compatibility with legacy code that expects integer references?
Another option would be to only enable this instruction in Integer mode, and dereference relative to the DDC.
It's possible to port this to CHERI, but requires further investigation and thought.

\hypsubsection{hyp:hw_cap_bounds_checks_amortized}{Fast-path checks}
This is also true, at least for Successful accesses.
Because the RVV spec requires that the faulting element is \emph{always} recorded\cite[Section 17]{specification-RVV-v1.0}, a Failure due to a capability violation requires elements to be checked individually.
CHERI-RVV could change the specification so the faulting element doesn't need to be calculated, which would make Failures faster, but that still requires Likely-Failures to take the slow-path.

There are many ways to combine the checks for a set of vector elements, which can take advantage of the range constraints.
For example, a unit-stride access could a hierarchy of checks: cache-line checks until a Likely-Failure, then tight $m$-element bounds until a Likely-Failure, then the slow-path.
However, the choice of fast-path checks is inherently a trade-off between latency, area, energy usage, and more.
Picking the right one for the job is highly dependent on the existing implementation, and indeed an implementation may decide that parallel per-element checks is better than a fast-path.
