\section{Testing hypotheses}

\hypsubsection{hyp:hw_cap_as_vec_mem_ref}{Feasibility}
% \subsubsection{In theory?}
This is true.
All vector memory access instructions index the scalar general-purpose register file to read the base address, and CHERI-RVV implementations can simply use this index for the scalar capability register file instead.
% rather than a general-purpose register.
This can be considered through the lens of adding CHERI to any RISC-V processor, and in particular adding Capability mode to adjust the behaviour of legacy instructions.
RVV instructions can have their behaviour adjusted in exactly the same way as the scalar memory access instructions.

That approach then scales to other base architectures that have CHERI variants.
For example, on Morello scalar Arm instructions were modified to use CHERI capabilities as memory references\todocite{There's definitely an Arm CHERI v8 document somewhere}, so one may simply try to apply those modifications to e.g. Arm SVE instructions.
This only works where Arm SVE accesses memory references in the same way as scalar Arm instructions did i.e. through a scalar register file.
Arm SVE has other addressing modes like \code{u64base}, which uses a vector as a set of 64-bit addresses\todocite{https://developer.arm.com/documentation/100891/0612/coding-considerations/using-sve-intrinsics-directly-in-your-c-code}, which require more specific attention.

% \subsubsection{In practice?}
% Yes.

% The practical hurdle for CHERI-RVV is defining the behaviour of vector instructions under Integer and Capability modes.
% The 

\hypsubsection{hyp:hw_cap_bounds_checks_amortized}{Fast-path checks}

\todomark{In a theoretical sense, definitely - $n$ individual checks can be replaced by 1 check, amortizing whatever the cost is}
\todomark{Evaluate cost in a hardware sense: compare a theoretical implementation which does each per-element check, potentially $m$ checks in parallel to issue $m$ requests in parallel, all pipelined with the actual accesses, vs. hardware that has a single capability check at the start which blocks it from initiating any accesses.}

\enquote{Cost} can be defined in multiple ways.
As with all concepts in hardware, implementing fast-path capability checks requires a trade-off between competing interests, which are each considered here.
As mentioned above, it is also assumed that successful accesses (i.e. those without any capability violations) are the common case.
Overall, it seems the key benefit of fast-path checks is power consumption.

\subsubsection*{Power - Better}
If the fast-path check succeeds, then no power needs to be wasted on capability checks for the remaining cycles of the access.
If the vector unit has its own dedicated capability check logic, it could even be clock gated to completely eliminate dynamic power.
This shows a clear benefit as long as the extra logic for bounds calculations uses less power than $n - 1$ capability checks.
Implementations which use large vectors or make careful use of the simplifications laid out in \cref{chap:hardware:sec:fastpath} should fulfil this condition easily.

\begin{equation}
\begin{array}{lr}
    \mathit{slow-path} =& n\ \mathit{checks} \\
&\\
    \mathit{fast-path} =& 1\ \mathit{check} \\
     &+ \mathit{bounds}\ \mathit{logic} \\

\end{array}
\end{equation}

\subsubsection*{Area - Worse or negligible change}
No matter how you slice it, slow-path circuitry is \emph{always} necessary for a fully conforming implementation.
If the slow-path is always required, and always takes up area, then adding any circuitry for fast-path must require more area.
Elements from the slow-path, e.g. capability decoding units, may be shared with the fast-path, and any spare space in a slow-path unit could also be shared with the fast-path, so it may have a \emph{negligible} impact.
Crucially, adding a fast-path will never \emph{decrease} the area of a conformant design.

\subsubsection*{Throughput - No change}
Assuming the bounds calculations can be pipelined to the same clock speed as a capability check, putting them before a set of pipelined accesses should not affect the throughput of those accesses.
If the fast-path check is subdivided between different registers in a group, the fast-path check for one register should be performed in parallel with the accesses for other registers for maximum throughput.
There is currently no reason to believe the bounds calculations have to significantly decrease clock speed or throughput.

\subsubsection*{Latency - Worse}
% If the actual accesses require some pre-processing, e.g. an implementation decided to calculate all the addresses it would access before actually accessing the
If the fast-path check could be done in parallel with other memory accesses, then it would not affect latency.
Unfortunately, performing a memory access without checking if it's allowed first completely undermines the security model!
Under very particular circumstances, it could be tolerable: if the access is a read with no side-effects, and the read data would be thrown away on a capability violation, and side-channel attacks were impossible (i.e. no caches were present) or ignored, then an unauthorized read out-of-bounds \emph{technically} has no impact on security; but it's implausible that any architect on a CHERI design would accept this.

Unfortunately, the fast-path check must always block the memory access that depends on it, so the fast-path will always increase latency (unless a separate memory access is performed in parallel).