%%%%%%%%%%%%%%%%%%%%%%%%%%%%%%%%%%%%%%%%%%%%%%%%%%%%%%%%%%%%%%%%%%%%%%%%%%%%%%%%
%% Thesis Meta-Information
%%

%% The title of the thesis:
\title{Capability-Based Memory Protection for Scalable Vector Processing}

%% The full name of the author (e.g.: James Smith):
\ifdefined\turnipanon
\author{Anonymous}
\college{Anonymous}
\collegeshield{}
\else
\author{Samuel Stark}
\college{Clare College}
\collegeshield{CollegeShields/Clare}
\fi

%% Submission date (optional):
\submissiondate{June 6th 2022}

%% You can redefine the submission notice:
% \submissionnotice{A badass thesis submitted on time for the Degree of PhD}

%% Declaration date:
\date{June 6th 2022}

%% Word count
\begin{luacode*}
wordcount_cmd = "./texcount.pl -1 -sum -merge -q chapters.tex"

local pip=io.popen(wordcount_cmd)
wordcount=pip:read()
pip:close()
    
-- From http://www.computercraft.info/forums2/index.php?/topic/8065-lua-thousand-separator/
function format_thousand(v)
    local s = string.format("%d", math.floor(v))
    local pos = string.len(s) % 3
    if pos == 0 then pos = 3 end
    return string.sub(s, 1, pos)
    .. string.gsub(string.sub(s, pos+1), "(...)", ",%1")
end
\end{luacode*}
\newcommand{\wordcountdata}{
{
\setlength{\parindent}{0pt}
\setlength{\parskip}{\baselineskip}

Total Page Count:~\pageref*{LastPage}

Main chapters (excluding front-matter, references, appendix): \pagedifference{wc:start}{wc:end} (pp~\pageref{wc:start}-\pageref{wc:end})

Main chapters word count: \luadirect{tex.sprint(format_thousand(wordcount))} words

Word count methodology: \texttt{\luadirect{tex.sprint(wordcount_cmd)}}\\
TeXcount version: 3.2.0.41\\
Counts body text, heading text, and caption text. Does not include text in tables, figures, code listings, or appendices.
The anonymous version redacts some trivially identifying information, but uses the same word count as the de-anonymized version.
The appendices contain non-essential data that is summarized in the main text (e.g. full results, details on how to compile code), or re-statements of work done in the main text (e.g. a summary of all changes made to the CHERI-RISC-V spec).
}
}

%% PDF meta-info:
\subjectline{Computer Science}
